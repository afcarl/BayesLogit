\documentclass[11pt]{article}

\input{commands}

\usepackage{natbib}

\newcommand{\Polya}{P\'{o}lya}
\newcommand{\PG}{\text{PG}}
\newcommand{\Pois}{\text{Pois}}
\newcommand{\Ga}{\text{Ga}}
\newcommand{\NB}{\text{NB}}
\newcommand{\floor}[1]{\lfloor #1 \rfloor}
\newcommand{\eqd}{\,{\buildrel d \over =}\,}
\newcommand{\KS}{\text{KS}}
\newcommand{\bbeta}{\boldsymbol{\beta}}

\newcommand{\point}{\noindent $\bullet$ }

% Page Layout
\setlength{\oddsidemargin}{0.0in}
\setlength{\textwidth}{6.5in}
\setlength{\textheight}{8in}
%\parindent 0in
%\parskip 12pt

% Set Fancy Style
%\pagestyle{fancy}

%\lhead{Left Header}
%\rhead{Right Header}

\title{\Polya-Gamma Data Augmentation for Dynamic Models}
\author{Carlos M. Carvalho, James G. Scott, Liang Sun, Jesse Windle}

\begin{document}

\maketitle

\abstract{ Dynamic linear models with Gaussian observations and Gaussian states
  lead to closed-form formulas for posterior simulation.  However, these
  closed-form formulas break down when the response or state evolution ceases to
  be Gaussian.  Dynamic, generalized linear models (DGLMs) exemplify a class of
  models for which this is the case, and include, amongst other models, dynamic
  binomial logistic regression and dynamic negative binomial regression.  It is
  important to appraise the preeminent posterior simulation techniques for these
  models as they find use across a range of fields, such as ecology,
  epidimeology, and neuroscience.  In this paper, we compare the performance of
  the \Polya-Gamma data augmentation technique \cite{polson-etal-2012} with the
  discrete mixture of normals approach based of
  \cite{fruhwirth-schnatter-etal-2009} and \cite{fussl-etal-2013}, and the
  Metropolis-Hastings based approach of \cite{ravines-etal-2006}.  We find that
  the \Polya-Gamma approach works well for dynamic binomial logistic regression
  when the number of trials for each response is small, and that the technique
  of \cite{fruhwirth-schnatter-etal-2009} works well for dynamic negative
  binomial regression.}

\tableofcontents

\newpage

\section{Introduction}

Generalized linear models (GLMs) consider responses, $y_t \sim P(\psi_t)$, with
expectations that depend on the linear combination of predictors $\psi_t = x_t'
\beta$ in a (usually) non-linear way \citep{mccullagh-nelder-1989}.  Extensions
to the dynamic case are straightforward via a dynamic regression coefficient:
$\psi_t = x_t' \beta_t$, where one must specify the dynamics of $\beta_t$ to
complete the model.  Unlike Gaussian, linear models, most likelihoods found in
generalized linear models lead to inconvenient posterior distributions for
$\beta$ or $\{\beta_t\}_{t=0}^T$.  Thus, when generating posterior samples one
must appeal to Metropolis-Hastings, Gibbs sampling with data augmentation, or
some other MCMC method.

The challenge of generating posterior samples is exacerbated for dynamic models
since there are many more unknown quantities to estimate than in the static
case, which makes, for instance, direct extensions of Metropolis-Hastings
samplers difficult.  This is the case for GLMs with binomial likelihoods, a
subclass of GLMs that have probability mass functions of the form
\[
p(y | \psi) \propto h(y) \prod_{i=1}^n
\frac{(e^{\psi_i})^{a_i(y)}}{(1+e^{\psi_i})^{b_i(y)}} \, ,
\]
and for which \Polya-Gamma data augmentation is applicable.  Such likelihoods
arise when modeling proportions on the log-odds scale, a common choice for
categorical or count data.  Two ubiquitous models, binomial logistic regression and
negative binomial regression, fit within this framework.  Dynamic versions of
these models are used in economics [McFadden], epidemiology [cite Lauren],
ecology [cite Lauren], and neuroscience [cite Jonathan and Larry] amongst
others.  Hence, it is important to know the most efficient methods for
generating posterior samples for such models.

This paper examines the relative merits of three approaches for posterior
inference in dynamic generalized linear models (DGLMs) with binomial
likelihoods: conjugate updating and backward sampling with a Metroplis-Hastings
step \citep{ravines-etal-2006}, forward filter and backwards sampling using data
augmentation and a discrete mixture of normals approximation
\citep{fruhwirth-schnatter-etal-2009, fussl-etal-2013}, and forward filter and
backwards sampling using the \Polya-Gamma data augmentation scheme.  We show
that the \Polya-Gamma technique always surpases the other techniques in terms of
effective sample size and often surpasses the other techniques in terms of
effective sampling rate.  In particular, the \Polya-Gamma method is the best
choice for dynamic binomial logistic regression when the number of trials is
small or dynamic negative binomial regression when the mean response is small.
This is because pace at which \Polya-Gamma random variates are generated depends
on the type and average size of the response.

The outline of the paper is as follows: Section 2 reviews other methods of
inference for dynamic regression for exponential families; Section 3 shows how
one may use the \Polya-Gamma data augmentation technique in the dynamic setting;
Section 4 presents the benchmarking results; Section 5 concludes.

\section{Previous Efforts}

Bayesian inference for dynamic generalized linear models dates back to at least
\cite{west-etal-1985} who used conjugate updating with backwards sampling (CUBS)
to sample the dynamic regression coefficients of DGLMs when the observation
$(y_i | \psi_i)$ comes from an exponential family; but their method is only
approximate.  Much effort has been devoted to developing exact posterior
samplers, though none has proved to be completely satisfactory.  A primary goal
of any such sampler is to sample states jointly, like the filter forward
backwards sampler (FFBS) of \cite{fruhwirth-schnatter-1994} and
\cite{carter-kohn-1994}, since jointly sampling states usually results in less
autocorrelation than sampling the states component-wise, an approach suggested
by \cite{carlin-etal-1992} prior to the advent of the advent of the FFBS.
However, the FFBS procedure requires Gaussian, linear state-space evolution
equations and observation equations.  Outside of these assumptions, as is
generally the case with DGLMs, the machinery of the FFBS breaks down.  To
resurrect the FFBS, one may approximate the posterior with some convenient
proposal density and then accept or reject using Metropolis-Hastings, or one may
use data augmentation so that, conditionally, the observations and states are
generated by a DLM.  Neither method is guaranteed to work well.

\cite{gamerman-1998} discusses various Metropolis-Hastings based approaches, all
of which rely on the Laplace approximation [CITE], and all of which are
flawed: component-wise proposals have decent acceptance rates, though
high autocorrelation between consecutive samples, while joint proposals suffer
from unacceptably small acceptance rates.  Gamerman's solution is to transform
the problem so that one samples the state disturbances componenent-wise using a
Laplace approximation with Metropolis-Hastings update, arguing that the new
coordinate system possesses less intrinsic correlation.  But this approach is
more computationally intensive since one must transform the proposals back to
the original coordinate system at each iteration to evaluate Metropolis-Hastings
acceptance probability.

\cite{shephard-pitt-1997} attempt to strike a balance between the
autocorrelation of consecutive samples and the acceptance probabilities of
proposed samples by drawing blocks of disturbances.  One can proceed similarly
by sampling blocks of states.  Sampling in larger blocks reduces Markov Chain
autocorrelation, while sampling smaller blocks encourages a reasonable
Metropolis-Hastings acceptance probability.  However, sampling in blocks can
still suffer from high autocorrelation between consecutive draws.  Sampling
blocks of the disturbances presents similar problems to those discussed above,
in particular, it is more computationally intensive than sampling blocks of
states directly.

More recently, techniques have emerged that generate joint draws of the entire
set of states.  \cite{ravines-etal-2006} built upon \cite{west-etal-1985} by
adding a Metropolis-Hastings step to sample the states exactly.  Though this at
first would seem like a poor choice due to the high-dimensionality often
encountered in time series, they find that, in fact, the technique results in
reasonable acceptance rates unlike a global Laplace approximation.  We verified
this assertion, implementing a Metropolis-Hastings scheme that sampled blocks of
disturbances using the Laplace approximation and finding that it did not fare as
well as the method of \cite{ravines-etal-2006}.
% The fact that conjugate
% update improves the Metropoli-Hastings proposal enough to allow for
% ``efficient'' joint draws is somewhat surprising.

Fr\"{u}hwirth-Schnatter and her colleagues have explored data augmentation
techniques that rely upon discrete mixtures of normals to arrive at
conditionally Gaussian posteriors in a variety of settings including binomial
logistic regression and multinomial regression
\citep{fruhwirth-schnatter-fruhwirth-2007, fruhwirth-schnatter-fruhwirth-2010,
  fussl-etal-2013}, Poisson regression \citep{fruhwirth-schnatter-wagner-2006,
  fruhwirth-schnatter-etal-2009}, and negative binomial regression for count
data \citep{fruhwirth-schnatter-etal-2009}.  All of these techniques may be used
in static and dynamic regressions.  While these methods work well, several rely
upon precomputing large tables of weights, means, and variances for the
components of a collection normal mixtures that approximate an entire family of
distributions.  Further all of the discrete mixture of normal techniques make
use of at least two layers of auxiliary variables.  One would prefer to avoid
many layers of latents since this may inhibit traversing the posterior landscape
and enlarges the memory footprint when storing the latent states.

Much of Fru\"{u}hwirth-Schnatter et al.'s work has been devoted to binomial
logistic regression. Initially, \cite{fruhwirth-schnatter-fruhwirth-2007}
employed the latent utilities interpretation of \cite{mcfadden-1974}.  Later,
\citep{fruhwirth-schnatter-fruhwirth-2010} found a discrete mixture of normals
approach based upon \cite{holmes-held-2006} to be superior. The 2010 paper
includes an extensive comparison of various data augmentation and
Metropolis-Hastings methods in the static case, including evidence that the
discrete mixture of normals approximation outperforms \cite{holmes-held-2006}
due to its computation efficiency.  Most recently, \cite{fussl-etal-2013} put
forth another posterior sampler for binomial logistic regression, using tables
of discrete mixtures of normals, that improves inference when the number of
trials at each observation is greater than one.  Much of this work can be traced
back the \cite{albert-chib-1993}, who develop a data augmentation approach to
probit regression with a straightforward dynamic analog.  We limit our
comparison below to \cite{fussl-etal-2013} since it appears to be the best
choice for dynamic binomial logistic regression within Fr\"{u}wirth-Schnatter et
al.'s cornucopia of methods.

DGLMs can be cast within the more general framework of non-linear, non-Gaussian
state-space models.  \cite{geweke-tanizaki-2001} highlight the various works of
Kitagawa, Tanizaki, and Mariano, amongst others, for filtering, smoothing, or
simulating states using numerical integration, resampling, or rejection sampling
within this context.  However, the more general setting does not provide more
insight into how one may jointly samples states in DGLMs.  Each of the
approaches they review is unsatisfactory: numerical integration does not work
well for any but the simplest settings, sampling marginally smoothed states
using sequential methods is time consuming, and rejection sampling may have poor
acceptance probabilities.  None of the methods cited by Geweke and Tanizaki are
useful for generating posterior samples of the states jointly, an extremely
desirable property.  Their solution is to sample the states component-wise using
a Laplace approximation and a Metropolis-Hastings step, which in the case of
exponential families returns us to the methods discussed by
\citep{gamerman-1998}.  \cite{godsill-etal-2004} show how one may jointly sample
states using particle methods; however, joint sampling using particle methods is
far less efficient than generating filtered samples and is not considered
herein.

\section{\Polya-Gamma Data Augmentation}

The \Polya-Gamma data augmentation technique \citep{polson-etal-2013} has a
single layer of latent variables and yields conditionally Gaussian posterior
distributions when working with binomial likelihoods.  DGLMs with binomial
likelihoods have probability mass functions of the form
\[
p(y | \bbeta) = c(y) \Big[ \prod_{i=1}^n
\frac{(e^{\psi_i})^{a_i}}{(1+e^{\psi_i})^{b_i}} \bbI \{\psi_i = x_i \beta_i\}
\Big] p(\bbeta)
\]
where $a_i$ and $b_i$ may be functions of $y$ or $\theta$, $\theta$ is a set of
parameters that we suppress from the notation above and below; $\bbeta =
\{\beta_t\}$; and $p(\bbeta)$ governs the dynamics of $\bbeta$.  For instance,
$p(\bbeta)$ might be the density of an AR(1) process.  As shown in
\cite{polson-etal-2012}, if $p(\bbeta)$ is a Gaussian prior and if one
augments this density with
\begin{equation}
\label{eqn:augmentation}
p(\omega | y, \bbeta) = \prod_{i=1}^n \Big[ p(\omega_i | b_i, \psi_i)
\Big]
\end{equation}
where $\omega_i \sim \PG(b_i, \psi_i)$, PG the \Polya-Gammd distribution, then
the complete conditional posterior distribution for $\bbeta$ will be Gaussian.
In particular, $p(\omega_i | b_i, \psi_i) \propto \cosh^{b_i}(\psi_i/2)
e^{-\omega_i \psi_i^2 / 2} p(\omega_i | b_i, 0)$ so that
% \[
% p(\beta | y, \omega) \propto 
% \Big[ \prod_{i=1}^n e^{\kappa_i \psi_i - \frac{\omega_i}{2} \psi_i^2}
% \bbI \{\psi_i = x_i' \beta\} \Big] p(\beta)
% \]
% where $\kappa_i = a_i - b_i / 2$.  A single term from the product has
% \[
% \exp \Big( \kappa_i \psi_i - \frac{\omega_i}{2} \psi_i^2 \Big)
% \propto
% \exp \Big(-\frac{1}{2} (\kappa_i / \omega_i - \psi_i)^2 \omega_i \Big),
% \]
% which is identical to the likelihood of an observation $z_i = \kappa_i /
% \omega_i$ drawn from
% \[
% z_i \sim N(x_i' \beta, \omega_{i}^{-1}).
% \]
\[
p(\bbeta | y, \omega) 
\propto 
\Big[ \prod_{i=1}^n \exp 
\Big(-\frac{\omega_i}{2} \Big(\frac{\kappa_i}{\omega_i} - \psi_i\Big)^2 \Big)
\bbI \{\psi_i = x_i' \beta\} \Big] p(\bbeta)
\]
where $\kappa_i = a_i - b_i / 2$.  A single term from the product above is
identical to the likelihood of a pseudo-data point $z_i = \kappa_i / \omega_i$
drawn from $z_i \sim N(\psi_i, 1/\omega_i)$.  Thus, if $p(\bbeta)$ specifies
that $\bbeta$ is a Gaussian AR(1) process, then sampling complete conditional
for $\bbeta$ is equivalent to sampling $\bbeta$ from the DLM
\[
\begin{cases}
z_t = \psi_t + \nu_t, & \nu_t \sim N(0, 1/\omega_t) \\
\psi_t = x_t \beta_t \\
\beta_t = \mu + \Phi (\beta_{t-1} - \mu) + \omega_t, & \omega_t \sim N(0, W).
\end{cases}
\]

Thus, for the dynamic binomial or negative binomial model, posterior samples may
be generated via Gibbs sampling: the states $\bbeta$ via the FFBS, the parameters
of the AR(1) process via standard conjugate updating procedures, and $(\omega |
\bbeta, y)$ via (\ref{eqn:augmentation}).  \cite{polson-etal-2012} describe how
to sample from $\PG$ distributions and implement this sampler in the \texttt{R}
package \texttt{BayesLogit}.

\begin{example}
Suppose one observes a binomial outcome $y_t$ where $P(y_t = 1) = p_t$ out of
$n_t$ trials at time $t$.  Letting $\psi_t$ be the log-odds, the data generating
distribution is
\[
p(y_t | \psi_t) = c(y_t) \frac{(e^\psi_t)^{y_t}}{(1+e^{\psi_t})^{n_t}}.
\]
Thus the complete conditional $(\bbeta | y, \omega)$ may be simulated by using
filter forward backward sampling with pseudo-data $z_t = \kappa_t / \omega_t$
where $\kappa_t = y_t - n_t / 2$.
\end{example}


\begin{example}
  Suppose that one observes counts according to $y_t \sim \NB(d, p_t)$ where $d$
  is the number of ``failures'' before observing $y_t$ successes, or dispersion,
  and $p_t$ is the probability of observing a success.  Letting $\psi_t$ be the
  log-odds, the data generating distribution is
\[
p(y_t | \psi_t) = c(y_t, d) \frac{(e^{\psi_t})^{y_t}}{(1+e^{\psi_t})^{y_t+d}}.
\]
In negative binomial regression, it is common to model the log-mean, $\lambda_t
= \psi_t + \log(h) = x_t \beta_t$, instead of the log-odds.  This requires only
a slight modification.  Following the work above, the complete conditional
$(\omega_t | \beta_t, h)$ is $\PG(b_t, \psi_t)$ where $b_t = y_t + d$.  However,
the DLM used to estimate $\bbeta$ is now
% \[
% \begin{cases}
% z_t = \psi_t + \ep_t, & \ep_t \sim N(0, 1/\omega_t) \\
% \psi_t = x_t \beta_t - \log(d) \\
% \beta_t = \mu + \Phi (\beta_{t-1} - \mu) + \omega_t, & \omega_t \sim N(0, W)
% \end{cases}
% \]
% where $z_t = \kappa_t / \omega_t$ and $\kappa_t = (y_t - d_t) / 2$.
\[
\begin{cases}
z_t = \lambda_t + \ep_t, & \ep_t \sim N(0, 1/\omega_t) \\
\lambda_t = x_t \beta_t \\
\beta_t = \mu + \Phi (\beta_{t-1} - \mu) + \omega_t, & \omega_t \sim N(0, W)
\end{cases}
\]
where $z_t = \kappa_t / \omega_t + \log(d)$ and $\kappa_t = (y_t - d) / 2$.
\end{example}

% To estimate $d$ in dynamic negative binomial regression one may use a
% Metropolis-Hastings step (Appendix \ref{sec:sample-d}).

%\subsection{AR(1) Example}

\begin{figure}
\begin{center}
\includegraphics[width=5.5in]{nb-ar1-tophalf.pdf}
\caption{\label{fig:nb-ar1} Incidence of influenza-like illness in Texas,
  2008--11, together with the estimated mean $\lambda_t$ from the
  negative-binomial AR(1) model.  The blanks in weeks 21-41 correspond to
  missing data.  The grey lines depict the upper and lower bounds of a 95$\%$
  predictive interval.}
\end{center}
\end{figure}

As an initial illustration of the point, we fit a negative-binomial AR(1) model
to four years (2008--11) of weekly data on flu incidence in Texas, collected
from the Texas Department of State Health Services.  Let $y_t$ denote the number
of reported cases of influenza-like illness (ILI) in week $t$.  We assume that
these counts follow a negative-binomial model, which will allow over-dispersion
relative to the Poisson.
% \begin{eqnarray*}
% % y_t &\sim& \mbox{NB}(h, p_t) \; , \quad p_t= \frac{e^{\psi_t}}{1+e^{\psi_t}} \\
% \psi_t &=& \alpha + \gamma_t  \; , \quad \gamma_t = \phi \gamma_{t-1} + \epsilon_t \\
% \epsilon_t &\sim& N(0, \sigma^2) \, .
% \end{eqnarray*}
Figure \ref{fig:nb-ar1} shows the results of the fit. For simplicity, we assumed
an improper uniform prior on the dispersion parameter $d$, and fixed $\phi$ and
$\sigma^2$ to $0.98$ and 1, respectively, but it is straightforward to place
hyper-priors upon each parameter, and to sample them in a hierarchical fashion.
It is also straightforward to incorporate fixed effects in the form of
regressors.

% We emphasize that there are many ways to handle the over-dispersion present in
% this data set, and that we do not intend our model to be taken as a definitive
% analysis.  We merely intend it as a proof of concept, showing how various
% aspects of Bayesian hierarchical modeling---in this case, a simple AR(1)
% model---can be combined routinely with binomial likelihoods using the
% \Polya-Gamma scheme.

\section{Comparison}

% Dynamic versions of binomial logistic regression and negative binomial
% regression are common models.  Hence, it is important to understand which
% methods work best for each model.

Since Markov Chains generate correlated samples we compare methods by measuring
how fast each procedure produces \emph{nearly} independent samples, that is we
measure the effective sampling rate (ESR).  To that end, we employ the effective
sample size (ESS), which approximates the number of ``independent'' draws
produced by a sample of $M$ correlated draws.  One may view the ESS as an
estimate of the number of samples produced after having thinned the $M$
correlated samples so that remaining draws appear independent.  From
\cite{holmes-held-2006}, who follow \cite{geyer-1992}, the effective sample size
is
\[
ESS_{it} = M / \Big( 1 + 2 \sum_{k=1}^\ell \rho_k(\beta_{it}) \Big)
\]
where $\rho_k(\beta_{it})$ is the $k$th lagged autocorrelation of the chain
corresponding to the $i$th component of $\{\beta_t\}_{t=1}^T$.  The effective
sampling rate is the ratio of the effective sample size to the time taken to
generate the post-burn-in samples; thus, it measures the rate at which the
Markov Chain produces independent draws after initialization and burn-in.

To mitigate MCMC sample variation, 10 batches of 12,000 samples are taken and
the last 10,000 draws are recorded.  For batch $m$, we compute the
component-wise effective sample size $ESS_{it}^{m}$ corresponding to the
univariate chain for $\beta_{it}$.  Taking the mean over batches produces the
average, component-wise effective sample size $\overline{ESS}_{it}$ and,
normalizing by time, the average, component-wise effective sampling rate
$\overline{ESR}_{it}$.  Following, \cite{fruhwirth-schnatter-fruhwirth-2010},
the primary metric of comparison is the median effective sampling rate,
\[
\text{med} \; \Big\{ \overline{ESR}_{it} : i=1, \ldots, P; \;  t=1, \ldots, T \Big\}.
\]

We consider data sets with a variety of characteristics since this is an
empirical, as opposed to theoretical assessment.  For dynamic binomial logistic
regression, we consider log odds of the form $\alpha + x_t' \beta_t$ where
$\alpha$ is a scalar and $\{\beta_t\}_{t=1}^T$ is a 2-dimensional AR(1) process
with large autocorrelation.  Four different synthetic data sets are constructed,
allowing for low and high correlation amongst $x_t$ and responses, $y_t \sim
\text{Binom}(p_t, n), t=1, \ldots, T$, with either $n=1$ or $n=20$ trials.  The
setup is almost identical for dynamic binomial regression except that we model
the log-mean as $\alpha + x_t \beta_t$ and consider responses with $\alpha =
\log(10)$ or $\alpha = \log(100)$ corresponding to average count sizes of
roughly 10 or 100.  Further details may be found in Appendix
\ref{sec:benchmark-details}.

Some caution is warranted when comparing methods as the effective sampling rate
of a routine is sensitive to its implementation and the hardware on which it is
run. The code is written primarily in R.  We use code from
\cite{binomlogit-2012} and \cite{fruhwirth-schnatter-book-2007} for the discrete
mixture of normals methods.  All benchmarks are carried out on a Ubuntu machine
with Intel Core i5-3570 3.4GHz processor and 8GB of RAM.  Some computations were
burdensome in R, and hence we wrote wrappers to C to speed up the MCMC
simulation.  In particular, both data augmentation methods implement forward
filtering and backward sampling using a C wrapper.  The PG technique calls C
code to sample \Polya-Gamma random variates using version 0.2-4 of the
\texttt{BayesLogit} package \citep{bayeslogit-2013}.  The
\cite{ravines-etal-2006} approach implements the conjugate updating and
backwards sampling using C.  Having a C wrapper to forward filter and backwards
sample is particularly important, as our C implementation is much faster than
the corresponding R code.  Were we to use the slower R version, our results
would favor the PG method, as it has better effective sample sizes and would
spend less time, proportionally, sampling the \Polya-Gamma random variates.

\cite{polson-etal-2013} outline the expected performance of the \Polya-Gamma
data augmentation technique, which depends heavily on how quickly one can
generate \Polya-Gamma random variates.  In their original algorithm, sampling
\Polya-Gamma random variates $\PG(b,\psi)$ is fast when $b$ is a small integer,
but slow when $b$ is large. \cite{windle-thesis-2013} provides an improved
method for sampling $\PG(b,\psi)$; however, even using that improvement,
sampling large $b$ is still slower than sampling small $b$.  These differences
in computational cost are important, as one must sample $\omega_i \sim \PG(b_i,
\psi_i), i=1,\ldots,T$ for each MCMC iteration under \Polya-Gamma data
augmentation.  In binomial logistic regression, $b_i = n_i$, the number of
trials at each response $y_i$.  Hence, when $n_i$ is small the PG method will do
well.  In negative binomial regression, $b_i = y_i + d_i$ where $y_i$ is a
response and $d_i$ is the dispersion.  Thus, larger average responses should
lead to longer MCMC simulations.

In general, we find these principles to hold.  The \Polya-Gamma data
augmentation technique performs well for dynamic binomial logistic regression
when the number of trials of the response is small, but the method of
\cite{fussl-etal-2013} does better when the number of trials is large.
Similarly, the \Polya-Gamma technique outperforms
\cite{fruhwirth-schnatter-etal-2009} in negative binomial regression when the
average number of counts is small, but loses when the average number of counts
is large.  The Metropolis-Hastings approach of \cite{ravines-etal-2006} performs
the worst in all of the tests.  Part of its poor performance is due to a
non-linear solve that must be performed for each $t=1, \ldots, T$ when filtering
forward.  We did not attempt to optimize the performance of this non-linear
solver, and hence some improvement may be possible, though the disparity in
performance suggests that any improvement would not be enough to compete with
either data augmentation approach.  As a check upon \cite{ravines-etal-2006}, we
also implemented a Metropolis-Hastings sampler that draws blocks of disturbances
using Laplace approximations.  This fared worse still. 

Of note, the \Polya-Gamma method almost always has superior effective sample
sizes.  Hence faster \Polya-Gamma samplers could push the \Polya-Gamma data
augmentation technique to the top for several or all of the models considered.
Table \ref{tab:benchmark-summary} provides a summary of the benchmarks; Tables
\ref{tab:dynlogit-detail} and \ref{tab:dynnb-detail} provide a more detailed
account of each benchmark.

\begin{table}
\small
\centering 

\begin{tabular}{l l c c}
\multicolumn{4}{c}{Dynamic Binomial Reg.} \\
\hline
& & Est.\ & ESR \\
n & f & AR & PG/dRUM \\
\hline
% Bench-Dyn-06/Tables/table.bench-dynlogit-low-2-n-1-wout.ar
% PG / dRUM 
$1$ & low & no & 1.26 \\
% PG / CUBS 47.5399
% Bench-Dyn-06/Tables/table.bench-dynlogit-low-2-n-1-with.ar
% PG / PG 1
% PG / dRUM 
$1$ & low & yes & 1.29 \\
% PG / CUBS 287.291

% Bench-Dyn-06/Tables/table.bench-dynlogit-high-2-n-1-wout.ar
% PG / PG 1
% PG / dRUM 
$1$ & high & no & 1.25 \\
% PG / CUBS 36.9012
% Bench-Dyn-06/Tables/table.bench-dynlogit-high-2-n-1-with.ar
% PG / PG 1
% PG / dRUM 
$1$ & high & yes & 1.26 \\
% PG / CUBS 175.964

% Bench-Dyn-06/Tables/table.bench-dynlogit-low-2-n-20-wout.ar
% PG / PG 1
% PG / dRUM 
$20$ & low & no & 0.52 \\
% PG / CUBS 29.4906
% Bench-Dyn-06/Tables/table.bench-dynlogit-low-2-n-20-with.ar
% PG / PG 1
% PG / dRUM 
$20$ & low & yes & 0.57 \\
% PG / CUBS 49.9306

% Bench-Dyn-06/Tables/table.bench-dynlogit-high-2-n-20-wout.ar
% PG / PG 1
% PG / dRUM 
$20$ & high & no & 0.48 \\
% PG / CUBS 13.323
% Bench-Dyn-06/Tables/table.bench-dynlogit-high-2-n-20-with.ar
% PG / PG 1
% PG / dRUM 
$20$ & high & yes & 0.58 
% PG / CUBS 38.9124
\end{tabular}
%
\hspace{12pt}
%
\begin{tabular}{l l c c}
\multicolumn{4}{c}{Dynamic Neg.\ Binomial Reg.} \\
\hline
& & Est.\ & ESR \\
$\mu$ & f & AR & PG/FS \\
\hline
% Bench-Dyn-06/Tables/table.bench-dynnb-low-2-mu-10-wout.ar
% PG / PG 1
% PG / FS 
$10$ & low & no & 1.03 \\
% PG / CUBS 42.3582
% Bench-Dyn-06/Tables/table.bench-dynnb-low-2-mu-10-with.ar
% PG / PG 1
% PG / FS 
$10$ & low & yes & 1.86 \\
% PG / CUBS 59.1934

% Bench-Dyn-06/Tables/table.bench-dynnb-high-2-mu-10-wout.ar
% PG / PG 1
% PG / FS 
$10$ & high & no & 1.15 \\
% PG / CUBS 20.6336
% Bench-Dyn-06/Tables/table.bench-dynnb-high-2-mu-10-with.ar
% PG / PG 1
% PG / FS 
$10$ & high & yes & 1.06 \\
% PG / CUBS 18.0189

% Bench-Dyn-06/Tables/table.bench-dynnb-low-2-mu-100-wout.ar
% PG / PG 1
% PG / FS 
$100$ & low & no & 0.76 \\
% PG / CUBS 103.273
% Bench-Dyn-06/Tables/table.bench-dynnb-low-2-mu-100-with.ar
% PG / PG 1
% PG / FS 
$100$ & low & yes & 0.82 \\
% PG / CUBS 102.229

% Bench-Dyn-06/Tables/table.bench-dynnb-high-2-mu-100-wout.ar
% PG / PG 1
% PG / FS 
$100$ & high & no & 0.76 \\
% PG / CUBS 34.7778
% Bench-Dyn-06/Tables/table.bench-dynnb-high-2-mu-100-with.ar
% PG / PG 1
% PG / FS
$100$ & high & yes & 0.78 \\
% PG / CUBS 69.6132

\end{tabular}

\caption{\label{tab:benchmark-summary} A summary of benchmarking results. ESR
  PG/dRUM 
  and ESR PG/FS report the ratio of
  the median effective sampling rate for the PG method compared to the best
  alternative, which in both cases are the discrete mixture of normal approaches
  of \cite{fussl-etal-2013} and \cite{fruhwirth-schnatter-etal-2009}.  A higher
  ratio corresponds to the PG method doing better than the best alternative.  The
  PG method always beats the CUBS approach and it does better than the discrete
  mixture of normals approach in dynamic binomial logistic regression when the number of
  trials is small and in dynamic negative binomial regression when the 
  mean is small.  $n$ corresponds to the number of trials for the binomial
  response while $\mu$ corresponds to the mean for the negative binomial
  response.  $f$ determines whether there is a low or high amount of correlation
  between the covariates.  Est.\ AR indicates whether the parameters of the AR
  process were estimated or not.
}
\end{table}



\section{Conclusion}

As a general guideline, we suggest using the \Polya-Gamma method as the default
approach for dynamic binomial logistic regression and the method of
\cite{fruhwirth-schnatter-etal-2009} as the default method for dynamic negative
binomial regression.  There are two exceptions to this rule: (1) if the number
of trials in the dynamic binomial logistic regression is large, it may be better
to use the method of \cite{fussl-etal-2013} and (2) if the average count size in
the dynamic negative binomial regression is small, it may be better to use the
\Polya-Gamma data augmentation approach.  If an end-user finds that his or her
modeling scenario does not definitively favor either technique, we recommend
using the \Polya-Gamma approach since it is easily incorporated into bespoke
dynamic models thanks to its conditionally Gaussian form and single layer of
latent variables.  This form ensures that all of the machinery from Gaussian
models apply and that one only need a \Polya-Gamma sampler, as is provided by
the R Packages \texttt{BayesLogit}, to generate posterior samples.

\appendix

% \section{Sampling $d$ in dynamic negative binomial regression}
% \label{sec:sample-d}

% We may sample $d$ by an independent Metropolis-Hastings or a random-walk
% Metropolis-Hastings.  In either case, we need the log-likelihood of $d$.

% The conditional density for a single term $y_t \sim N(d, p_t)$ is
% \[
% \frac{\Gamma(y_t + d)}{\Gamma(d) \Gamma(y_t) y_t} p_t^{y_t} (1-p_t)^{d} ,
% \]
% or 
% \[
% \frac{\Gamma(y_t + d)}{\Gamma(d) \Gamma(y_t) y_t} 
% \Big( \frac{\mu_t}{\mu_t + d} \Big)^{y_t} \Big( \frac{d}{\mu_t + d} \Big)^d
% \]
% in terms of the mean $\mu_t$.  Since we chose to model $\log \mu_t = x_t
% \beta_t$ it will be easier to work with the latter version.  In that case, the
% log-likelihood is
% \[
% \ell(d|\mu, y) = \sum_{t=1}^T [ \log \Gamma(y_t + d)  - \log \Gamma(d) ]  + 
% \sum_{t=1}^T y_t \log \big( \frac{\mu_t}{\mu_t + d} \Big) +
% d \sum_{t=1}^T \log \Big( \frac{d}{\mu_t + d} \Big).
% \]

% If we assume that $d$ is a natural number then we can rid ourselves of $T$
% evaluations of the gamma function.  We may expand $\Gamma(y_t + d)$ and
% $\Gamma(d)$ as products, in which case the log-likelihood is
% \[
% \ell(d|\mu, y) = \sum_{t=1}^T \sum_{j=0}^{y_t-1} \log(d + j) + 
% \sum_{t=1}^T y_t \log \big( \frac{\mu_t}{\mu_t + d} \Big) +
% d \sum_{t=1}^T \log \Big( \frac{d}{\mu_t + d} \Big).
% \]
% We can rewrite the first term of the sum as
% \[
% \sum_{t=1}^T \sum_{k=1}^{\max(y_t)} \sum_{j=0}^{k-1} \log(d+j) \one \{y_t = k\}
% \]
% which becomes
% \[
% \sum_{k=1}^{\max(y_t)} n_k \sum_{j=0}^{k-1} \log(d+j)
% \]
% where $n_k = \{ \# y_t = k \}$.  We have thus
% \begin{align*}
% \sum_{k=1}^{\max(y_t)} \sum_{j=0}^{\max(y_t)-1} n_k \log(d+j) \one\{j < k\}
% & = 
% \sum_{j=0}^{\max(y_t)-1} \log(d+j) \sum_{k=j+1}^{\max(y_t)} n_k \\
% & = \sum_{j=0}^{\max(y_t)-1} \log(d+j) G_j
% \end{align*}
% where $G_j = \{ \# y_t > j \}$.  Notice that $1-G_j = F(j) = \{\# y_t \leq j\}$
% and that $G$ may be pre-processed.  Hence the log-likelihood is
% \[
% \sum_{j=0}^{\max(y_t)-1} \log(d+j) G_j + 
% \sum_{t=1}^T y_t \log \big( \frac{\mu_t}{\mu_t + d} \Big) +
% d \sum_{t=1}^T \log \Big( \frac{d}{\mu_t + d} \Big).
% \]
% Preprocessing $G$ saves us from having to compute a doubly indexed summation.

\section{Benchmarks}
\label{sec:benchmark-details}

% \begin{table}

% \begin{tabular}{l r r r r r r r r } 
% \hline
% \multicolumn{9}{c}{Dataset: Tokyo.  Prior: $\phi = 1, W \sim IGa(150, 15)$.  $T=366$.} \\
% \hline
%           Method  &     time &    ARate &  ESS.min &  ESS.med &  ESS.max &  ESR.min &  ESR.med &  ESR.max \\ 
%               PG  &    21.72 &     1.00 &  4352.80 &  7735.08 & 10081.46 &   200.43 &   356.18 &   464.22 \\ 
%             dRUM  &    20.60 &     1.00 &  1627.61 &  3649.46 &  5428.93 &    79.03 &   177.19 &   263.60 \\ 
%             CUBS  &   233.98 &     0.47 &   652.86 &   761.98 &   910.62 &     2.79 &     3.26 &     3.89
%  \end{tabular}

% % ./table.bench-dynlogit-low-2-n-1-wout.ar
% \begin{tabular}{l r r r r r r r r } 
% \hline
% \multicolumn{9}{c}{Dataset: Synth Low.  Prior $\phi_i = 0.95, W_i=0.172$.  $T=500,
%   P=2$.} \\
% \hline
%           Method  &     time &    ARate &  ESS.min &  ESS.med &  ESS.max &  ESR.min &  ESR.med &  ESR.max \\ 
%               PG  &    28.34 &     1.00 &  8309.00 &  9395.80 &  9894.45 &   293.16 &   331.50 &   349.09 \\ 
%             dRUM  &    29.08 &     1.00 &  5299.68 &  7646.07 &  9800.60 &   182.26 &   262.96 &   337.05 \\ 
%             CUBS  &   609.04 &     0.68 &  2424.95 &  3930.01 &  5168.38 &     3.98 &     6.45 &     8.49
%  \end{tabular}

% % ./table.bench-dynlogit-high-2-n-1-wout.ar
% \begin{tabular}{l r r r r r r r r } 
% \hline
% \multicolumn{9}{c}{Dataset: Synth High.  Prior: $\phi_i = 0.95, W_i=0.172$.
%   $T=500, P=2$.} \\
% \hline
%           Method  &     time &    ARate &  ESS.min &  ESS.med &  ESS.max &  ESR.min &  ESR.med &  ESR.max \\ 
%               PG  &    28.46 &     1.00 &  8611.53 &  9441.44 &  9894.56 &   302.60 &   331.76 &   347.68 \\ 
%             dRUM  &    29.19 &     1.00 &  6204.70 &  7879.60 &  9656.77 &   212.54 &   269.91 &   330.78 \\ 
%             CUBS  &   610.43 &     0.61 &  2562.38 &  4291.96 &  6932.42 &     4.20 &     7.03 &    11.36
%  \end{tabular}

% % ./table.bench-dynlogit-low-2-n-1-with.ar
% \begin{tabular}{l r r r r r r r r } 
% \hline
% \multicolumn{9}{c}{ Dataset: Synth Low. Prior $\phi_i \sim N(0.95, 0.01), W_i
%   \sim IGa(10, 1)$.  $T=500, P=2$.} \\
% \hline
%           Method  &     time &    ARate &  ESS.min &  ESS.med &  ESS.max &  ESR.min &  ESR.med &  ESR.max \\ 
%               PG  &    31.03 &     1.00 &  1348.88 &  7063.34 &  9914.37 &    43.46 &   227.65 &   319.53 \\ 
%             dRUM  &    31.81 &     1.00 &  1213.67 &  5643.54 &  9060.83 &    38.15 &   177.40 &   284.82 \\ 
%             CUBS  &   627.81 &     0.65 &   125.22 &   576.81 &  1695.55 &     0.20 &     0.92 &     2.70
%  \end{tabular}

% % ./table.bench-dynlogit-high-2-n-1-with.ar
% \begin{tabular}{l r r r r r r r r } 
% \hline
% \multicolumn{9}{c}{Dataset: Synth Low. Prior $\phi_i \sim N(0.95, 0.01), W_i \sim IGa(10, 1)$.  $T=500, P=2$.} \\
% \hline
%           Method  &     time &    ARate &  ESS.min &  ESS.med &  ESS.max &  ESR.min &  ESR.med &  ESR.max \\ 
%               PG  &    31.02 &     1.00 &   686.61 &  4802.36 &  9523.97 &    22.13 &   154.81 &   307.01 \\ 
%             dRUM  &    31.84 &     1.00 &   479.20 &  3774.41 &  8505.58 &    15.05 &   118.55 &   267.15 \\ 
%             CUBS  &   618.25 &     0.55 &   181.61 &   541.04 &  1617.91 &     0.29 &     0.87 &     2.61
%  \end{tabular}

% % \begin{tabular}{l r r r r r r r r } 
% % \hline
% % \multicolumn{9}{c}{./table.bench-dynlogit-low-2-n-10-wout.ar} \\
% % \hline
% %           Method  &     time &    ARate &  ESS.min &  ESS.med &  ESS.max &  ESR.min &  ESR.med &  ESR.max \\ 
% %               PG  &    49.77 &     1.00 &  7116.22 &  9374.76 &  9911.13 &   142.98 &   188.36 &   199.13 \\ 
% %             dRUM  &    28.98 &     1.00 &  3540.13 &  7482.71 &  9884.88 &   122.14 &   258.16 &   341.04 \\ 
% %             CUBS  &   638.99 &     0.60 &  1233.54 &  3317.85 &  5278.43 &     1.93 &     5.19 &     8.26
% %  \end{tabular}

% % \begin{tabular}{l r r r r r r r r } 
% % \hline
% % \multicolumn{9}{c}{./table.bench-dynlogit-high-2-n-10-wout.ar} \\
% % \hline
% %           Method  &     time &    ARate &  ESS.min &  ESS.med &  ESS.max &  ESR.min &  ESR.med &  ESR.max \\ 
% %               PG  &    50.17 &     1.00 &  7327.10 &  9211.07 &  9826.71 &   146.05 &   183.61 &   195.88 \\ 
% %             dRUM  &    28.97 &     1.00 &  3939.55 &  6967.89 &  9766.74 &   135.99 &   240.52 &   337.14 \\ 
% %             CUBS  &   643.20 &     0.52 &  2370.69 &  4240.41 &  5878.98 &     3.69 &     6.59 &     9.14
% %  \end{tabular}

% % \begin{tabular}{l r r r r r r r r } 
% % \hline
% % \multicolumn{9}{c}{./table.bench-dynlogit-low-2-n-10-with.ar} \\
% % \hline
% %           Method  &     time &    ARate &  ESS.min &  ESS.med &  ESS.max &  ESR.min &  ESR.med &  ESR.max \\ 
% %               PG  &    52.54 &     1.00 &   626.75 &  5161.88 &  9855.10 &    11.93 &    98.24 &   187.56 \\ 
% %             dRUM  &    31.84 &     1.00 &   674.18 &  3477.35 &  9119.24 &    21.17 &   109.21 &   286.40 \\ 
% %             CUBS  &   661.84 &     0.56 &   364.16 &  1443.76 &  3065.32 &     0.55 &     2.18 &     4.63
% %  \end{tabular}

% % \begin{tabular}{l r r r r r r r r } 
% % \hline
% % \multicolumn{9}{c}{./table.bench-dynlogit-high-2-n-10-with.ar} \\
% % \hline
% %           Method  &     time &    ARate &  ESS.min &  ESS.med &  ESS.max &  ESR.min &  ESR.med &  ESR.max \\ 
% %               PG  &    52.99 &     1.00 &   174.62 &  1001.65 &  8922.27 &     3.29 &    18.91 &   168.37 \\ 
% %             dRUM  &    31.79 &     1.00 &   151.89 &   867.68 &  7060.28 &     4.78 &    27.30 &   222.06 \\ 
% %             CUBS  &   651.22 &     0.47 &   192.77 &   772.78 &  2680.11 &     0.30 &     1.19 &     4.12
% %  \end{tabular}
% \end{table}

% \newpage

% \begin{table}
% \begin{tabular}{l r r r r r r r r } 
% \hline
% \multicolumn{9}{c}{./table.bench-dynnb-low-2-mu-10-wout.ar} \\
% \hline
%           Method  &     time &    ARate &  ESS.min &  ESS.med &  ESS.max &  ESR.min &  ESR.med &  ESR.max \\ 
%               PG  &    63.86 &     1.00 &  8305.72 &  9668.71 & 10000.00 &   130.07 &   151.41 &   156.60 \\ 
%               FS  &    29.70 &     1.00 &  1755.48 &  6508.09 &  9870.64 &    59.11 &   219.11 &   332.32 \\ 
%             CUBS  &   648.25 &     0.60 &   384.08 &  3407.23 &  5894.51 &     0.59 &     5.26 &     9.09
%  \end{tabular}

% \begin{tabular}{l r r r r r r r r } 
% \hline
% \multicolumn{9}{c}{./table.bench-dynnb-high-2-mu-10-wout.ar} \\
% \hline
%           Method  &     time &    ARate &  ESS.min &  ESS.med &  ESS.max &  ESR.min &  ESR.med &  ESR.max \\ 
%               PG  &    59.63 &     1.00 &  8967.77 &  9684.87 & 10000.00 &   150.39 &   162.42 &   167.70 \\ 
%               FS  &    29.65 &     1.00 &  2844.82 &  5812.03 &  9773.66 &    95.95 &   196.02 &   329.64 \\ 
%             CUBS  &   656.19 &     0.74 &  4605.04 &  7412.28 &  8063.25 &     7.02 &    11.30 &    12.29
%  \end{tabular}

% \begin{tabular}{l r r r r r r r r } 
% \hline
% \multicolumn{9}{c}{./table.bench-dynnb-low-2-mu-10-with.ar} \\
% \hline
%           Method  &     time &    ARate &  ESS.min &  ESS.med &  ESS.max &  ESR.min &  ESR.med &  ESR.max \\ 
%               PG  &    66.80 &     1.00 &   149.03 &  4880.10 &  9833.67 &     2.23 &    73.06 &   147.21 \\ 
%               FS  &    32.65 &     1.00 &   149.47 &  1793.39 &  7952.70 &     4.58 &    54.92 &   243.58 \\ 
%             CUBS  &   667.39 &     0.55 &   135.02 &  1249.47 &  2792.73 &     0.20 &     1.87 &     4.18
%  \end{tabular}

% \begin{tabular}{l r r r r r r r r } 
% \hline
% \multicolumn{9}{c}{./table.bench-dynnb-high-2-mu-10-with.ar} \\
% \hline
%           Method  &     time &    ARate &  ESS.min &  ESS.med &  ESS.max &  ESR.min &  ESR.med &  ESR.max \\ 
%               PG  &    62.28 &     1.00 &   244.92 &  2039.14 &  9767.92 &     3.93 &    32.73 &   156.83 \\ 
%               FS  &    32.51 &     1.00 &   284.53 &  1219.91 &  5614.48 &     8.75 &    37.52 &   172.66 \\ 
%             CUBS  &   669.68 &     0.62 &   363.77 &  1609.55 &  4586.74 &     0.54 &     2.40 &     6.85
%  \end{tabular}

% \begin{tabular}{l r r r r r r r r } 
% \hline
% \multicolumn{9}{c}{./table.bench-dynnb-low-2-mu-100-wout.ar} \\
% \hline
%           Method  &     time &    ARate &  ESS.min &  ESS.med &  ESS.max &  ESR.min &  ESR.med &  ESR.max \\ 
%               PG  &   393.08 &     1.00 &  1898.18 &  7298.22 &  9990.29 &     4.83 &    18.57 &    25.42 \\ 
%               FS  &    29.68 &     1.00 &  2386.51 &  7229.31 & 10000.00 &    80.40 &   243.58 &   336.93 \\ 
%             CUBS  &   661.62 &     0.37 &   198.71 &  1150.20 &  2867.16 &     0.30 &     1.74 &     4.33
%  \end{tabular}

% \begin{tabular}{l r r r r r r r r } 
% \hline
% \multicolumn{9}{c}{./table.bench-dynnb-high-2-mu-100-wout.ar} \\
% \hline
%           Method  &     time &    ARate &  ESS.min &  ESS.med &  ESS.max &  ESR.min &  ESR.med &  ESR.max \\ 
%               PG  &   292.81 &     1.00 &  4090.32 &  7283.70 &  9697.62 &    13.97 &    24.87 &    33.12 \\ 
%               FS  &    29.65 &     1.00 &  3417.13 &  6864.88 &  9566.54 &   115.26 &   231.58 &   322.68 \\ 
%             CUBS  &   660.24 &     0.39 &  1065.38 &  3124.58 &  4242.90 &     1.61 &     4.73 &     6.43
%  \end{tabular}

% \begin{tabular}{l r r r r r r r r } 
% \hline
% \multicolumn{9}{c}{./table.bench-dynnb-low-2-mu-100-with.ar} \\
% \hline
%           Method  &     time &    ARate &  ESS.min &  ESS.med &  ESS.max &  ESR.min &  ESR.med &  ESR.max \\ 
%               PG  &   395.37 &     1.00 &  1766.07 &  4793.62 &  9445.17 &     4.47 &    12.12 &    23.89 \\ 
%               FS  &    32.61 &     1.00 &  2217.92 &  4814.16 &  9682.45 &    68.02 &   147.64 &   296.91 \\ 
%             CUBS  &   670.26 &     0.32 &   347.70 &   783.63 &  1561.49 &     0.52 &     1.17 &     2.33
%  \end{tabular}

% \begin{tabular}{l r r r r r r r r } 
% \hline
% \multicolumn{9}{c}{./table.bench-dynnb-high-2-mu-100-with.ar} \\
% \hline
%           Method  &     time &    ARate &  ESS.min &  ESS.med &  ESS.max &  ESR.min &  ESR.med &  ESR.max \\ 
%               PG  &   295.55 &     1.00 &  1170.90 &  3414.61 &  8363.79 &     3.96 &    11.55 &    28.30 \\ 
%               FS  &    32.48 &     1.00 &  1139.88 &  3228.07 &  7966.98 &    35.10 &    99.39 &   245.29 \\ 
%             CUBS  &   674.96 &     0.30 &   381.00 &   744.60 &  1115.24 &     0.56 &     1.10 &     1.65
%  \end{tabular}

% \caption{Using old non-hybrid sampler.}

% \end{table}

% \newpage

% \begin{table}

% % ./bench-dynnb-low-2-mu-10-wout.ar.table
% \begin{tabular}{l r r r r r r r r } 
% \hline
% \multicolumn{9}{c}{Low: $P=2$, $N=10$, $\Phi = 0.95I$, $W=0.172 I$} \\
% \hline
%           Method  &    time  &   ARate  &  ESS.min  &  ESS.med  &  ESS.max  &  ESR.min  &  ESR.med  &  ESR.max  \\ 
%              PG   &    42.19 &     1.00 &   7909.83 &   9605.44 &   9989.04 &    187.49 &    227.68 &    236.77 \\ 
%              FS   &    29.49 &     1.00 &   1828.26 &   6526.06 &   9794.63 &     62.00 &    221.31 &    332.15
%  \end{tabular}

% %./bench-dynnb-low-2-mu-10-with.ar.table
% \begin{tabular}{l r r r r r r r r } 
% \hline
% \multicolumn{9}{c}{Low: $P=2$, $N=10$, $\phi \sim N(0.95, 0.1 I)$, $W_i \sim Ga(300/2, 30/2)$} \\
% \hline
%           Method  &    time  &   ARate  &  ESS.min  &  ESS.med  &  ESS.max  &  ESR.min  &  ESR.med  &  ESR.max  \\ 
%              PG   &    45.03 &     1.00 &    159.08 &   4846.11 &   9717.61 &      3.53 &    107.62 &    215.79 \\ 
%              FS   &    32.36 &     1.00 &    140.04 &   1824.40 &   7436.81 &      4.33 &     56.37 &    229.79
%  \end{tabular}

% % ./bench-dynnb-high-2-mu-10-wout.ar.table
% \begin{tabular}{l r r r r r r r r } 
% \hline
% \multicolumn{9}{c}{High: $P=2$, $N=10$, $\Phi = 0.95I$, $W=0.172 I$} \\
% \hline
%           Method  &    time  &   ARate  &  ESS.min  &  ESS.med  &  ESS.max  &  ESR.min  &  ESR.med  &  ESR.max  \\ 
%              PG   &    42.08 &     1.00 &   9092.48 &   9724.85 &   9968.42 &    216.06 &    231.09 &    236.88 \\ 
%              FS   &    29.52 &     1.00 &   2915.06 &   5866.55 &   9705.36 &     98.74 &    198.71 &    328.74
%  \end{tabular}

% % ./bench-dynnb-high-2-mu-10-with.ar.table
% \begin{tabular}{l r r r r r r r r } 
% \hline
% \multicolumn{9}{c}{High: $P=2$, $N=10$, $\phi \sim N(0.95, 0.1 I)$, $W_i \sim Ga(300/2, 30/2)$} \\
% \hline
%           Method  &    time  &   ARate  &  ESS.min  &  ESS.med  &  ESS.max  &  ESR.min  &  ESR.med  &  ESR.max  \\ 
%              PG   &    44.93 &     1.00 &    276.43 &   1969.84 &   9528.81 &      6.15 &     43.84 &    212.09 \\ 
%              FS   &    32.39 &     1.00 &    239.37 &   1090.05 &   5350.24 &      7.39 &     33.65 &    165.19
%  \end{tabular}

% % ./bench-dynnb-low-2-mu-100-wout.ar.table
% \begin{tabular}{l r r r r r r r r } 
% \hline
% \multicolumn{9}{c}{Low: $P=2$, $N=100$, $\Phi = 0.95I$, $W=0.172 I$} \\
% \hline
%           Method  &    time  &   ARate  &  ESS.min  &  ESS.med  &  ESS.max  &  ESR.min  &  ESR.med  &  ESR.max  \\ 
%              PG   &    38.41 &     1.00 &   1886.58 &   7345.99 &   9855.94 &     49.12 &    191.26 &    256.61 \\ 
%              FS   &    29.49 &     1.00 &   2447.72 &   7362.03 &   9927.30 &     83.00 &    249.64 &    336.62
%  \end{tabular}

% % ./bench-dynnb-low-2-mu-100-with.ar.table
% \begin{tabular}{l r r r r r r r r } 
% \hline
% \multicolumn{9}{c}{Low: $P=2$, $N=100$, $\phi \sim N(0.95, 0.1 I)$, $W_i \sim Ga(300/2, 30/2)$} \\
% \hline
%           Method  &    time  &   ARate  &  ESS.min  &  ESS.med  &  ESS.max  &  ESR.min  &  ESR.med  &  ESR.max  \\ 
%              PG   &    41.25 &     1.00 &   1832.73 &   4804.28 &   9270.06 &     44.43 &    116.46 &    224.71 \\ 
%              FS   &    32.34 &     1.00 &   2283.24 &   4706.06 &   9081.44 &     70.60 &    145.52 &    280.80
%  \end{tabular}

% % ./bench-dynnb-high-2-mu-100-wout.ar.table
% \begin{tabular}{l r r r r r r r r } 
% \hline
% \multicolumn{9}{c}{High: $P=2$, $N=100$, $\Phi = 0.95I$, $W=0.172 I$} \\
% \hline
%           Method  &    time  &   ARate  &  ESS.min  &  ESS.med  &  ESS.max  &  ESR.min  &  ESR.med  &  ESR.max  \\ 
%              PG   &    40.60 &     1.00 &   4037.45 &   7248.51 &   9720.22 &     99.44 &    178.52 &    239.40 \\ 
%              FS   &    29.49 &     1.00 &   3449.77 &   6959.22 &   9727.76 &    116.98 &    235.98 &    329.85
%  \end{tabular}

% % ./bench-dynnb-high-2-mu-100-with.ar.table
% \begin{tabular}{l r r r r r r r r } 
% \hline
% \multicolumn{9}{c}{High: $P=2$, $N=100$, $\phi \sim N(0.95, 0.1 I)$, $W_i \sim Ga(300/2, 30/2)$} \\
% \hline
%           Method  &    time  &   ARate  &  ESS.min  &  ESS.med  &  ESS.max  &  ESR.min  &  ESR.med  &  ESR.max  \\ 
%              PG   &    43.42 &     1.00 &   1308.62 &   3487.57 &   8016.66 &     30.13 &     80.31 &    184.62 \\ 
%              FS   &    32.35 &     1.00 &   1199.19 &   3212.34 &   7733.74 &     37.07 &     99.29 &    239.05
%  \end{tabular}

% \caption{Using hybrid sampler.  $d$ set to 4.  $x_t^{(i)} \propto
%   \cos( \pi \omega_i u_t)$.  $\beta_t = \phi \beta_{t-1} + \omega_t, \; \omega_t
%   \sim N(0, W)$.  $\psi_t = \log(nb.mean) + x_t' \beta_t$.}

% \end{table}

%%%%%%%%%%%%%%%%%%%%%%%%%%%%%%%%%%%%%%%%%%%%%%%%%%%%%%%%%%%%%%%%%%%%%%%%%%%%%%%%
                              %% 06 Benchmarks %%
%%%%%%%%%%%%%%%%%%%%%%%%%%%%%%%%%%%%%%%%%%%%%%%%%%%%%%%%%%%%%%%%%%%%%%%%%%%%%%%%

For both binomial responses and negative binomial responses, we create four
synthetic data sets whose log-odds and log-mean, respectively, take the form
\begin{equation}
\label{eqn:synth-psi}
\psi_i = \alpha + x_i' \beta_i, \; i = 1, \ldots, T
\end{equation}
with
\[
\beta_i = \Phi \beta_{i-1} + \omega_i, \; \omega_i \sim N(0, W)
\]
where $\Phi = 0.95 I_2$ and $W = 0.172 I_2$.  The covariates $x_i \in \bbR^2$
are 
\[
x_{ij} = \cos(\pi f_j t_i) / \sqrt{8}
\]
where $\{t_i\}$ is an equally spaced partition of $[-1,1]$ and $f$ is either
$f^{(low)} = (1,2)$ or $f^{(high)} = (1,1.1)$, corresponding to covariates with
a lower or higher degree of correlation.  The binomial response is
\[
y_i \sim \text{Binom}(n, p_i)
\]
where $p_i = \frac{e^{\psi}}{(1+e^{\psi})}$, $n$ is either $1$ or $20$, and
$\alpha$ from (\ref{eqn:synth-psi}) set to 1.4.  The negative binomial response
is
\[
y_i \sim \text{Nbinom}(d, \mu_i)
\]
where $d=4$, $\mu_i = e^{\psi_i}$, and $\alpha$ from (\ref{eqn:synth-psi}) is
either $\log(10)$ or $\log(100)$.  Thus, for dynamic binomial logistic
regression, we create four data sets by varying $n$ and $f$, and for dynamic
negative binomial regression, we create four data sets by varying $f$ and
$\alpha$.

For each data set and each method, we run MCMC simulations assuming that the
parameters of the AR model are known and unknown.  When the parameters are
unknown, both $\Phi$ and $W$ are assumed to be diagonal; the prior for $\Phi$ is
$\Phi_{ii} \sim N(0.95, 0.1) \bbI_{(0,1)}$ and the prior for $W$ is $W_{ii} \sim
Ga(300/2, 30/2)$.  For all simulations, the prior for $\alpha$ is $N(0, 100)$
and the prior for $\beta_0$ is $N(0, 100 I_2)$.  For the negative binomial
benchmarks, $d$ is taken to be known.
%% My note: $d$ unknown will ``hurt'' results in that it will induce extra
%% correlation, which presumably will hurt PG,mu=10 results.

Four data sets for each type of response, along with two prior structures yields
eight MCMC simulations for each method of posterior inference.  Tables
\ref{tab:dynlogit-detail} and \ref{tab:dynnb-detail} provide a complete summary
of the results.

%-------------------------------------------------------------------------------
% dyn logit

\begin{table}
\small
\centering
%Bench-Dyn-06/Tables/table.bench-dynlogit-low-2-n-1-wout.ar
\begin{tabular}{l r r r r r r r r } 
         Method  &    time  &   ARate  &  ESS.min  &  ESS.med  &  ESS.max  & ESR.min  &  ESR.med  &  ESR.max  \\ 

\hline
\hline
 \multicolumn{9}{c}{$n=1$, $f = (1,2)$, known AR parameters} \\
\hline
         %Method  &    time  &   ARate  &  ESS.min  &  ESS.med  &  ESS.max  &  ESR.min  &  ESR.med  &  ESR.max  \\ 
             PG   &    28.66 &     1.00 &   8333.94 &   9388.18 &   9904.42 &    290.79 &    327.55 &    345.57 \\ 
           dRUM   &    29.13 &     1.00 &   5344.83 &   7550.10 &   9757.91 &    183.48 &    259.19 &    334.98 \\ 
           CUBS   &   612.14 &     0.71 &   2799.89 &   4219.25 &   5845.51 &      4.57 &      6.89 &      9.55
\\ %\end{tabular}

%Bench-Dyn-06/Tables/table.bench-dynlogit-low-2-n-1-with.ar
%\begin{tabular}{l r r r r r r r r } 
\hline
 \multicolumn{9}{c}{$n=1$, $f = (1,2)$, unknown AR parameters} \\
\hline
         %Method  &    time  &   ARate  &  ESS.min  &  ESS.med  &  ESS.max  &  ESR.min  &  ESR.med  &  ESR.max  \\ 
             PG   &    31.33 &     1.00 &   1855.82 &   7109.57 &   9739.31 &     59.24 &    226.96 &    310.89 \\ 
           dRUM   &    31.95 &     1.00 &   1535.90 &   5632.82 &   8921.27 &     48.07 &    176.30 &    279.22 \\ 
           CUBS   &   612.37 &     0.56 &    242.71 &    487.21 &    775.24 &      0.39 &      0.79 &      1.26
 \\ %\end{tabular}

%Bench-Dyn-06/Tables/table.bench-dynlogit-high-2-n-1-wout.ar
%\begin{tabular}{l r r r r r r r r } 
\hline
\multicolumn{9}{c}{$n=1$, $f = (1,1.1)$, known AR parameters} \\
\hline
         %Method  &    time  &   ARate  &  ESS.min  &  ESS.med  &  ESS.max  &  ESR.min  &  ESR.med  &  ESR.max  \\ 
             PG   &    28.60 &     1.00 &   8596.25 &   9402.62 &   9933.72 &    300.60 &    328.79 &    347.36 \\ 
           dRUM   &    29.13 &     1.00 &   6095.80 &   7675.76 &   9634.57 &    209.27 &    263.51 &    330.77 \\ 
           CUBS   &   620.95 &     0.72 &   4120.25 &   5531.41 &   6817.09 &      6.64 &      8.91 &     10.98
 \\ %\end{tabular}


%Bench-Dyn-06/Tables/table.bench-dynlogit-high-2-n-1-with.ar
%\begin{tabular}{l r r r r r r r r } 
\hline
\multicolumn{9}{c}{$n=1$, $f = (1,1.1)$, unknown AR parameters} \\
\hline
         %Method  &    time  &   ARate  &  ESS.min  &  ESS.med  &  ESS.max  &  ESR.min  &  ESR.med  &  ESR.max  \\ 
             PG   &    31.37 &     1.00 &    765.97 &   4636.70 &   9415.57 &     24.42 &    147.81 &    300.15 \\ 
           dRUM   &    31.96 &     1.00 &    656.58 &   3734.29 &   8079.58 &     20.55 &    116.85 &    252.82 \\ 
           CUBS   &   615.39 &     0.56 &    211.12 &    516.69 &    998.70 &      0.34 &      0.84 &      1.61
 \\ %\end{tabular}

%Bench-Dyn-06/Tables/table.bench-dynlogit-low-2-n-20-wout.ar
%\begin{tabular}{l r r r r r r r r } 
\hline
\multicolumn{9}{c}{$n=20$, $f = (1,2)$, known AR parameters} \\
\hline
        %Method  &    time  &   ARate  &  ESS.min  &  ESS.med  &  ESS.max  &  ESR.min  &  ESR.med  &  ESR.max  \\ 
             PG   &    73.54 &     1.00 &   7161.18 &   9238.58 &   9923.40 &     97.38 &    125.63 &    134.95 \\ 
           dRUM   &    28.87 &     1.00 &   3528.34 &   6959.36 &   9893.99 &    122.21 &    241.04 &    342.69 \\ 
           CUBS   &   650.67 &     0.51 &    727.34 &   2770.01 &   3699.75 &      1.12 &      4.26 &      5.69
 \\ %\end{tabular}


%Bench-Dyn-06/Tables/table.bench-dynlogit-low-2-n-20-with.ar
%\begin{tabular}{l r r r r r r r r } 
\hline
\multicolumn{9}{c}{$n=20$, $f = (1,2)$, unknown AR parameters} \\
\hline
        %Method  &    time  &   ARate  &  ESS.min  &  ESS.med  &  ESS.max  &  ESR.min  &  ESR.med  &  ESR.max  \\ 
             PG   &    76.26 &     1.00 &   1897.09 &   5482.92 &   9758.10 &     24.88 &     71.90 &    127.95 \\ 
           dRUM   &    31.69 &     1.00 &   1754.39 &   4003.81 &   9536.59 &     55.36 &    126.33 &    300.91 \\ 
           CUBS   &   662.42 &     0.47 &    498.89 &    953.03 &   1980.98 &      0.75 &      1.44 &      2.99
 \\ %\end{tabular}


%Bench-Dyn-06/Tables/table.bench-dynlogit-high-2-n-20-wout.ar
%\begin{tabular}{l r r r r r r r r } 
\hline
\multicolumn{9}{c}{$n=20$, $f = (1,1.1)$, known AR parameters} \\
\hline
         %Method  &    time  &   ARate  &  ESS.min  &  ESS.med  &  ESS.max  &  ESR.min  &  ESR.med  &  ESR.max  \\ 
             PG   &    73.16 &     1.00 &   7918.31 &   9416.01 &   9810.23 &    108.23 &    128.70 &    134.09 \\ 
           dRUM   &    28.88 &     1.00 &   4654.77 &   7771.95 &   9041.13 &    161.17 &    269.10 &    313.05 \\ 
           CUBS   &   659.75 &     0.63 &   3813.36 &   6375.18 &   6741.68 &      5.78 &      9.66 &     10.22
 \\ %\end{tabular}

%Bench-Dyn-06/Tables/table.bench-dynlogit-high-2-n-20-with.ar
%\begin{tabular}{l r r r r r r r r } 
\hline
\multicolumn{9}{c}{$n=20$, $f = (1,1.1)$, unknown AR parameters} \\
\hline
         %Method  &    time  &   ARate  &  ESS.min  &  ESS.med  &  ESS.max  &  ESR.min  &  ESR.med  &  ESR.max  \\ 
             PG   &    75.98 &     1.00 &    408.79 &   5735.76 &   9484.30 &      5.38 &     75.49 &    124.83 \\ 
           dRUM   &    31.71 &     1.00 &    418.48 &   4106.12 &   8269.33 &     13.20 &    129.48 &    260.76 \\ 
           CUBS   &   665.71 &     0.47 &    273.76 &   1290.78 &   1893.79 &      0.41 &      1.94 &      2.84
 \end{tabular}

 \caption{\label{tab:dynlogit-detail} The dynamic logit benchmarks: $n$
   corresponds to the number of trials for each response; $f$ corresponds to the
   frequency used when constructing the covariates.  When $f = (1,2)$ the
   covariates are less correlated when $f=(1,1.1)$ the covariates are more
   correlated.  The PG method does well when $n=1$.  The FS method does well
   when $n=20$.  Time refers to the time taken to produce 10,000 post burn-in
   samples.  ARate refers to the Metroplis-Hastings acceptance rate.}
\end{table}

%-------------------------------------------------------------------------------
% dyn-nb

\begin{table}
\small
\centering
%Bench-Dyn-06/Tables/table.bench-dynnb-low-2-mu-10-wout.ar
\begin{tabular}{l r r r r r r r r } 
Method  &    time  &   ARate  &  ESS.min  &  ESS.med  &  ESS.max  &  ESR.min  & ESR.med  &  ESR.max \\
\hline
\hline
 \multicolumn{9}{c}{$\mu=10$, $f = (1,2)$, known AR parameters} \\
\hline
         %Method  &    time  &   ARate  &  ESS.min  &  ESS.med  &  ESS.max  &  ESR.min  &  ESR.med  &  ESR.max  \\ 
             PG   &    42.50 &     1.00 &   7972.97 &   9649.48 &   9947.01 &    187.60 &    227.04 &    234.04 \\ 
             FS   &    29.54 &     1.00 &   1861.10 &   6530.79 &   9831.67 &     62.99 &    221.05 &    332.77 \\ 
           CUBS   &   655.71 &     0.61 &    800.95 &   3517.15 &   5603.81 &      1.22 &      5.36 &      8.55
 \\ %\end{tabular}

%Bench-Dyn-06/Tables/table.bench-dynnb-low-2-mu-10-with.ar
%\begin{tabular}{l r r r r r r r r } 
\hline
 \multicolumn{9}{c}{$\mu=10$, $f = (1,2)$, unknown AR parameters} \\
\hline
         %Method  &    time  &   ARate  &  ESS.min  &  ESS.med  &  ESS.max  &  ESR.min  &  ESR.med  &  ESR.max  \\ 
             PG   &    45.29 &     1.00 &    156.63 &   4852.66 &   9812.26 &      3.46 &    107.14 &    216.67 \\ 
             FS   &    32.31 &     1.00 &    135.06 &   1856.03 &   7837.19 &      4.18 &     57.45 &    242.58 \\ 
           CUBS   &   674.23 &     0.55 &    116.36 &   1223.44 &   2700.52 &      0.17 &      1.81 &      4.01
 \\ %\end{tabular}

%Bench-Dyn-06/Tables/table.bench-dynnb-high-2-mu-10-wout.ar
%\begin{tabular}{l r r r r r r r r } 
\hline
 \multicolumn{9}{c}{$\mu=10$, $f = (1,1.1)$, known AR parameters} \\
\hline
         %Method  &   time &  ARate & ESS.min & ESS.med & ESS.max & ESR.min & ESR.med & ESR.max \\ 
            PG  &    42.39 &     1.00 &   9142.96 &   9692.30 &   9943.59 &    215.67 &    228.62 &    234.55 \\ 
            FS  &    29.55 &     1.00 &   2820.14 &   5898.99 &   9765.03 &     95.43 &    199.61 &    330.43 \\ 
          CUBS  &   656.58 &     0.73 &   4509.25 &   7276.52 &   7891.49 &      6.87 &     11.08 &     12.02

 \\ %\end{tabular}

%Bench-Dyn-06/Tables/table.bench-dynnb-high-2-mu-10-with.ar
%\begin{tabular}{l r r r r r r r r } 
\hline
 \multicolumn{9}{c}{$\mu=10$, $f = (1,1.1)$, unknown AR parameters} \\
\hline
         %Method  &    time  &   ARate  &  ESS.min  &  ESS.med  &  ESS.max  &  ESR.min  &  ESR.med  &  ESR.max  \\ 
             PG   &    45.16 &     1.00 &    276.72 &   1725.30 &   9555.14 &      6.13 &     38.20 &    211.58 \\ 
             FS   &    32.28 &     1.00 &    263.56 &   1164.83 &   5458.53 &      8.16 &     36.08 &    169.10 \\ 
           CUBS   &   677.20 &     0.62 &    405.83 &   1438.07 &   3812.09 &      0.60 &      2.12 &      5.63
 \\ %\end{tabular}

%Bench-Dyn-06/Tables/table.bench-dynnb-low-2-mu-100-wout.ar
%\begin{tabular}{l r r r r r r r r } 
\hline
 \multicolumn{9}{c}{$\mu=100$, $f = (1,2)$, known AR parameters} \\
\hline
         %Method  &   time &  ARate & ESS.min & ESS.med & ESS.max & ESR.min & ESR.med & ESR.max \\ 
            PG  &    38.56 &     1.00 &   1860.03 &   7287.72 &   9927.30 &     48.23 &    188.99 &    257.43 \\ 
            FS  &    29.17 &     1.00 &   2352.61 &   7301.91 &   9827.50 &     80.65 &    250.31 &    336.89 \\ 
          CUBS  &   669.34 &     0.38 &    228.97 &   1226.28 &   2833.25 &      0.34 &      1.83 &      4.23

 \\ %\end{tabular}

%Bench-Dyn-06/Tables/table.bench-dynnb-low-2-mu-100-with.ar
%\begin{tabular}{l r r r r r r r r } 
\hline
 \multicolumn{9}{c}{$\mu=100$, $f = (1,2)$, unknown AR parameters} \\
\hline
         %Method  &    time  &   ARate  &  ESS.min  &  ESS.med  &  ESS.max  &  ESR.min  &  ESR.med  &  ESR.max  \\ 
             PG   &    41.33 &     1.00 &   1798.58 &   4986.44 &   9178.54 &     43.51 &    120.63 &    222.06 \\ 
             FS   &    32.28 &     1.00 &   2098.04 &   4727.49 &   9132.24 &     65.00 &    146.45 &    282.90 \\ 
           CUBS   &   677.94 &     0.32 &    459.22 &    799.84 &   1194.99 &      0.68 &      1.18 &      1.76
 \\ %\end{tabular}

%Bench-Dyn-06/Tables/table.bench-dynnb-high-2-mu-100-wout.ar
%\begin{tabular}{l r r r r r r r r } 
\hline
 \multicolumn{9}{c}{$\mu=100$, $f = (1,1.1)$, known AR parameters} \\
\hline
         %Method  &    time  &   ARate  &  ESS.min  &  ESS.med  &  ESS.max  &  ESR.min  &  ESR.med  &  ESR.max  \\ 
             PG   &    40.73 &     1.00 &   4061.25 &   7265.75 &   9856.66 &     99.72 &    178.41 &    242.03 \\ 
             FS   &    29.44 &     1.00 &   3396.35 &   6938.80 &   9867.25 &    115.36 &    235.66 &    335.13 \\ 
           CUBS   &   668.03 &     0.42 &   1676.35 &   3429.40 &   4146.91 &      2.51 &      5.13 &      6.21
 \\ %\end{tabular}

%Bench-Dyn-06/Tables/table.bench-dynnb-high-2-mu-100-with.ar
%\begin{tabular}{l r r r r r r r r } 
\hline
 \multicolumn{9}{c}{$\mu=100$, $f = (1,1.1)$, unknown AR parameters} \\
\hline
         %Method  &    time  &   ARate  &  ESS.min  &  ESS.med  &  ESS.max  &  ESR.min  &  ESR.med  &  ESR.max  \\ 
             PG   &    43.57 &     1.00 &   1186.07 &   3214.67 &   6880.24 &     27.22 &     73.79 &    157.93 \\ 
             FS   &    32.25 &     1.00 &   1115.12 &   3071.79 &   6879.66 &     34.58 &     95.25 &    213.31 \\ 
           CUBS   &   682.59 &     0.30 &    412.62 &    723.17 &   1028.33 &      0.60 &      1.06 &      1.51
 \end{tabular}
 \caption{\label{tab:dynnb-detail} The dynamic logit benchmarks: $\mu$
   corresponds to the mean of the response; $f$ corresponds to the frequency
   used when constructing the covariates.  When $f = (1,2)$ the covariates are
   less correlated when $f=(1,1.1)$ the covariates are more correlated.  The PG
   method does well when $\mu=10$.  The FS method does well when $\mu=100$. Time
   refers to the time taken to produce 10,000 post burn-in samples.  ARate
   refers to the Metroplis-Hastings acceptance rate.}
\end{table}

%-------------------------------------------------------------------------------

% If you have a bibliography.
% The file withe bibliography is name.bib.
\bibliography{bayeslogit}{}
\bibliographystyle{abbrvnat}

\end{document}

% \begin{tabular}{l r r r r r r r r } 
% \hline
% \multicolumn{9}{c}{Tokyo Rainfall} \\
% \hline
%           Method  &     time &    ARate &  ESS.min &  ESS.med &  ESS.max &  ESR.min &  ESR.med &  ESR.max \\ 
%             CUBS  &   233.98 &     1.00 &   652.86 &   761.98 &   910.62 &     2.79 &     3.26 &     3.89 \\ 
%               PG  &    21.72 &     1.00 &  4352.80 &  7735.08 & 10081.46 &   200.43 &   356.18 &   464.22 \\ 
%             dRUM  &    20.60 &     1.00 &  1627.61 &  3649.46 &  5428.93 &    79.03 &   177.19 &   263.60
%  \end{tabular}
% \end{table}

% \begin{table}
% \centering
% \begin{tabular}{l r r r r r r r r } 
% \hline
% \multicolumn{9}{c}{Low corr X: $P$=2, $T$=400.  Estimate $\alpha, \beta$.} \\
% \hline
%           Method  &    time &  ARate & ESS.min & ESS.med & ESS.max & ESR.min & ESR.med & ESR.max \\ 
%              PG  &    22.59 &     1.00 &   7367.68 &   9229.91 &  10189.37 &    326.10 &    408.52 &    450.97 \\ 
%            dRUM  &    23.19 &     1.00 &   2759.83 &   7260.46 &   9967.66 &    118.99 &    313.04 &    429.77
%          \end{tabular}

% \begin{tabular}{l r r r r r r r r }
% \hline
% \multicolumn{9}{c}{High corr X: $P$=2, $T$=400. Estimate $\alpha, \beta$.} \\
% \hline 
%           Method  &    time &  ARate & ESS.min & ESS.med & ESS.max & ESR.min & ESR.med & ESR.max \\ 
%              PG  &    22.52 &     1.00 &   8494.04 &   9662.06 &  10225.35 &    377.13 &    428.98 &    454.00 \\ 
%            dRUM  &    23.17 &     1.00 &   4963.68 &   8105.02 &   9936.46 &    214.26 &    349.86 &    428.91
% \end{tabular}

% \end{table}


% \begin{table}
% \centering
% \begin{tabular}{l r r r r r r r r } 
% \hline
% \multicolumn{9}{c}{Low corr X: $P$=2, $T$=400.  Estimate $\alpha, \beta, \phi, W$.} \\
% \hline
%           Method  &   time &  ARate & ESS.min & ESS.med & ESS.max & ESR.min & ESR.med & ESR.max \\ 
%             PG  &    25.04 &     1.00 &    643.23 &   6531.34 &  10420.88 &     25.69 &    260.82 &    416.13 \\ 
%           dRUM  &    25.73 &     1.00 &    430.79 &   4490.34 &  10124.47 &     16.74 &    174.55 &    393.48
%  \end{tabular}

% \begin{tabular}{l r r r r r r r r } 
% \hline
% \multicolumn{9}{c}{High corr X: $P$=2, $T$=400. Estimate $\alpha, \beta, \phi, W$.} \\
% \hline
%           Method  &   time &  ARate & ESS.min & ESS.med & ESS.max & ESR.min & ESR.med & ESR.max \\ 
%             PG  &    25.16 &     1.00 &    988.83 &   5654.23 &  10337.07 &     39.31 &    224.75 &    410.87 \\ 
%           dRUM  &    25.84 &     1.00 &   1040.12 &   5496.94 &   9760.64 &     40.25 &    212.74 &    377.75
%  \end{tabular}



