\documentclass[]{article}

\input{commands}

\usepackage{mathtools}
\usepackage{cases}
\usepackage{algorithm}
%\usepackage{algorithmic}
\usepackage{algpseudocode}
% \usepackage{algcompatible}
\usepackage{outlines}
\usepackage{natbib}

\newcommand{\PG}{\text{PG}}
\newcommand{\Pois}{\text{Pois}}
\newcommand{\Ga}{\text{Ga}}
\newcommand{\NB}{\text{NB}}
\newcommand{\floor}[1]{\lfloor #1 \rfloor}
\newcommand{\eqd}{\,{\buildrel d \over =}\,}
\newcommand{\KS}{\text{KS}}

\newcommand{\graddel}[1]{\frac{\del}{\del #1}}
\newcommand{\hessdel}[2]{\frac{\del^2}{\del #1 \del #2}}

% Page Layout
\setlength{\oddsidemargin}{0.0in}
\setlength{\textwidth}{6.5in}
\setlength{\textheight}{8in}
%\parindent 0in
%\parskip 12pt

% Set Fancy Style
%\pagestyle{fancy}

%\lhead{Left Header}
%\rhead{Right Header}

\begin{document}

\tableofcontents

% Change Font and Spacing
\large % change the font size to 12pt
\linespread{1.1} % change the line spacing

% Set the counter
\setcounter{section}{0}

\section{Logistic Regression}

Suppose the probability of a success is
\[
p = 1 / (1 + \exp(-x \beta)).
\]
Then the probability of $n y$ successes in $n$ trials is
\[
{ny \choose n}
\Big(\frac{1}{1+e^{-x\beta}}\Big)^{ny}
\Big(\frac{e^{-x\beta}}{1+e^{-x\beta}}\Big)^{n-ny}
\]
which is equivalent to
\[
{ny \choose n} \Big( \frac{e^{yx\beta}}{1 + e^{x\beta}} \Big)^{n}.
\]
Repeating this $N$ times we have
\[
p(y|\beta) = \prod_{i=1}^N {n_i y_i \choose n_i}
\Big( \frac{e^{y_i x_i \beta}}{1 + e^{x_i \beta}} \Big)^{n_i}.
\]
Hence
\[
\ell(\beta | y) \propto
\prod_{i=1}^N
\Big( \frac{e^{y_i x_i \beta}}{1 + e^{x_i \beta}} \Big)^{n_i}.
\]

We want to write this as a (normal) mixture.  The key is express the likelihood
as
\[
\propto e^{u \beta} \prod_{i} \cosh^{-n_i} (x_i \beta / 2), \; u = \sum_{i} n_i
(y_i - 1/2) x_i,
\]
which we can do since
\[
\frac{e^{-x_i \beta / 2}}{e^{-x_i \beta / 2}} \Big( \frac{e^{y_i x_i \beta}}{1 +
  e^{x_i \beta}} \Big)
= \frac{e^{(y_i - 1/2) x_i \beta}}{e^{-x_i \beta / 2} + e^{x_i \beta / 2}}
= \frac{e^{(y_i - 1/2) x_i \beta}}{ 2 \cosh(x_i \beta / 2) }.
\]
A clever choice of prior can be treated as an extra data point, which simply
alters the above description by taking a product over $i=0, \ldots, N$.  In that
case
\[
p(\beta | y) \propto e^{u \beta} \prod_{i=0}^N \cosh^{-n_i} (x_i \beta / 2).
\]

Someone clever noticed that $\cosh^{-n}$ is related to the MGF (Laplace
transform) of the Jacobi distribution.  In particular, if $J^*$ is as from
Devroye, then
\[
\bbE[e^{-\lambda J^*}] = \cosh^{-1}(\sqrt{2 \lambda})
\]
and the MFG of a sum of $n$ independent $J^*$s is
\[
\bbE[e^{-\lambda J^*(n)}] = \cosh^{-n}(\sqrt{2 \lambda}).
\]

Thus, we can rewrite the posterior for $\beta$ as
\begin{equation}
\label{post1}
\propto e^{u \beta} \prod_{i=0}^N \exp \{-\frac{1}{2} \Big(\frac{z_i}{2} \Big)^2
\omega_i \} p_{J^*(n)}(\omega_i) \textmd{ where } z_i = x_i \beta.
\end{equation}

\textbf{UPDATE: I WAS WRONG THERE IS A CORRESPONDING DATA GENERATING
  REPRESENTATION}

Consider
\[
\Big( \frac{e^{\psi_i}}{1+e^{\psi_i}} \Big) \delta_1(y_i) + 
\Big( \frac{1}{1+e^{\psi_i}} \Big) \delta_0(y_i).
\]
We can write this as
\[
\Big( e^{\psi_i/2} \delta_1(y_i) + e^{-\psi_i/2} \delta_0(y_i) \Big) 2^{-1}
\cosh^{-1} (\psi_i / 2).
\]
Then the data augmentation is
\[
p(y_i, \omega_i | \psi_i) = \Big( e^{\psi_i/2} \delta_1(y_i) + e^{-\psi_i/2}
\delta_0(y_i) \Big) 2^{-1} \exp(-\omega_i \psi_i^2 / 2) p(\omega_i).
\]
Or if we want to factor it we have
\[
\Big( e^{\psi_i/2} \delta_1(y_i) + e^{-\psi_i/2} \delta_0(y_i) \Big) 2^{-1}
\cosh^{-1} (\psi_i / 2) \cosh(\psi_i / 2) \exp(-\omega_i \psi_i^2 / 2) p(\omega_i).
\]
In which case we can see that one generates $y_i$ by
\[
\begin{cases}
P(y_i = 1) = e^{\psi_i / 2} / (2 \cosh(\psi_i/2)) \\
\omega_i \sim \PG(1, \psi_i).
\end{cases}
\]
Notice that $\omega_i$ are independent to $y_i$.
This produces the correct posterior representation
\[
e^{\psi_i(y_i - 1/2)} 2^{-1} \cosh \ldots
\]
since
\[
\Big( e^{\psi_i/2} \delta_1(y_i) + e^{-\psi_i/2} \delta_0(y_i) \Big)
= e^{\psi_i(y_i - 1/2)}.
\]
Is it important that $\omega_i$ is independent from $y_i$?  So if I was to
synthetically generate the data $y$ would have no effect on $\omega$.

\section{Jacobi and Polya-Gamma}

A couple distributions will be helpful here.  Really we want the terms of the
product in (\ref{post1}) to look like
\[
\exp \Big( \frac{-1}{2} z^2 \omega^* \Big) p(\omega^*)
\]
so that we have a Normal-precision mixture.  We can do that by a change of
variables, $w / 4= w^*$.

If $w$ has distribution $J^*(n)$ with some density, then the density for
$w^*$ is
\[
p_{PG(n)}(\omega^*) = 4 p_{J^*(n)} (4 \omega^*).
\]
By our change of variables we know that
\[
PG(n) = \frac{1}{4} J^*(n).
\]
Define the density $p(\omega | J^*(n,z) )$ by the kernel
\[
\exp \{ \frac{-z^2}{2} \omega \} p_{J^*(n)}(\omega).
\]
Using the Laplace transform (of $J^*$) one can see the true density to be
\[
\cosh(z) \exp \{ \frac{-z^2}{2} \omega \} p_{J^*(n)}(\omega).
\]
Similarly, define the density $p(\omega | PG(n, z))$ by the kernel
\[
\exp \{ \frac{-z^2}{2} \omega \} p_{PG(n)}(\omega).
\]
Then the two RV's are related by (again $w^* = w/4$)
\[
p(\omega^* | PG(n, z)) = 4 \; p( 4 \omega^* | J^*(n,z/2) )
\]
and
\[
PG(n, z) = \frac{1}{4} J^*(n, z/2).
\]

This relationships will be useful because it lets us easily transfer the work of
Devroye and of Baine, Pitman, and Yor to what we are doing.  For instance, if we
can draw from $J^*(n, z/2)$ then we can immediately draw from $PG(n,z)$.
Furthermore, we can easily move between various representations using this
transformation.  As an example, let's consider deriving the sum of exponentials
representation for $J^*$.  To begin, a key observation is that
\[
\cosh(z) = \prod_{i=1}^\infty \Big(1 + \frac{z^2}{(n-1/2)^2 \pi^2} \Big)
\]
by the Wierstrauss factorization theorem.  Thus
\begin{align*}
\bbE[\exp(-\lambda J^*)]
& = 1 / \cosh(\sqrt{2 \lambda}) \\
& = \prod_{n=1}^\infty \Big(1 + \frac{\lambda}{(n-1/2)^2 \frac{\pi^2}{2}} \Big)^{-1}.
\end{align*}
Let $c_n = (n-1/2)^2 \frac{\pi^2}{2}$.  Since
\[
\bbE[\exp(-\lambda E/c)] = (1 + \frac{\lambda}{c})^{-1}
\]
the product above becomes
\[
\prod_{i=1}^\infty \bbE[\exp(-\lambda \frac{E_i(1)}{c_n})]
\]
which shows that
\[
J^* = \sum_{i=1}^\infty \frac{E_i(1)}{c_n}.
\]
Similarly, if you tilt $J^*$ by $\exp(-\alpha x)$ then, using the Laplace
transform identity
\[
\mathcal{L}(e^{-\alpha x} p(x))(\lambda) = \mathcal{L}(p(x))(\lambda + \alpha),
\]
we have
\[
\mathcal{L}(e^{-\alpha x} p_{J^*}(x))(\lambda)
=
\prod_{n=1}^\infty \Big( 1 + \frac{\lambda + \alpha}{c_n} \Big)^{-1}.
\]
Using the identity $\cosh(\sqrt{2\alpha}) = \prod_{i=1}^\infty (1 + \alpha /
c_n)$ we see that the MGF of the density
\[
\cosh(\sqrt{2\alpha}) e^{-\alpha x} p_{J^*}(x)
\]
is
\[
\prod_{i=1}^\infty (1 + \alpha / c_n) \Big( 1 + \alpha/c_n + \frac{\lambda}{c_n}
\Big)^{-1} =
\prod_{i=1}^\infty \Big(1+ \frac{\lambda}{c_n + \alpha}\Big)^{-1}
\]
and hence the the RV can be represented as
\[
\sum_{i=1}^\infty \frac{E_i(1)}{c_n + \alpha}.
\]
The same technique can be used to show that
\[
J^*(n,z) = \sum_{i=1}^\infty \frac{Ga(n, 1)}{c_n + z^2/2}
\]
and hence
\[
PG(n, z) = 2 \; \sum_{i=1}^\infty \frac{Ga(n,1)}{(n-1/2)^2 4 \pi^2 + z^2}.
\]

\section{Sampling from $J^*(n,z)$}

One could sample $J^*(n,z)$ using the sum of Gammas representation, but this is
rather slow.  For small, integral $n$ one can take advantage of
\[
J^*(n,z) = \sum_{i=1}^n J^*(1.0, z).
\]
We can sample $J^*(1.0, z)$ relatively quickly by augmenting the algorithm in
Devroye (2009).  Everything follows almost immediately from that work.  In particular,
we have that
\[
J^*(1.0, z) = \cosh(z) \exp(\frac{-1}{2} z^2 w) p_{J^*}(w).
\]

Really we can look at this as wanting to sample from the kernel
\[
f(w|z) = \exp(\frac{-1}{2} z^2 w) f(w)
\]
where $f(w)$ is the density for $J^*$.

Devroye's goal is to sample $f(w) = p_{J^*}(w)$.  He notes that there are two
representations for this density,
\begin{numcases}{f(w) =}
\label{jacobi:d1}
\pi \sum_{n=0}^\infty (-1)^n \big(n+\frac{1}{2} \big)
  \exp \big( (-n+1/2)^2 \pi^2 x / 2 \big) \\
\label{jacobi:d2}
(\frac{2}{\pi x})^{3/2} \pi \sum_{n=0}^\infty (n+\frac{1}{2})
  \exp \Big( - \frac{2(n+1/2)^2}{x} \Big),
\end{numcases}
which we may express as
\[
f(x) = \sum_{n=0}^\infty (-1)^n a_{n}(x)
\]
where $a_n(x)$ is positive for all $n$ and $x$.  All the conditions that Devroye
outlines holds for $\exp(-\alpha x) a_n(x)$ as well so the same technique
applies to generating a random variate with density defined by the kernel
\[
f(x|z) = \sum_{n=0}^\infty (-1)^n
\underbrace{\exp(-\frac{1}{2} z^2 x) a_n(x)}_{\tilde a_n(x)}.
\]
Devroye uses both representations of the density $f$ to sample from the Jacobi.
In particular, he observes the following:
\begin{enumerate}

\item Devroye's first lemma: $a_i(x) \leq a_{i+1}(x)$ is true when $x \geq (\log
  3)/\pi^2$ under representation (\ref{jacobi:d1}) and is true when $x \leq 4 /
  \log 3$ under representation (\ref{jacobi:d2}).

\item These intervals overlap.

\item For $x \geq (\log 3) / \pi^2$ and representation (\ref{jacobi:d1}) and for
  $x \leq 4 / \log 3$ and (\ref{jacobi:d2}) the partial sums $A_n(x)$ have
  \[
  A_0(x) \geq A_2(x) \geq \cdots \geq f(x) \geq \cdots \geq A_3(x) \geq A_1(x),
  \]
  which suggests we can use Von Neumann's alternating sum method (I guess).

\item Since $A_0(x) \geq f(x)$ for all $x$ and motivates the proposal:
  Representation (\ref{jacobi:d1}) suggests using an exponential proposal while
  representation (\ref{jacobi:d2}) suggests using an inverse $\chi^2$.

\item In particular, the proposal kernel is defined by
\[
f(x) \leq A_0(x) = g(x) =
\begin{dcases}
\frac{\pi}{2} \exp \Big( - \frac{\pi^2 x}{8} \Big), & x \geq t \\
\Big( \frac{2}{\pi x} \Big)^{3/2} \frac{\pi}{2} \exp \Big(- \frac{1}{2x}\Big), &
x \leq t,
\end{dcases}
\]
where $t = 2 / \pi$ is a threshold found in the intersection of the two
aforementioned intervals for which the value of the two representations are
identical, thus allowing for a continuous pasting of proposals.

\end{enumerate}

The same process applies to the tilted kernel $f(\omega | z)$ defined by
\[
f(x|z) = \exp(-\frac{1}{2} z^2 w) f(x)
\]
and corresponding proposal kernel
\[
g(x|z) = \exp(-\frac{1}{2} z^2 w) g(x).
\]
This produces
\[
f(x|z) \leq g(x|z) =
\begin{dcases}
\frac{\pi}{2} \exp \Big( - \Big[ \frac{\pi^2}{8} + \frac{z^2}{2} \Big] w \Big)
& x \geq t
\\
\Big( \frac{2}{\pi x} \Big)^{3/2}
\frac{\pi}{2} \exp \Big( \frac{-1}{2x} - \frac{z^2}{2} x \Big)
& x \leq t,
\end{dcases}
\]
The former is an exponential kernel and the latter is an Inverse-Gaussian
kernel.  We want to draw from the exponential when we are below a certain
threshold and from the inverse-gaussian when we are above a certain threshold.

To sample from this proposal one first flips a coin to determine which piece of
the proposal will be used.  The weight of this coin depends on $z$.  Thus we
need to figure out how much each piece contributes to the total mass in terms of
$z$.  The (unnormalized) weight of the first piece is
\[
p(z) = \int_{t}^\infty g(x|z) dx =
  \frac{\pi}{2}
  \frac{1}{\frac{\pi^2}{8} + \frac{z^2}{2}}
  \exp \Big( - \Big[ \frac{\pi^2}{8} + \frac{z^2}{2} \Big] t \Big).
\]
For the second piece we need to match parameters with the Inverse-Gaussian.  The
density for an inverse-Gaussian RV is
\[
IG(x \mid \mu, \lambda) = \Big(\frac{\lambda}{2\pi x^3}\Big)^{1/2}
\exp \Big( \frac{-\lambda (x - \mu)^2)}{2 \mu^2 x} \Big).
\]
Our expression above is tied to this density by
\begin{align*}
\frac{-1}{2x} - \frac{z^2}{2}x
& = \frac{-z^2}{2x} (x^2 - 2x|z|^{-1} + 2x|z|^{-1} + z^{-2}) \\
& = \frac{-z^2}{2x} (x - |z|^{-1})^2 - |z|.
\end{align*}
Thus the (unnormalized) weight of the second piece is
\[
q(z) = 2 \exp(-|z|) \int_{0}^t IG(x \mid \mu=|z|^{-1}, \lambda=1.0) dx.
\]

The program is as follows: use the first density if $U < p / (p+q)$, the
probability that drawing from $g*(x|z)$ you draw $x < t$, otherwise, use the
second density.  Propose using the chosen piece.  Use the alternating series
method.  Drawing from a truncated exponential is straightforward.  Drawing from a
truncated inverse Gamma is less straightforward.

We break the $IN$ draw up into two separate scenarios.  Nic suggests using
method (1) if $\mu > t \; (Z < 1/t)$ and using (2) if $\mu < t \; (Z > 1/t)$.

\begin{enumerate}

\item When $Z$ is small do accept/reject with a proposal from a \(IG(1/2,
  scale=1/2) \bbI_{(0,t)} = 1 / \chi^2_1 \bbI_{(0,t)}\).  If we consider the ratio of
  the kernels
  \[
  \frac{x^{-3/2} \exp \Big( \frac{-1}{2x} - \frac{z^2}{2}x \Big)}
  {x^{-3/2} \exp \Big( \frac{-1}{2x} \Big)}
  = \exp \Big( \frac{-z^2}{2} x \Big)
  \]
  we see that the $M$, the supremem of the above ratio, is one and hence the
  acceptance probability is
  \[
  \exp(\frac{-z^2}{2} x).
  \]

  To sample from a truncated inverse $\chi^2_1$ Devroye generates from a
  truncated normal, $x \sim N \bbI_{[1/\sqrt{t}, \infty)}$, to generate $x \sim
  1/\chi^2_1 \bbI_{(0,t]}$.  To generate a truncated normal Devroye (2009) follows
  Devroye (1986, p. 382).  In particular, keep generating independent pairs $(E,
  E')$ until $E^2 \leq 2 E' / t$, then return $(1 + tE) / \sqrt{t}$.

\item When $Z$ is large, we do a rejection algorithm with an Inverse-Gaussian
  proposal.  Recall that $Z$ is inverse mean, so large $Z$ corresponds to small
  mean.  Here we follow Devroye (1986) wherein one finds that you can generate
  an Inverse-Gaussian using a $\chi^2$ distribution or a normal.  In particular,
  if $X \sim IN(\mu, \lambda)$ then
  \[
  Y = \frac{\lambda(X-\mu)^2}{\mu^2 X} = \chi_1^2 = N^2.
  \]
  Solving for $Y$ we have two solutions, corresponding to $X < \mu$ and $X >
  \mu$; respectively,
  \[
  X_1 = \mu + \frac{\mu^2 Y}{2\lambda} - \frac{\mu}{2\lambda} \sqrt{4 \mu \lambda
    Y + \mu^2 Y^2} \textmd{ and } X_2 = \frac{\mu^2}{X_1}.
  \]

  To generate $Y$, you draw $Y \sim N^2$ and then set $X_1$ from $Y$ as
  described above.  Draw $U \sim U$.  If $U \leq \frac{\mu}{\mu + X_1}$ then
  return $X = X_1$, otherwise return $X = \mu^2 / X_1$ (Devroye, 1986, p. 149).

  To draw the truncated inverse-Gaussian we use rejection.  We know that $\mu <
  t$ so if we find $U < \frac{\mu}{\mu + X_1}$ we can immediately accept as $X_1
  < \mu$.  Otherwise we must check that $X = \mu^2 / X_1 < t$.

  \textbf{Many-to-one Sampling} (Deroye, 1986, p. 145): Suppose I can simulate
  $Y$ and $Y = \Psi(X)$ where $\Psi^{-1}$ is many-to-one.  I want to simulate
  $X$.  For the moment, let's assume that $\Psi^{-1}$ is two-to-one, for
  instance, it could be quadratic.  Further suppose that $\Psi'$ is of one sign
  on $(-\infty, T)$ and of the other sign on $(T, \infty)$ and that
  \[
  x =
  \begin{cases}
    l(y) & x < T, \\
    r(y) & x > T.
  \end{cases}
  \]
  Suppose $x$ has density $f$.  Then the densit of $y$ can be calculated using
  the change of variables formual and is
  \[
  h(y) = \underbrace{|l'(y)| f(l(y))}_{p_l(y)}
       + \underbrace{|r'(y)| f(r(y))}_{p_r(y)}.
  \]
  Then one can sample $X$ by sampling $Y$ and then setting $X := l(Y)$ with
  probabiltiy $p_l(y) / (p_l(y) + p_r(y))$ and $X := r(y)$ otherwise.  In
  essence you know, given the value of $Y$ that $X$ can only take on two values,
  the distribution of which is given by $(p_l(y), p_r(y)) / h(y)$.

\end{enumerate}

Once we have sampled our proposal we can use the series method (Devroye, 1986,
p. 151).

\noindent \textbf{Time Series Method}: Copying from Devroye (1986).  We need $Y$
with density $f$ to be approximated by $f_n \nearrow f$ and $g_n \searrow f$ and
to have some easy proposal $h$ for which $f \leq c h$ where $c$ is larger than
but preferably close to one.  Then you can generate $Y$ by the following:
\begin{algorithmic}
\Repeat
\State Generate $X \sim h$.
\State $Y \leftarrow \mcU(0, c h(X))$
\State $n \leftarrow 0$
\Repeat
\State $n \leftarrow n + 1$
\State If $Y \leq f_n(X)$ then return $X$.
\Until {$Y > g_n(X)$}
\Until {FALSE}
\end{algorithmic}
The key idea, I think, is that the second loop is such that you are really
checking that $Y \leq f(X)$.  So really, this is a fancy way to sample $Y \sim
\mcU(0,c h(X))$ and then check that $Y \leq f(X)$, which is just accept-reject
sampling. (Check, do I want $\leq$? what happens with discrete distributions.)

Now, when the density can be expressed an alternating series, as we have in this
case, the same method applies, except that you can now calculate $f_n$ and
$g_n$, which in this case are partial sums, on the fly.  In particular, the
algorithm is (again, taking from Devroye... almost, p. 152 is slightly
different):
\begin{algorithmic}
\Repeat
\State $X \sim h$, the proposal
\State $n \leftarrow 0, S \leftarrow a_0(X)$.
\State $Y \leftarrow \mcU(0, c h(X))$ (same as $\mcU(0,1) S$ when $g = A_0$)
\Repeat
\State $n \leftarrow n + 1$
\If {$n$ is odd}
\State $S \leftarrow S - a_n(X)$, if $Y < S$, then return $X$
\Else
\State $S \leftarrow S + a_n(X)$, if $Y > S$, then break
\EndIf
\Until {FALSE}
\Until {FALSE}
\end{algorithmic}


\section{Sampling a Polya-Gamma$(n, z)$}

From above, we know that
\[
PG(n, z) = \frac{1}{4} J^*(n,z/2).
\]
Thus, to sample $PG(n, z)$ where $n$ is a natural number we can sum $n$
independent samples of $PG(1,z)$.  The sampling of $PG(1,z)$ can then be adapted
from our discussion above and Devroye as follows:

\begin{algorithm}
\label{alg:tiltedpg}
\caption{Sampling from $PG^*(1,z)$}
\begin{algorithmic}
\State \textbf{Input}: $z$, a real number
\State \textbf{Define}: \texttt{pigauss}, the CDF of the inverse Gaussian distribution
\State \textbf{Define}:  $a_n(x)$, the peicewise defined coefficients found
  in Devroye (2009, p.5)
\State $z \leftarrow |z| / 2$,
       $t \leftarrow 0.64$,
       $K \leftarrow \pi^2 / 8 + z^2 / 2$
\State $p \leftarrow \frac{\pi}{2 K} \exp(-K t)$
\State $q \leftarrow 2 \exp(-|z|) \; \texttt{pigauss}(t, \mu = 1/z, 1.0)$
% \STATE Set $p, q$ given $z$
\Repeat
\State Generate $U, V \sim \mathcal{U}(0,1)$
\If{$U < p / (p+q)$}
\State $X \leftarrow t + E/K$ where $E \sim \mathcal{E}(1)$
\Else
%\STATE $X \sim \mathcal{IN}(\mu=1/z, 1.0) \bbI_{(0,t)}$
\State $\mu \leftarrow 1/z$
\If {$\mu > t$}
\Repeat
\State Generate $1/X \sim \chi^2_1 \bbI_{(t,\infty)}$
\Until {$\mathcal{U}(0,1) < \exp(-\frac{z^2}{2} X)$}
\Else
\Repeat
\State Generate $X \sim \mathcal{IN}(\mu, 1.0)$
\Until {$X < t$}
\EndIf
% \STATE if $\mu > t$ generate $1/W \sim \chi^2_1 \bbI_{(t,\infty)}$ until $\mathcal{U}(0,1) < \exp(-\frac{z^2}{2} W)$
% \STATE else generate $W \sim \mathcal{IN}(\mu, 1.0) \bbI_{(0,t)}$
\EndIf
\State $S \leftarrow a_0(X)$, $Y \leftarrow VS$, $n \leftarrow 0$
\Repeat
\State $n \leftarrow n + 1$
\If{$n$ is odd}
\State $S \leftarrow S - a_n(X)$; \textbf{if} $Y < S$, \textbf{then} \Return $X$
/ 4
\Else
\State $S \leftarrow S + a_n(X)$; \textbf{if} $Y > S$, \textbf{then} \textbf{break}
\EndIf
\Until {FALSE}
\Until {FALSE}
\end{algorithmic}
\end{algorithm}

\section*{Posterior Inference}

\subsection*{Gibbs}

\subsubsection*{Polychotomous Data}

Imagine that we are drawing from a multinomial distribution, i.e. $\mb{y} \sim MN(n,
\{p_i\}_{i=1}^J)$ has pmf
\[
\frac{n!}{y_1! \cdots y_J !} p_1^{y_1} \cdots p_J^{y_J} \; \textmd{ where } \;
n = \sum_{j=1}^J y_j.
\]
Then the likelihood for a single draw is
\[
\ell(\mb{p} | \mb{y}) = p_1^{y_1} \cdots p_J^{y_J} \textmd{ where }
\sum_{j=1}^J p_j = 1 \textmd{ and } \sum_{j=1}^J y_j = n.
\]
Taking into account the constraints we can also write this as
\[
\ell(\mb{p}|\mb{y}) = p_1^{y_1} \cdots p_{J-1}^{y_{J-1}} (1 - \sum_{k < J}
p_k)^{n-\sum_{k<J} y_k}.
\]

The logistic transformation is
\[
p_j = \frac{e^\psi_j}{\sum_{j} e^{\psi_j}}.
\]
This is not uniquely identified.  If we set $\psi_J=0$ then this becomes
identified and we can interpret $\psi_j$ as the log odds of $j$ against $J$,
i.e.
\[
\psi_j = \log(p_j / p_J).
\]
When we model $\psi_j = \mb{x} \beta_j$ then we can interpret $\mb{x} \beta_j$ as the
increase or decrease in the log odds over even odds, i.e. $\psi_j = 0, j=1,
\ldots, J$.

We want to sample from the posterior of $\beta$ using our PG trick and the
method of Holmes and Held.  That is we want to sample $\psi_j | \psi_{-j},
\omega_j$.  It will be easier to work here if we write the likelihood without
invoking the constraints.  In particular, we know that
\[
\ell(\mb{p} | \mb{y}) \propto p_1^{y_1} \cdots p_J^{y_J}.
\]
Writing this in terms of the log-odds, the likelihood for $\psi$ is
\[
\ell(\psi | \mb{y}) \propto \prod_j \Big(\frac{e^{\psi_j}}{\sum_{k} e^{\psi_k}}
\Big)^{y_j}.
\]
Conditioning on $\psi_{-j}$ (we implicitly condition on $\mb{y}$)
\[
\ell(\psi_j | \psi_{-j}) \propto 
\Big(\frac{e^{\psi_j}}{\sum_{k} e^{\psi_k}} \Big)^{y_j}
\Big(\frac{1}{\sum_k e^{\psi_k}} \Big)^{n-y_j}
\]
Letting $e^{c_j} = \sum_{k \neq j} e^{\psi_j}$ we have
\begin{align*}
\ell(\psi_j | \psi_{-j}) & \propto 
\Big(\frac{e^{\psi_j}}{e^c_j + e^{\psi_j}} \Big)^{y_j}
\Big(\frac{1}{e^{c_j} + e^{\psi_j}} \Big)^{n-y_j} \\
& \propto
\Big(\frac{e^{\psi_j - c_j}}{1 + e^{\psi_j - c_j}} \Big)^{y_j}
\Big(\frac{1}{1 + e^{\psi_j-c_j}} \Big)^{n-y_j} \\
& \propto
\Big(\frac{e^{\eta_j}}{1 + e^{\eta_j}} \Big)^{y_j}
\Big(\frac{1}{1 + e^{\eta_j}} \Big)^{n-y_j}.
\end{align*}
We recognize this as the likelihood found in logistic regression with binary
outcome.  Incorporating an $i$ indicy for multiple observations, this becomes
\[
\ell(\psi_{ij} | \psi_{i,-j}) = 
\prod_{i=1}^N
\Big(\frac{e^{\eta_{ij}}}{1 + e^{\eta_{ij}}} \Big)^{y_{ij}}
\Big(\frac{1}{1 + e^{\eta_{ij}}} \Big)^{n_i-y_{ij}}.
\]
where
\[
\eta_{ij} = \psi_{ij} - c_{ij} \; \textmd{ and } \; 
e^{c_{ij}} = \sum_{k \neq j} e^{\psi_{ik}}.
\]
Now employing the PG trick we have
\begin{align*}
\ell(\psi_{ij} | \psi_{i,-j}) & \propto 
\prod_{i=1}^N \Big( \frac{e^{\eta_{ij} y_{ij}/n_i}}{1 + e^{\eta_{ij}}} \Big)^{n_i} \\
& \propto
\prod_{i=1}^N \Big( \frac{e^{\eta_{ij} y_{ij}/n_i - \eta_{ij}/2}}{e^{-\eta_{ij}/2}
  + e^{\eta_{ij}/2}} \Big)^{n_i} \\
& \propto
\prod_{i=1}^N e^{\kappa_{ij} \eta_{ij}} \cosh^{-n_i}(\eta_{ij}).
\end{align*}
where $\kappa_{ij} = (y_{ij}/n_i - 1/2) n_i$.  (Note: can be precomputed.)
Now employing the PG trick we have
\[
\ell(\psi_{ij}, \omega_{ij} | \psi_{i,-j}) \propto
\prod_{i=1}^N e^{\kappa_{ij} \eta_{ij}} e^{-\frac{\eta_{ij}^2}{2} \omega_{ij}}
pg(\omega_{ij} | n_{i}, 0).
\]
Conditional upon $\omega_{ij}$,
\begin{align*}
\ell(\psi_{ij} | \psi_{i,-j}, \omega_{ij}) 
& \propto
\prod_{i=1}^N \exp \Big( \kappa_{ij} \eta_{ij} - \frac{1}{2} \eta_{ij}^2
\omega_{ij} \Big) \\
& \propto
\exp \Big( \sum_{i=1}^N 
\kappa_{ij} (x_i \beta_j - c_{ij}) - 
\frac{1}{2} \omega_{ij} (x_i\beta_j - c_{ij})^2 \Big) \\
& \propto
\exp \Big( \frac{-1}{2} \sum_{i=1}^N -2 \kappa_{ij} (x_i \beta_j - c_{ij}) +
\omega_{ij} (x_i \beta_j - c_{ij})^2 \Big)
\end{align*}
Expanding the quadratic term, we can wewrite the sum as
\[
\sum_{i=1}^N \omega_{ij} (x_i \beta_j)^2 - 2 (\kappa_{ij} + \omega_{ij} c_{ij})
x_i \beta_j + \textmd{the rest}.
\]
which is
\[
\beta_j' X' \Omega_j X \beta_j - 2 (\kappa_{\cdot,j} + \Omega_j c_{\cdot, j}) X
\beta_j + \textmd{the rest}.
\]
If we have a normal prior for $\beta_j$ we can now calculate the posterior in a
conjugate way.

Conditioning on $\psi_{ij}$ we have
\begin{align*}
\ell(\omega_{ij} | \psi_{ij}, \psi_{i,-j}) & \propto
\prod_{i=1}^N e^{-\eta_{ij}^2/2 \omega_{ij}} pg(\omega_{ij} | n_i, 0) \\
& = \prod_{i=1}^N pg(\omega_{ij} | n_i, \eta_{ij}).
\end{align*}

Thus to sample the posterior, we can employ Gibss and the following steps.
\begin{algorithm}
\begin{algorithmic}
\State Assume $\tilde y$ is the \textbf{proportional} response data.
\State Assume $\beta_J = 0$ for identification.
\State $\kappa \leftarrow (\tilde y - 1/2) n$
% \State $A \leftarrow \sum_{k \neq 1} \exp X \beta_{k}$.
\Loop
\For{$j = 1 \to J-1$}
% \State Calculate $c_{ij} = \log \sum_{k \neq j} \exp(x_i \beta_k)$, that is
% $c_{\cdot, j} \leftarrow \log A$.
\State $c_{\cdot, j} \leftarrow \log \sum_{k \neq j} \exp(X \beta_k)$
\State Let $\eta_{\cdot, j} = X \beta_j - c_{\cdot, j}$.
\State Draw $\omega_{ij} \sim PG(n_i, \eta_{ij})$, $i = 1, \ldots, N$.
\State Draw $\beta_j \sim N(m_{1j}, V_{1j})$ where
\[
\begin{cases}
P_{1j} = P_{\ell j} + P_{0 j} \\
V_{1j} = P_{1j}^{-1} \\
m_{1j} = V_{1j} ( b_{\ell j} + P_{0 j} m_{0 j} ).
\end{cases}
\]
% \State $A += \exp X \beta_j - \exp X \beta_{j+1 \mod J}$.
\EndFor
\EndLoop
\end{algorithmic}
\end{algorithm}

%% \subsubsection*{Binary Trees}

I need to look up what precisely polychotomous means.

Note: James mentions that a problem with multinomial data is that it is may be
perfectly seperable.  Intuitively, as you increase the number of categories,
then it is easier to find a few points that may be isolated from the rest.
Geometrically, I think this comes down to finding a way to split things by
hyperplanes or something like that.  Anyway, something is weird about the
likelihood in that case.  I think that there is a saddle or something in that
case so that $\beta$ might go off to $\infty$ in one of its coordinates.  I'm
not sure what it is.  I think this matters.  It is one thing for the likelihood
to be flat.  It is another for it to grow.

In any event, the multinomial logit is a problem for everyone.  In the PG case
the Gibbs sampling of $\beta$ is a huge problem.

\subsection*{EM}

From Gelman's red book... We want to maximize the posterior mode of $\phi$.
In addition we have another (auxillery) variable $\gamma$, which seems to ensure
that one has convenient densities.  From probability calculus,
\[
p(\phi | y) = \frac{p(\gamma, \phi | y)}{p(\gamma | \phi, y)}.
\]
Taking the log of both sides one gets
\[
\log p(\phi | y) = \log p(\gamma, \phi | y) - \log p(\gamma | \phi, y).
\]
Taking the expectation of both over $\gamma$ using the density $p(\gamma |
\phi^{old}, y)$ one has
\[
\log p(\phi | y) = \bbE_{old} [\log p(\gamma, \phi | y)] - \bbE_{old} [\log
p(\gamma | \phi, y)].
\]
The key is that the last term of the right side, \( \bbE_{old} [\log p(\gamma |
\phi, y)] \) is maximized over $\phi$ at $\phi = \phi^{old}$.  Thus if you
choose any value, $\phi^*$, that increases the first term on the right hand
side, \( \bbE_{old} [ \log p(\gamma, \phi | y) ], \), then you will have
\[
\log p(\phi^* | y) \geq \log p(\phi^{old} | y).
\]
One way to pick $\phi^*$: at each step
\[
\phi^* = \argmax{\phi} \bbE_{old} [\log p(\gamma, \phi | y)].
\]
Thus one can break this into two steps:
\begin{enumerate}
\item Expectation: $f(\phi | \phi^{old}) = \bbE_{old} [\log p(\gamma, \phi |
  y)]$ where we are integrating over $p(\gamma | \phi^{old}, y) d\gamma$.
\item Maximization: $\argmax{\phi} f(\phi | \phi^{old})$.
\end{enumerate}
When $p(\gamma, \phi | y)$ is an exponentially tilted version of some underlying
distribution for $\gamma$ things are particularly convenient.  For intance, in
our case we have ($z_i = x_i \beta$)
\begin{align*}
p(\beta, \omega | y) & = p(\beta | y) p(\omega | \beta, y) \\
& \propto e^{u \beta} \prod_{i=0}^N \exp( -\frac{z_i^2}{2} w_i ) p_{PG(n,0)}(w_i).
\end{align*}
Taking the expectation of the log of the joint density we have
\[
C + u \beta + \sum_{i=0}^N -\frac{z_i^2}{2} w_i
\]
where $C$ does not depend on $\beta$.  If we let $w_i^\bbE = \bbE[w_i | p(w_i |
z^{old}, y)] = \bbE[w_i | PG(n_i, z_i^{old})]$ then we want to maximize
\[
u \beta + \sum_{i=0}^N - \frac{z_i^2}{2} w_i^\bbE = u \beta - \frac{1}{2} \beta'
(X' \Omega^\bbE X) \beta,
\]
for which we can complete the square to get
\[
C - \frac{1}{2} (\beta - m)' (X' \Omega^\bbE X) (\beta - m), \textmd{ where }
(X' \Omega^\bbE X) m = u.
\]

Thus we want to calculate the expectation $\bbE[PG(w|n,z)]$.  We know from above
that
\[
PG(n, z) = \frac{1}{4} J^*(n, z/2).
\]
We will use this connection to help construct our EM algorithm.  A key component
of the algorithm is the first moment of $PG$, which we can derive from the first
moment of $J^*$.  Let $\mcL(X)(\lambda) = \bbE[\exp(-\lambda X)]$ denote the
``Laplace transform'' of $X$.  We will abuse notation and also use $\mcL$ for
the Laplace transoform of a function.  Then we know from above that
\[
\bbE[ e^{-\lambda J^*} ] = \cosh(\sqrt{2\lambda}).
\]
Hence the Laplace transform of $J^*(1,z)$ is
\begin{align*}
\bbE[ e^{-\lambda J^*(1,z)} ]
& = \cosh(z) \mcL(e^{-\frac{z^2}{2} w} p_{J^*}(w))(\lambda) \\
& = \cosh(z) \mcL(p_{J^*})(\lambda + \frac{z^2}{2}) \\
& = \cosh (z) / \cosh( \sqrt{2 \lambda + z^2} ).
\end{align*}
Immediately, we have that
\[
\bbE[e^{-\lambda J^*(n,z)}] = \frac{\cosh^n(z)}{\cosh^n(\sqrt{2\lambda+z^2})}.
\]
We also know that the moment generating function is related to the ``Laplace
transform'' by
\[
MGF(X)(\lambda) = \mcL(X)(-\lambda).
\]
Further, we know that
\[
MGF[PG(n,z)](\lambda)
= MGF[\frac{1}{4}J^*(n,z/2)](\lambda)
= MGF[J^*(n,z/2)](\frac{\lambda}{4}).
\]
Thus to get the first moment of $J^*(n,z)$ we calculate
\[
\del_\lambda MGF[J^*(n,z)](0) = - \del_\lambda \mcL[J^*(n,z)] (-0).
\]
Differentiating we have
\begin{align*}
\del_\lambda \mcL[J^*(n,z)] & = \\
 \cosh^n(z) 2 & \frac{1}{2} (2 \lambda + z^2)^{-1/2}
    \sinh(\sqrt{2 \lambda + z^2})
    (-n) \cosh^{-n-1} (\sqrt{2 \lambda + z^2}) \\
& = -n \cosh^n(z) (2 \lambda + z^2)^{-1/2} \tanh(\sqrt{2 \lambda + z^2}) \cosh^{-n}(\sqrt{2
  \lambda + z^2}) \\
& = -n \Big(\frac{\cosh(z)}{\cosh(\sqrt{2\lambda+z^2})}\Big)^{n} (2 \lambda + z^2)^{-1/2} \tanh(\sqrt{2 \lambda + z^2}).
\end{align*}
Hence
\[
\del_\lambda MGF[J^*(n,z)](0) = n |z|^{-1} \tanh(|z|).
\]
Since
\[
\del_\lambda MGF[PG(n,z)](\cdot) = \frac{1}{4} \del_\lambda MFG[J^*(n,z/2)](\cdot/4)
\]
we have that
\begin{align*}
\del_\lambda MGF[PG(n,z)](0) & = \frac{n}{4} \frac{2}{|z|} \tanh(|z|/2) \\
& = \frac{n}{2} \frac{\tanh(z/2)}{z}.
\end{align*}

\section{Expected number of iterations}

Theorem 5.1 (p. 155) of Devroye can be used to calculate the expected number of
iterations for the algorithm.  It relies upon Wald's equation, which can be
found on p. 50 (Thm 3.5), i.e. section 3.4.  The introduction to that section
explains things well.  Imagine a rejection-like algorithm.  Each time you make a
proposal and then check it you generate a vector of random variables, $W_i =
(X_i, Y_i)$.  It may be the case that $W_i$ has more than two random variables.
The $W_i$ are iid.  Furthermore, there is some stopping rule $N$, which I assume
is the same thing as a stopping time and there is some function which maps $W_i$
to some quantity of interest $\psi(W_i)$, such as the number of floating point
operations, the number of if statements evaluated, or the number of terms
evaluted in a partial sum.  We know that $N$ is a stopping time because, as
pointed out by Devroye, since $P(N=n)$ is determined by $W_1, \ldots, W_n$
residing in some set $B_n$.  Thus $N = n$ is $\mcF_n$-measurable.  Wald's
theorem tells us how to calculate the \emph{total} expected number of whatever
we are interested in, i.e. the expected value of
\[
\sum_{i=1}^N \psi(W_i).
\]
Given the above assumptions, Wald's theorem says that
\[
\bbE \Big[ \sum_{i=1}^N \psi(W_i) \Big] = E(N) E(\psi(W_1)).
\]
The proof is succinct.  The key is to recognize that (let $Z_i = \psi(W_i)$)
$Z_i$ and $\bbI \{N \geq i\}$ are independent for all $i$ as $Z_i$ and $\bbI \{N
< i\}$ are independent.

Devroye's theorem 5.1 is a bit short on detail.  Since I am used to thinking in
terms of the partial sums without the proposal factored out, I will phrase
things in an alternate manner.  The theorems says: Suppose $f$ can be
represented as an alternating sum
\[
f(x) = \sum_{i=0}^\infty (-1)^i a_i(x)
\]
where $a_i(x)$ is positive and decreasing in $i$ for all $x$ and that the
proposal you will use is $g(x) \propto a_0(x)$.  Let $N$ be the number of
``outer-loop'' iterations in the alternating sum algorithm, that is the total
number of times you make a proposal, and let $Z_i = \psi(X_i, Y_i)$ be the total
number of inner-loop interations, that is the the total number of times you
compare $Y_i$ to $S_n(X_i)$.  Then
\[
\bbE(N) = \int_0^\infty \sum_{i=0}^\infty a_i(x) dx.
\]

One can see this by an application of Wald's equation.  Consider the following:
When following the alternating sum algorithm one proposes $X \sim g$, generates
$Y \sim \mcU(0, cg(X))$, and then check $Y \leq f(X)$ by iterating through the
partial sums.  We can calculate the probability that $Y$ will be
accpeted/rejected on the $i$th iteration by checking that $Y \in (S_{i-2}, S_i]$
if $i$ is odd (with the convention that $S_{-1} = 0$) and that $Y
\in (S_i, S_{i-2}]$ if $i$ is even.  Given $X$, this probability is just the
length of that interval divided by $S_0 = a_0$ as seen in the figure below.
\begin{center}
\includegraphics[scale=0.75]{Images/expected_iterations.png}
\end{center}
Further more
\[
a_{i-1} - a_{i} = 
\begin{cases}
S_{i} - S_{i-2}, & \text{$i$ is even} \\
S_{i-2} - S_i, & \text{$i$ is odd}
\end{cases}
\]
Thus the probability on getting out of the inner loop on iteration $i$ given $X=x$
is
\[
P(Z = i | X = x) = \frac{a_{i-1}(x) - a_i(x)}{a_0(x)}
\]
where $Z = \psi(X, Y)$.  And hence the expected value of
getting out of the inner loop when $X=x$ is
\[
\frac{1}{a_0} \sum_{i=1}^\infty i (a_{i-1} - a_i) = \frac{1}{a_0(x)}
\sum_{i=0}^\infty a_i.
\]
Wald's theorem then tells us that
\begin{align*}
\bbE \Big[ \sum_{i=1}^N \psi(W_i) \Big] 
& = \bbE(N) \bbE(\psi(W_1)) \\
& = \bbE(N) \bbE( \bbE(\psi(W_1)|X_1) ).
\end{align*}
Since $\bbE(N) = c$ we have that this product is
\[
c \int_{0}^\infty \Big[ \frac{1}{a_0(x)} \sum_{i=0}^\infty a_i(x) \Big] g(x) dx
= \int_{0}^\infty \sum_{i=0}^\infty a_i(x) dx.
\]

We may interchange the sum and the integral by the MCT.  We can calculate each
integral analytically as the piecewise definition of $a_i$ is composed of the
exponential kernel and the inverse Gaussian kernel.  For the exponential kernel
we have (letting $r_n = a_n \bbI \{x > t\}$)
\[
r_n(x) = \cosh(z) \pi \Big(n + \frac{1}{2}\Big) 
\exp \Big\{ -\frac{x}{2} \Big( z^2 + (n+1/2)^2 \pi^2 \Big) \Big\}.
\]
And for the inverse Gaussian kernel
\[
\cosh(z)
\pi (n+\frac{1}{2}) \Big(\frac{2}{\pi x}\Big)^{3/2}
  \exp \Big\{ - \Big( \frac{z^2}{2} x + \frac{(2n+1)^2}{2x} \Big) \Big\},
\]
we may rearrange the term in the exponent
\begin{align*}
- \frac{z^2}{2}x - \frac{(2n + 1)^2}{2x}
& = -\frac{z^2}{2x} \Big( x^2 + \mu_n^2 \Big), \; \mu_n^2 = (2n+1)^2 / z^2
\\
& = \frac{-z^2}{2x} \Big( x^2 - 2 \mu_n x + \mu_n^2 + 2 \mu_n x \Big) \\
& = -z^2 \mu_n - \frac{\lambda_n}{2x \mu_n^2} (x - \mu_n)^2, \; \lambda_n =
(2n+1)^2.
\end{align*}
Thus we are left with an inverse Gaussian kernel (letting $\ell_n = a_n \bbI \{x
< t\}$)
\[
\ell_n(x) = 
\cosh(z) \pi (n + 1/2) \Big(\frac{2}{\pi}\Big)^{3/2} \exp(-z (2n+1))
\Big(\frac{1}{x}\Big)^{3/2}
\exp \Big\{ -\frac{\lambda_n}{2x\mu_n^2}(x-\mu_n)^2 \Big\}.
\]
which becomes
% \[
% \cosh(z) \pi \big(n + \frac{1}{2}) \exp(-z(2n+1))
% \Big(\frac{2}{\pi}\Big)^{3/2} \Big(\frac{\lambda}{2\pi}\Big)^{-1/2}
% \Big(\frac{\lambda}{2\pi}\Big)^{1/2} \Big(\frac{1}{x}\Big)^{3/2}
% \exp \Big\{ -\frac{\lambda_n}{2x\mu_n^2}(x-\mu_n)^2 \Big\}.
% \]
\[
2 \cosh(z) \exp(-z(2n+1))
\Big(\frac{\lambda_n}{2 \pi x^3}\Big)^{1/2} 
\exp \Big\{ -\frac{\lambda_n}{2x\mu_n^2}(x-\mu_n)^2 \Big\}.
\]
Closed form antiderivatives exists for both components.  We can use that to
calculate the average number of iterations given $x$.

I found that 
\[
\bbE \Big[\sum_{i=1}^N \psi(W_i) \Big]
\]
around $1.0014$ using empirical methods and $1.0016$ using analytical methods
when $z = 1.37$.  The discrepency may be due to an error on my part or some sort
of numerical accumulation.

\subsection{Distribution of inner-loop iterations}

The argument above shows how to compute $P(Z = j | X = x)$ thus one can also
calculate
\[
P(Z = j) = \int_{0}^\infty P(Z = j | X = x) g(x) dx,
\]
the expected probability that $Z = j$, which one can compute using the kernels
above.  In particular
\[
\int_{0}^\infty P(Z = j | X = x) g(x) dx = \int_0^\infty \frac{a_{j-1}(x) -
  a_j(x)}{c} dx
\]
where $c = \int_0^\infty a_0(x)$ is our normalizing constant.  Once we have the
integrals above we can calculate this via
\[
\frac{1}{c} \Big[ \int_0^\infty a_{j-1}(x) dx - \int_0^\infty  a_j(x) dx \Big].
\]
Doing this I found the first 4 probabilties to be (rouding to the 10th decimal
place) 0.9991977085, 0.0008022898, 0.0000000017, 0.0000000000.  This is kind
of absurd.

\subsection{Acceptance constant}

The kernel for the right portion of the proposal is
\[
\cosh(z) \frac{\pi}{2} \exp \Big( - (\pi^2 / 8 + z^2 / 2) x \Big).
\]
Thus after integrating $t$ to $\infty$ we have
\[
p(z,t) = \cosh(z) \frac{\pi}{2} \frac{1}{\pi^2 / 8 + z^2 / 2} \exp \Big( - (\pi^2 / 8 +
z^2 / 2) t \Big).
\]
The kernel for the left portion of the proposal is
\[
\cosh(z) \Big(\frac{2}{\pi x}\Big)^{3/2} \exp \Big(-\frac{z^2x}{2} -
\frac{1}{2x} \Big).
\]
An inverse Gaussian density looks like
\[
\Big( \frac{\lambda}{2 \pi x^3} \Big)^{1/2} \exp \Big( - \frac{\lambda
  (x-\mu)^2}{2 \mu^2 x} \Big).
\]
Completing the square we have
\[
- \frac{z^2}{2 x} (x^2 + 1/z^2) = - \frac{z^2}{2 x} (x^2 - 2x/z + 2x/z + 1/z^2)
= -z - \frac{z^2}{2 x}(x-1/z)^2.
\]
Thus we can rewrite the left portion as
\[
\cosh(z) e^{-z} (2/\pi)^{3/2} (2\pi)^{1/2} (2\pi)^{-1/2} x^{-3/2} \exp \Big( -
\frac{z^2}{2x} (x-1/z)^2 \Big).
\]
Simplifying we have
\[
(1+e^{-2z}) (2/\pi) p_{IG}(x | 1/z, 1).
\]
Thus integrating from $0$ to $t$ we have
\[
(1+e^{-2z}) (2/\pi) IG(t | 1/z, 1).
\]
Consider for the density $p_{IG}$.  Suppose we do a change of variables from $x
\sim IG(1/z,\lambda)$ to $y = x/t$.  Then $dx = t dy$ and
\begin{align*}
\Big( \frac{\lambda}{2 \pi x^3} \Big)^{1/2} \exp \Big( - \frac{\lambda
  (x-\mu)^2}{2 \mu^2 x} \Big) dx
& = \Big( \frac{\lambda}{2 \pi (yt)^3} \Big)^{1/2} \exp \Big( - \frac{\lambda
  z^2 (yt-1/z)^2}{2 (yt)} \Big) t dy \\
& = \Big( \frac{\lambda t}{2 \pi y^3} \Big)^{1/2} \exp \Big( - \frac{\lambda
  t (zt)^2 (yt-1/(zt))^2}{2 y} \Big)  dy 
\end{align*}
Thus we have $x \sim IG(1/z, \lambda)$ and $y \sim IG(1/(zt), t \lambda)$.  So,
the left integral is
\[
\int_0^t \ldots = \int_0^1 p_{IG}(y | 1/(zt), 1/t) dy.
\]
Thus we have that 
\[
q(z,t) = (1+e^{-2z}) (2/\pi) IG(1, 1/(zt), 1).
\]

We want to show that $p(z,t)$ is decreasing to $0$ and $q(z,t)$ in increasing to
$1$.  I can show the latter.  The former, I think, depends upon the value of
$t$.  Thus we can only show that it is decreasing for sufficiently large $z$.

For the former, the derivative in $z$ of
\[
\cosh(z) \exp(-(\pi^2/8 + z^2/2)t)
\]
is
\[
(\sinh (z) - \cosh(z) z t) \exp(-(\pi^2/8 + z^2/2)t).
\]
The derivative of
\[
\frac{1}{\frac{z^2}{2} + \frac{\pi^2}{8}}
\]
is
\[
\frac{-z}{\Big( \frac{z^2}{2} + \frac{\pi^2}{8} \Big)^2}
\]
Thus
\[
\del_z p(z,t) = p(z,t) \Big( \sinh (z) - \cosh(z) z t - 
\frac{z}{\frac{z^2}{2} + \frac{\pi^2}{8}} \Big).
\]
Thus for sufficiently large $z$ the quantity on the right is negative.  It
appears that it is always negative for $t^*$.

For the latter term, consdier the probability space $(\Omega, \mcF, \bbP)$, the
brownian motion $(W_t)$, and the brownian motion with drift $X_s^{\nu} = \nu s +
W_s$.  The stopping time
\[
T^{\nu} = \inf \{ s > 0 | X_s^{\nu} \geq 1 \}.
\]
is distributed as $IG(1/\nu, 1)$ and has
\[
\{ T^{\nu} < t \} = \{ \sup_{s \in [0,t)} X_s \geq 1 \}.
\]
Thus, if $\nu_1 < \nu_2$ and $T^{\nu_1}$ and $T^{\nu_2}$ are defined using the
same Brownian motion, then
\[
\bbP \{ T^{\nu_1} < t \} = \bbP \{ \sup_{s \in [0,t)} X_s^{\nu_1} \geq 1 \} <
\bbP \{ \sup_{s \in [0,t)} X_s^{\nu_2} \geq 1 \} = \bbP \{ T^{\nu_2} < t \}.
\]
This tells us that $q(z,t)$ converges to $1$ as $z$ goes to $\infty$.

To examine how $q(z,t)$ goes to zero it is instructive to examine its
derivative.  We have
\[
\del_z \Big[ (1+e^{-2z}) IG(t | 1/z, 1) ].
\]
Informlly bringing the derivative inside the integral (can we always do that
with exponential families?), we have
\[
\int_0^t \del_z \frac{1}{\sqrt{2 \pi} x^{3/2}} \exp \Big( - \frac{z^2}{2
  x}(x-1/z)^2 \Big).
\]
Taking the derivative of the exponent we have
\begin{align*}
-\frac{z}{x} (x-1/z)^2 + - \frac{z^2}{2x} \frac{1}{z^2} 2 (x-1/z)
& = -\frac{z}{x}(x^2 - \frac{2x}{z} + \frac{1}{z^2} \Big) - 1 + 1 / (zx) \\
& = 1 - zx.
\end{align*}
Thus 
\[
\del_z IG(t|1/z,1) = IG(t|1/z,1) - z \int_{0}^t x p_{IG}(x|1/z,1) dx.
\]
So the derivative is
\begin{align*}
-2 e^{-2z} IG(t|1/z,1) & + (1+e^{-2z}) \del_z IG(t|1/z,1) \\
& = (1 + e^{-2z} - 2 e^{-2z}) IG(t|1/z,1) - (1 + e^{-2z}) z \int_{0}^t x
p_{IG}(x|1/z,1) dx \\
& = (1 - e^{-2z}) IG(t|1/z,1) - (1 + e^{-2z}) z \int_{0}^t x
p_{IG}(x|1/z,1) dx.
\end{align*}

% (I think) We could also let $\psi$ map $(X,Y)$ to the probability mass
% function of $Z$ given $X$.  I need to think more about that.

\section{EM Warning}

When using the spambase dataset I run into problems.  In particular, I get okay
results when I look at the data up till \texttt{word.freq.hp}.  When I include
that variable then things go astray.

Trying to figure out what is going on, I found out that while
\texttt{word.freq.hp} is usually very small, $x_{i^*} = 20$ for $i^*=2552$.
Then $psi_{i^*} = -72$, which is large, causing $w_{i^*}$ to be small.  This in
turn affects $beta$, which in turn affects $\psi$.  Thus a feedback loop is
created which diverges.

The generalized linear model command uses an iteratively reweighted least
squares (IRLS) method, which according to what I read, can handle outliers.
Thus that method actually produces sensible coefficients, while the EM algorithm
returns non-sense (or breaks).

Another potential problem, when I start the EM algorithm at the MLE estimate
than I move to values that decrease the log-likelihood.  This is unsatisfying
because we want to maximize the log-likilihood, i.e the posterior mode (since I
am using a constant prior inthis case example).

UPDATE: If we write the likelihood as
\[
e^{u\beta} \prod_{i=0}^N \cosh^{-n_i}(z_i/2)
\]
then the log-likelihood is
\[
u \beta - \sum_{i=0}^N n_i \log \cosh (z_i/2).
\]
The first partial derivative of the log-likelihood is
\[
\del_j llh = u_j - \sum_{i=0}^N n_i \tanh(x_i \beta/2) \frac{x_{ij}}{2}
\]
and the second partial derivative is
\[
\del_k \del_j llh = - \sum_{i=0}^N \frac{n_i}{4} \cosh^{-2}(x_i \beta/2) x_{ij} x_{ik},
\]
which is certainly non-negative definite.

\section{Random Notes}

Factrs for FS\&F:
\begin{outline}

\;

\1 She says this is based on \cite{scott-2011} and \cite{mcfadden-1974}, which
is actually a representation for multinomial.  (I think my Scott reference may
have been a technical paper in 2004, but wasn't published until 2011.  Or maybe
I have the reference wrong.)

\1 The type I extreme value distribution has density
\[
\frac{1}{\sigma} t(y) \exp (-t(y)) \; \text{ where } \;
t(y) = \exp \Big(-\frac{y-\mu}{\sigma} \Big).
\]

\1 A ``standard'' EV(I) random variable $Y$ is a standard exponential random variable
$X$ by the transformation $x = \exp(-y) \iff y = - \ln(x)$.

Since $dx = e^{-y} dy$ we have
\[
\exp(-e^{-y}) e^{-y} dy = \exp(-x) dx.
\]

\1 This affects the auxiliary representation in that
\[
y_{iu} > y_{i0} \iff e^{y_{iu}} > e^{y_{i0}} \iff e^{-y_{i0}} > e^{-y_{iu}} \iff
x_{i0} > x_{iu}.
\]

FS\&F use this fact for Gibbs sampling.

\1 The ratio of independent exponential has PDF $(1+x)^{-2}$ and CDF $1 -
(1+x)^{-1}$ or $x / (1+x)$.  

You can see that this is the logistic distribution by letting $x = e^z$.  Thus,
a difference of independent standard EV(I) distributions is standard logistic:
\[
P(x_0 / x_1 < e^{\alpha}) = P(-y_{0} + y_{1} < \alpha).
\]

\1 Doing an Albert and Chib type thing, we have
\[
\begin{cases}
y_i = \one \{z_i > 0 \} \\
z_i = \psi_i + \text{Lo}(0,1).
\end{cases}
\]
Then we can let
\[
\text{Lo}(0,1) = \ep_{iu} - \ep_{i0} \; \text{ where } \ep_{ij} \sim EV(I)
\]
to get
\[
z_i = \psi_i + \ep_{iu} - \ep_{i0}.
\]

\1 The pdf for the logistic distribution is
\[
\frac{e^x}{(1+e^x)^2} = \cosh^{-2}(x/2).
\]
This comes up for HH.

In the PG representation we have
\[
\frac{e^{x/2}}{1+e^x} = 2 \cosh^{-1}(x/2).
\]
What distribution is this?  Also, consider Andrews and Mallows for a moment.
This can be written as a normal scale of mixtures, if it satisfies the totally
monotone (or whatever) property.  (Update)  This distribution is related to the
Cauchy distribution.  Let $y = e^{x/2}$ then $dy = 0.5 e^{x/2} dx$.  Hence
\[
\frac{1}{2} \frac{e^{x/2}}{1+e^x} dx = \frac{1}{1+y^2} dy.
\]
Hence
\[
\frac{1}{2 \pi} \frac{e^{x/2}}{1+e^x} dx = \frac{1}{\pi} \frac{1}{1+y^2} dy.
\]
Thus if $y$ is Cauchy then $x = 2 \ln (2y)$ comes from our distribution above.
But his is only for $y > 0$.  Really we want $y \in \bbR$.  Thus we should have
$x = 2 \ln (2 |y|)$.  But we now need to take into account the non-injective
nature of the transformation.  Let's be more careful.

Suppose $z = \ln y$ and $y$ is Cauchy.  Then
\[
\frac{1}{\pi} \frac{1}{1+y^2} dy = \frac{1}{\pi} \frac{y}{1+y^2} \frac{dy}{y},
\; y \in (0, \infty).
\]
This is equal to
\[
\frac{1}{\pi} \frac{e^z}{1+e^{2z}} dz.
\]
If we let $z = \ln |y|$ then we get the same thing.  But this isn't an injective
function.

Integrating over $y \in (0, \infty)$ we only have half the mass of the Cauchy
distribution.  Hence we need a normalizing constant of $1/2$ for $z$.  Thus the
density in $z$ must be
\[
\frac{2}{\pi} \frac{e^z}{1+e^{2z}} dz,
\]
which is
\[
\frac{1}{\pi} \cosh^{-1}(z) dz.
\]
Letting $x = 2z$ we have $dx = 2 dz$ and hence
\[
\frac{1}{\pi} \frac{e^{x/2}}{1+e^{x}} dx.
\]

However, Nick and James have shown, really, that for Gibbs sampling you do not
want to find a mixture of normals along those lines, but rather you want to use
some sort of integral transformation.  This lets you ignore the normaization
term in the normal.

Can we use mixture theory from Polson and Scott for infinite sum representation?

By Laplace transform
\[
\frac{1}{1+x^2} = \int_0^\infty e^{-xz} \sin(z) dz.
\]



\end{outline}

\section{A note on priors}

I think this must be known, though I don't recall encountering it.  The
transformation we have is essentially
\[
p = \frac{1}{1 + e^{\psi}} \; \text{ where } \psi = x \beta.
\]
According to my simulations, as the prior for $\beta$ becomes more vague the
distribution of $p$ approaches point mass mixture at 0 and 1.  In other words,
the implicit prior on $p$ is that it is with certainty either 0 or 1.  Actually,
this can't be right.  If it were, then your likelihood couldn't do anything.  I
wonder a uniform prior on $\psi$ produces a binomial prior for $p$.

When $p$ is uniform, then $\psi$ is logistic with cdf $\frac{1}{1+e^{-x}}$ and
density
\[
2^{-2} \cosh^{-2}(x/2) dx.
\]

\section{Accept/Reject}

How do you generate from $f$?  You could generate uniformly over $A = \{ (y,u) :
0 \leq u \leq f(y) \}$ and the marginalize out $u$.  How can you generate from
that though.  Well you could generate uniformly over a set $B$ that had $A
\subset B$ and then simply keep those samples that were in $A$.  So how can you
generate $B$?  Pick a distribution that is easy to sample from and for which
there is $c$ such that $f < cg$ everywhere, i.e. the best $c$ is
$\|f/g|_\infty$.  Generate $x \sim g$ and $u \sim \mcU(0, cg(x))$ and you have a
sample that is uniform over $\{(x,u) : 0 \leq u \leq cg(x) \}$, which is
precisly what we are looking for.  Save the sample as $y$ only if $u < f(x)$ and
you will have a uniform draw from $A$.

\includegraphics[scale=0.75]{Images/accept-reject.png}

Suppose for the moment one is doing accept/reject sampling with target density
$f(x)$ and proposal $g(x)$, that $k_f(x)$ and $k_g(x)$ are the kernels of the
respective densities, and that $\alpha \geq a(x) = k_f(x) / k_g(x)$ for all $x$.
Then one may sample $f$ by proposing $X \sim g$ until $U \sim \mcU(0, \alpha
k_g(X))$ has $U \leq k_f(X)$.  Then the probability of accepting a proposal is
\[
\bbE[ \bbE[\one \{U \leq k_f(X)\} | X] ] = \bbE \Big[ \frac{k_f(X)}{\alpha
  k_g(X)} \Big] = \int \frac{k_f(X) / c_f}{\alpha k_g(X)/c_g}
\frac{c_f}{g(x)}{c_g} dx = \frac{c_f}{\alpha c_g}
\]
where $c_f$ and $c_g$ are the normalizing constants of the kernels.

Alternatively, we have that probability of accepting a proposal is
\[
\int \frac{k_f(x)}{\alpha k_g(x)} g(x) dx = \int \frac{a(x)}{\alpha} g(x) dx.
\]

\section{Inverse Gaussian}

I spent way too long trying to figure this out, eventually had to look at a
paper (Shuster, 1968) and even then it wasn't clear.  Both Devroye and Shuster
say this is trivial.  I'm not sure how this is trial--this is the second time I
tried to figure this out (I had forgot the first until I looked at Shuster's
paper).

Recall the inverse Gaussian distribution
\[
f(x) = \Big( \frac{\lambda}{2 \pi x^3} \Big)^{1/2} 
\exp \Big[ - \frac{\lambda(x-\mu)^2}{2 \mu^2 x} \Big].
\]

Shuster breaks things into two pieces. First he says transform to $Y = \min(X,
\mu^2 / X)$.  This ensures that $Y \in (0, \mu)$ and makes the next step
easier.  The second transform is $Z = \frac{\lambda (x-\mu)^2}{\mu^2 x}$.

Refering to p. 53 of Casella and berger we know that
\[
f_Y(y) = \sum_{i=1}^k f_X(g_i^{-1}(y)) |\frac{d}{dy} g_i^{-1}(y)|.
\]
The two pieces we have are
\[
h_1(y) = g_1^{-1}(y) = y, \; 0 < x < \mu
\]
and
\[
h_2(y) = g_2^{-1}(y) = \frac{\mu^2}{y}, \; x > \mu.
\]
Calculating the Jacobians etc. we have the density for $Y$ is
\[
\Big( \frac{\lambda}{2 \pi} \Big)^{1/2} 
\exp \Big( -\frac{1}{2} \frac{\lambda (y - \mu)^2}{\mu^2 y} \Big)
\Big[ \frac{1}{y^{3/2}} + \frac{1}{\mu y^{1/2}} \Big] dy, \; y \in (0, \mu).
\]
as
\[
dx = dy \text{ for } h_1
\]
and
\[
dx = \mu^2 dy / y^2 \text{ for } h_2.
\]
Since
\[
z = \frac{\lambda}{\mu^2} \Big( \frac{(y-\mu)^2}{y} \Big)
\]
we have
\begin{align*}
\frac{dz}{dy} 
& = \frac{\lambda}{\mu^2} \Big(1 - \mu^2 / y^2) \\
& = \frac{\lambda}{\mu^2} \Big(\frac{y^2 - \mu^2}{y^2}\Big) \\
& = \frac{\lambda}{\mu^2} \Big(\frac{y - \mu}{y}\Big) \Big(\frac{y +
  \mu}{y}\Big) \\
& = \frac{\lambda^{1/2}}{\mu} \Big(\frac{y - \mu}{y^{1/2}}\Big)
\frac{\lambda^{1/2} (y + \mu)}{\mu y^{3/2}} \\
& = z^{1/2} \frac{\lambda^{1/2} (y + \mu)}{\mu y^{3/2}} .
\end{align*}
But the latter term is
\[
\lambda^{1/2} \Big[ \frac{1}{y^{3/2}} + \frac{1}{\mu y^{1/2}} \Big].
\]
Thus we get
\[
\frac{dz}{z^{1/2}} = \lambda^{1/2} \Big[ \frac{1}{y^{3/2}} + \frac{1}{\mu
  y^{1/2}} \Big] dy
\]
and the density for $z$ is thus
\[
\frac{1}{(2\pi)^{1/2}} z^{-1/2} \exp(-z/2) dz.
\]
This is only clear by following that route with $dz/dy$.  I tried writing out
$dy = stuff(z) dz$ and it did not work well.

Of course, you work backwards to sample $X$.

\section{Negative Binomial}

If $y \sim NB(p, d)$ is negative binomial (this isn't necessarily the usual
parameterization) then $y$ represents the number of successes before there are
$d$ failures where $p$ is the probability of success.  The pdf is
\[
{y+d-1 \choose y} (1-p)^d p^y.
\]
(Note ${y+d-1 \choose y} = {y+d-1 \choose d-1}$.)  The first two moments are
$\bbE[y] = \frac{pd}{1-p}$ and $\Var[y] = \frac{pd}{(1-p)^2}$.  Further we can
write the negative binomial distribution as a mixture
\[
\begin{cases}
y \sim \Pois(\lambda) \\
\lambda \sim \Ga(d, \text{scale}=\alpha), & \alpha = \frac{p}{1-p}.
\end{cases}
\]
In this case, $\bbE[y] = \mu = d \alpha$ and $\Var[y] = \mu / (1-p)$ so that
\[
\Var[y] = \mu + \mu \alpha = \mu + \mu^2 / d.
\]

Suppose now that we want $y_i \sim \NB$ where $\mu_i = x_i \beta$ and
\[
\bbE[y_i] = \mu_i \text{ and } \Var[y_i] = \mu_i + \mu_i^2 / d_i.
\]
Nelder and Mead have $\Var(y_i) = \mu_i \sigma_i^2$ in which case $\sigma_i^2 =
1 + \alpha_i$.  Then $\alpha_i = \mu_i / d_i$, $\psi_i = \log(\alpha_i)$, $p_i =
\frac{\alpha_i}{1+\alpha_i}$.  Hence we can go between the parameters $(\mu_i,
d_i)$ and $(p_i, d_i)$ by these transformations.  Here $\psi_i$ is the log odds.
I believe it is traditional to model $\mu_i$ as $\exp(x_i \beta)$ in which case
\[
x_i \beta = \log(\mu_i) = \log(\alpha_i) + \log(d_i) = \psi_i + \log(d_i).
\]
So the likelhood for a single term is
\begin{align*}
{y_i+d_i-1 \choose y_i} (1-p_i)^{d_i} p_i^{y_i} 
& = {y_i+d_i-1 \choose y_i} \Big( \frac{e^{\psi_i}}{1+e^{\psi_i}} \Big)^{y_i} \Big(
  \frac{1}{1+e^{\psi_i}} \Big)^{d_i} \\
& = {y_i+d_i-1 \choose y_i} \frac{e^{\psi_i y_i}}{(1+e^{\psi_i y_i})^{y_i + d_i}}
\end{align*}
Of course, we could also model $\psi_i = x_i \beta$.  When $d_i =d$ for all $i$
these are related by a linear change of variables (assuming we are including an
intercept).  The forecasts should be the same I think, regardless.  Right?  What
about identifiability?  If we model $\psi_i = x_i \beta$ then $\mu_i = x_i \beta
+ d_i$.  This isn't exactly the same.  However, there is nothing to stop us from
putting $\psi_i = x_i \beta - \log d_i$ into our PG step.  In which case we
would be doing traditional NB regression.  Nelder and Mead mention $\mu =
\mu(\beta)$ so that must be the usual (p. 132).

FSF use the formulation
\[
\begin{cases}
y_i \sim \Pois(\lambda_i) \\
\lambda_i \sim \Ga(\rho, \lambda_i^u \rho).
\end{cases}
\]
You can see a slight alteration of that method in the DynLogit notes.  Instead
of following FSF exactly, just take the decomposition we started with
\[
\begin{cases}
y_i \sim \Pois(\lambda_i) \\
\lambda_i \sim \Ga(d_i, \text{scale}=\alpha_i), & \alpha_i = \frac{p_i}{1-p_i}.
\end{cases}
\]
Then
\[
\log \lambda_i = \log (\alpha_i) + \log( \Ga(d_i, 1) ).
\]
If we have $\psi_i = \log (\alpha_i)$ and $x_i \beta - \log(d_i) = \psi_i$ then
this becomes
\[
\log \lambda_i = x_i \beta - \log(d_i) + \log (\Ga(d_i, 1)).
\]
And now we can use the mixture trick.  In that case our Gibbs sampler is
\begin{enumerate}
\item Sample $p(r, \lambda, d | y, \alpha)$:

  \begin{enumerate} 

  \item Sample $p(d | y, \alpha = e^{\psi})$, which we can do by NB lh and MH.
    Notice this is not conditional on $\lambda$.

  \item Then we can sample $p(\lambda | d, y, \alpha)$.

  This gives us a joint draw $p(\lambda, d | y, \alpha)$.

  \item Then sample $p(r | \lambda, d, y, \alpha)$

  This gives us a joint draw $p(r, \lambda, d | y, \alpha)$.

  \end{enumerate}

\item Sample $p(\beta | y, r, \lambda, d)$.
\end{enumerate}

I need to think more about $r$.  Suppose for the momment that $d_i = d$ for all
$i$.  I think this is reasonable.  First, we can compare the two methods by just
fixing $d$, I think that is reasonable since they each lose the same MH step.
Okay, instead, suppose $d$ is unknown.  Then $r$ depends on $d$ in the sense
that we know the distribution of $r_i | d$.  We have not written down the
distribution for $r_i$ marginally.  I think we want the dimension of $r$ to be
the same in each instance, e.g. we should use all six component mixtures.  Maybe
it is okay not to do that on the grounds that we can set some of the weights to
$0$ to ``pad out'' r so that it has the proper number of components.  But if $d$
is fixed over $i$ our current routine should work alright I think, though the
above principle needs to be kept in mind.  The marginal distribution of $r$ is
rather weird.

I believe FSF may use $-\log(\Ga(d,1))$ so I need to reverse the mean values for
their normal mixture.

\subsection{Gibbs Sampling - PG}

From above
\[
\ell(\beta | y, d) = \prod_{i=1}^n \frac{e^{\psi_i y_i}}{(1+e^{\psi_i y_i})^{y_i + d_i}}
\]
where $\psi_i = \log(\alpha_i)$.  Extracting a single term this becomes
\[
e^{\psi_i (y_i - (y_i + d_i)/2)} 2^{-(y_i + d_i) } \cosh^{-(y_i+d_i)}(\psi_i / 2)
\]
which becomes
\[
\propto e^{\psi_i \kappa_i} e^{-\omega_i \psi_i^2 / 2} p(\omega_i | y_i + d_i).
\]
so that
\[
\propto \exp \Big\{ (\zeta_i - m_i) \kappa_i  - \frac{(\zeta_i - m_i)^2}{2}
\omega_i \Big\} p(\omega_i | y_i + d_i)
\]
where $\zeta_i = x_i \beta$ and $m_i = \log(d_i)$.  We are going to assume $d_i =
d$ for all $i$.  Reconstituting the product and then summing within the exponent
we have (for the exponent)
\[
\kappa' (X \beta - m) - \frac{1}{2} (X \beta - m)' \Omega (X \beta - m).
\]
which becomes
\[
\kappa' X \beta - \frac{1}{2} \Big( \beta' X' \Omega X \beta - 2 \beta' X'
\Omega m + m' \Omega m \Big).
\]
Thus we have llh precision $X' \Omega X$ and
\[
a_\ell' = (m' \Omega X + \kappa' X) \implies a_\ell = (X' \kappa + X' \Omega m) .
\]
Thus if we have a normal prior with mean $b_0$ and precision $P_0$ then the
posterior precisions is
\[
P_N = X' \Omega X + P_0
\]
and mean
\[
m_N = P_N^{-1} (a_\ell + P_0 b_0).
\]
The posterior $\omega | \beta, y, d$ is 
\[
\prod_{i=1}^n \PG(\psi_i,  y_i + d).
\]

We still need to explain how to sample from $(d | \beta, y)$.

So in the Gibbs sampler we will do

\begin{enumerate}
\item Sample $p(d | \beta, y)$ and then $p(\omega | d, \beta, y)$ to get the
  joint draw $p(d, \omega | \beta, y)$.
\item Sample $p(\beta | \beta, d, y)$.
\end{enumerate}

\subsection{Gibbs Sampling - FSF}

Recall we have
\[
\begin{cases}
y_i \sim \Pois(\lambda_i) \\
\lambda_i \sim \alpha_i \Ga(d_i, \text{scale}=1).
\end{cases}
\]
We have $x_i \beta = \zeta_i = \log(\mu_i) = \log(\alpha_i) + \log(d_i) = \psi_i
+ \log(d_i)$.

\begin{outline}

\1 As explained above, we sample the joint $p(d, \lambda, r | y, \alpha)$:

\2 We will sample $p(d | y, \alpha)$ using MH. 

\2 To sample $(\lambda | d, y, \alpha)$, we have conditional independence and
\[
y_i \sim \Pois(\lambda_i) \text{ and } \lambda_i \sim \Ga(d, \alpha_i).
\]
Thus the posterior is
\[
\propto e^{-\lambda} \frac{\lambda^y}{y!} \; \frac{\lambda^{d-1}}{\alpha^k} e^{-\lambda / \alpha} 
\]
which is
\[
\propto \lambda^{y+d-1} e^{-\lambda (1 + 1/\alpha)} \sim \Ga(y + d, \text{scale}=\frac{\alpha}{1+\alpha}).
\]
Thus the posterior is $\Ga(y_i + d_i,
\text{scale}=\frac{\alpha_i}{1+\alpha_i})$.

\2 To sample $(r | \lambda, d, y, \alpha)$ we use $\lambda_i \sim \alpha_i
  \Ga(d_i, 1)$
\[
\log(\lambda_i) = \log(\alpha_i) + \log(\Ga(d, 1))
\]
\[
-\log(\lambda_i) = -\log(\alpha_i) + \ep_i, \; \ep_i \sim N(m_{r_i(d)}, v_{r_i(d)}).
\]
Thus let $res_i = \log(\alpha_i) - \log(\lambda_i)$.  Then use the usual
posterior.

Looking at Nelder and mean it appears that the parmeter that is used is $\phi_i
= 1 / \alpha_i$, which is related to the odds of failure, I think.  Their
mixture representation is for negative log gamma I think.

\1 To sample $p(\beta | y, d, r, \lambda)$ we have 
\[
\log \lambda_i = x_i \beta - \log d_i - \ep_i, \; \ep_i \sim N(m_{r_i(d)}, v_{r_i(d)}).
\]
so
\[
\log \lambda_i + \log d_i = x_i \beta - \ep_i, \; \ep_i \sim N(m_{r_i(d)}, v_{r_i(d)}).
\]

\end{outline}

\subsection{MH stop for $d$}

For the moment, assume that $d$ is a natural number.

The conditional density for y is
\[
\frac{(y_i+d-1)!}{(d-1)! y_i!} (1-p_i)^d p_i^{y_i}.
\]
The liklihood for $d$ is then
\[
\prod_{i} \frac{(y_i+d-1)!}{(d-1)!} (1-p_i)^d = \prod_i (1-p_i)^d \prod_{j=0}^{d-1} (y_i+ j).
\]
Thus the log-liklihood is
\[
f(d) = \sum_{i} \sum_{j=0}^{y_i-1} \log (d + j) + d \sum_i \log(1-p_i).
\]
Let $\alpha = \sum \log(1-p_i)$.  Take the first term.  We can write
\[
\sum_{i} \sum_{k=1}^{\max(y_i)} \sum_{j=0}^{k-1} \log(d+j) \one \{y_i = k\}
\]
which becomes
\[
\sum_{k=1}^{\max(y_i)} n_k \sum_{j=0}^{k-1} \log(d+j)
\]
where $n_k = \{ \# y_i = k \}$.  We have thus
\begin{align*}
\sum_{k=1}^{\max(y_i)} \sum_{j=0}^{\max(y_i)-1} n_k \log(d+j) \one\{j < k\}
& = 
\sum_{j=0}^{\max(y_i)-1} \log(d+j) \sum_{k=j+1}^{\max(y_i)} n_k \\
& = \sum_{j=0}^{\max(y_i)-1} \log(d+j) G_j
\end{align*}
where $G_j = \{ \# y_i > j \}$.  We can preprocess $G$!  Thus it shouldn't be
too much work to optimize $\texttt{sum}(\log(d+0:(M-1)) * G)$ over $d$.  That
will give us a mean and variance and then we can use a ? proposal.  (Actually,
it is harder than that.  We need to use a proposal that generates a Markov chain
that will traverse the space and have the proper stationary distribution.
Initially, I thought about just using a normal proposal, but we need to keep
things discrete and that walk the entire space.  See below for options.)  Notice
that $1-G_j = F(j) = \{\# y_i \leq j\}$.

We must deal with something.  If we were using the model $\psi_i = x_i \beta$,
then we would be good to go.  BUT! we are using $\phi_i = x_i \beta$ and $\mu_i
= e^{\phi_i}$.  We know that
\[
\mu_i = d \alpha_i, \; \text{ where } \alpha_i = \frac{p_i}{1-p_i};
\]
Thus
\[
\phi_i = \log(\mu_i) = \log(d) + \psi_i
\]
along with
\[
p_i = \frac{e^{\psi_i}}{1 + e^{\psi_i}}
\]
yield
\[
p_i = \frac{\mu_i/d}{1+\mu_i/d}.
\]
Thus
\[
1 - p_i = \frac{1}{1+\mu_i/d} = \frac{d}{d+\mu_i}.
\]
I think we need to use that when picking $d$.  This becomes more clear if we think
about our parameterization in terms of $\mu$.  The conditional density is then
\[
\frac{\Gamma(y+d)}{\Gamma(d) \Gamma(y) y} \Big( \frac{\mu}{\mu+d} \Big)^y
\Big(\frac{d}{\mu+d}\Big)^d.
\]
I suppose this shows why it is so noice to model the probability $p$.  Our work
above is still valid for the sum, but really we are interested in
\[
\sum_{j=0}^{\max(y_i)-1} G_j \log(d+j) + \sum_{i} y_i \log \frac{\mu_i}{\mu_i+d} +
d \sum_i \log \frac{d}{\mu_i+d}.
\]

If one includes an intercept, which is often the case, and would seem to be
necessary when modeling counts, then I think one can reconcile the two models.
In that case, let $\gamma = \beta - a e_1$.  Then
\[
x_i \gamma = x_i \beta - a.
\]
If we let $a = \log(d)$ then we have
\[
\psi_i = x_i \gamma = x_i \beta - \log(d).
\]
In any event we get a joint estimate of $(\beta, d)$ or $(\gamma, d)$.  It may
be easier to produce the joint density $(\gamma, d)$.  At the same time, we have
to MH $d$ either way, so why not just get $\beta$ since that is what we are after.

Another possibility.  If we use a continuous proposal, e.g. $\Ga(\mu, \sigma^2)$
where now the distribution is parameterized in terms of a mean and variance,
then given $X \sim \Ga(\mu, \sigma^2)$ we have
\[
P(\floor{X} = k) = \int_{k}^{k+1} \Ga(x; \mu, \sigma^2) dx.
\]
Thus calculating the ratio of proposals for Ind. MH is okay so long as
calculating the CDF of the proposal distribution isn't too computationally
intensive.

\section{Regression}

This is something I have been over before, but need to go over again.  Suppose
we have $x$ and $y$ and these are collections of elements in an inner product
space.  If we want to project $x$ onto $y$ that is equivalent to projecting each
element of $x$ onto $y$.  Imagine $\{x,y\} \sim N(0, V)$.

First, Grahm-Schmidt $y$ so that we have $e = Ly$ where $e$ is now a collection
of orthonormal elements.  $L$ will have full rank when $y$ is a linearly
independent collection.  $L$ will have less than full rank when $y$ is a
linearly dependent collection.  When $L$ has full rank we know that $I = L (y,
y') L'$ so that $L'L = (y,y')^{-1}$.  I write $(y,x')$ to mean
$[(y_i,x_j)]_{ij}$.

Now to project $x$ onto $Y$ we can project $x$ onto $e$ and then use the
tranform above.  Specifically, $\text{proj } x_i = \sum_{j} (x_i, e_j) e_j$.
Further we know that $(x_i, e_j) = \sum_{k} (x_i, L_{jk} y_k)$ so that $(x_i,
e_j)$ = $\sum_{k} L_{jk} (x,y')_{ik} = (x,y')_{i\cdot} L'$.  Thus we have
$\text{proj } x = (x,y') L' L y$.  When $(y,y')$ is full rank that is
$\text{proj } x = (x,y') (y,y')^{-1} y$, otherwise we are left using $L$.  I may
want to use $L$ instead of $(y,y')^{-1}$ because sometimes we have rank
deficiency.

But the question I ultimately wanted to ask was what happens when elements of
$x$ and $y$ are perfectly correlated?  That enters the picture through the
variance of $x|y$.  To this point we have taken $x$ and $y$ to have mean zero.
Thus the variance and the uncentered second moment are the same.  From the above
decomposition we have
\[
x = Ay + \ep \; \text{ where } y \perp \ep.
\]
The variance can then be written as
\[
\Var(x) = A\Var(y)A' + \Var(ep).
\]
The distribution of $(x|y)$ is centered at $\bbE[x|y] = Ay$ and has variance
$\Var(\ep)$.  The variance of $\Var(\ep) = \Var(x) - A \Var(y) A' = V_{11} -
V_{12} L'L V_{12}'$.

Really, what I set out to show though was that it is possible to have
projections with zero variance.  We can have
\[
x_i = a \cdot y + \ep \; \text{ where } y \perp \ep.
\]
We know that $A_{il} = \sum_{k,j} L_{jk} (x_i, y_k) L_{jl} = \sum_{k,j}
(x_i,y_k) L_{kj}' L_{jl}$, which is $[V_{12} L'L]_{il}$.  So
\[
a = V_{12,i\cdot} L'L.
\]
The variance is then
\[
\Var(x)_{ii} = V_{12,i \cdot} L'L V_{12,i\cdot}' + \Var(\ep).
\]
so that
\[
\Var(\ep) = \Var(x)_{ii} - V_{12,i \cdot} L'L V_{12,i\cdot}'
\]

Really, what I am trying to show is that if $z = \{x, y\}$ (where $x$ is
scalar-valued) has rank deficient variance then we can write $x$ as a linear
combination of $y$.  I believe we can do this from the perspective of linaer
transormations.  We can find a matrix $A$ so that $AVA$ has zeros in the first
row and the first column.  Then $A z$ will have a first element that has zero
variance and zero covariance.  An element with zero variance must be zero, since
we are assuming no means.  In that case, $0 = a_{00} x + \sum_{a_{0j}} y_j$ and
we can hence write $x$ as a linear comibination of $y$ and the variance of $x -
a \cdot y$ will be zero.

To construct: Consider the columns space of $V$.  Suppose it has rank $n-1$.
Then you can use the $n-1$ last columns to negate the first column.  Something
like
\[
0 = V a \; \text{ where } a = (1, b).
\]
Then also $a' V = 0'$.  So $V [a \; I] = [0 \; V_{:,2:n+1}]$.  And further $[a
\; I]' V [a \; I]$ is
\[
\begin{bmatrix}
0 & 0 \\
0 & V_{22}
\end{bmatrix}.
\]
So it must be the case that when you have a rank deficient variance matrix that
you can project some elements onto a subspace of the random vector and end up
with no residual variance.

In the case of dynamic regression with a coefficient that doesn't change in
time, call it $\iota$, that the submatrix of $(\iota_{t}, \iota_{t-1})$ is rank
deficient of the correct order so that $\iota_{t-1}$ projects perfectly onto
$\iota_t$ (i.e. there is no residual variance).  It may not be obvious from the
matrix algebra, but that is gaurenteed to be the case from the discussion above.

\section{Relationship between PG and KS}

Andrews and Mallows show that one may express the logistic distribution as a
normal mixture.  In particular,
\[
2^{-2} \cosh^{-2}(x/2) = \int_0^\infty \frac{1}{\sqrt{2\pi}y} e^{-x^2 / 2 y^2}
p_{2KS}(y) dy.
\]
Similarly, we know that we can write the logistic distribution using the PG
distribution as
\[
2^{-2} \cosh^{-2}(x/2) = 2^{-2} \int_0^\infty e^{-\omega x^2 / 2} p(\omega | 2, 0).
\]
In the second, representation we can get a normal mixture by doing
\[
\int_0^\infty \frac{\omega^{1/2}}{\sqrt{2\pi}} e^{-\omega x^2 / 2} \frac{\sqrt{2
    \pi}}{4} \omega^{-1/2} p(\omega | 2, 0) d \omega.
\]
Notice the first representation is a scale mixture while the second is a
precision mixture.  We can do a change of variables in the first representation
to generate a precision mixture.  Let $w = 1/y^2$.  Then $1/2\omega d \omega =
1/y dy$ so that 
\[
p_{2KS}(y) dy = y p_{2KS}(y) \frac{dy}{y} = \frac{1}{\sqrt{z}} \frac{1}{2z}
p_{2KS}(1/\sqrt{z}) dz.
\]
Then the scale mixture becomes a precision mixture
\[
\frac{\sqrt{z}}{\sqrt{2 \pi}} e^{-zx^2/2} \; \frac{1}{\sqrt{z}} \frac{1}{2z}
p_{2KS}(1/\sqrt{z}) dz.
\]
Since a normal mixture uniquely defines a distribution we must have that
\[
\frac{1}{\sqrt{z}} \frac{1}{2z} p_{2KS}(1/\sqrt{z}) dz \eqd \frac{\sqrt{2
    \pi}}{4} \omega^{-1/2} p(\omega | 2, 0) d \omega.
\]
Thus if $Y \sim 2 \KS$ then $Z \sim 1/(2KS)^2$ and $X \sim \propto \omega^{-1/2}
p(\omega | 2, 0)$ then $Z \sim X$.  In other words, there is a connection between
$\KS$ and a polynomial tilted $\PG(2,0)$.

Further, this shows that the distribution of $\omega^{-1/2}$ tilted $\PG(2,0)$
is, in fact,
\[
\frac{\sqrt{2 \pi}}{4} \omega^{-1/2} p(\omega | 2, 0);
\]
since we can integrate in both $\omega$ and $x$ we can reverse the order of
integration to see that this is a density.

Since $PG(2,0) = J^*(2,0) / 4$ if we let $\omega = \omega^*/4$ we have
\[
\omega^{-1/2} p(\omega | 2, 0) d\omega = (\omega^*/4)^{-1/2} p(\omega^*/4 | 2,
0) d(\omega^*/4) = 2 (\omega^{*})^{-1/2} p_{J^*}(\omega^* | 2, 0) d \omega^*.
\]
Further, we know that $4 KS^2 = \pi^2 J$.  Hence we have that
\[
\frac{1}{\pi^2 J} = \propto {\omega^*}^{-1/2} p_{J^*}(\omega^* | 2, 0).
\]

\subsection{Marginalizing $\psi$}

If we assume a non-informative prior on $\beta$ then we may be able to do
something like the following.  We have joint distribution
\[
e^{\psi \kappa} e^{-\omega \psi^2 / 2} pg(\omega | 1, 0).
\]
Marginalize that to get something like
\[
\sqrt{\omega} e^{\kappa^2}{2} pg(\omega | 1, 0).
\]
Thus a joint draw will be from polynomial tilted $\PG$.  The problem here is
that when we marginalize over $\beta$ we will actually get $|X'\Omega X|$
somewhere, which won't exactly be a polynomial tilted PG, it will be something
else completely.

\section{Data Generating Data Augmentation}

Above I showed how to do the data augmentation for data generation as opposed to
for posterior simulation in binary logistic regression.  We can generalize this
approach.  For logistic regression we have
\begin{align*}
{n_i \choose y_i}
\Big( \frac{e^{\psi_i}}{1 + e^{\psi_i}} \Big)^{y_i}
\Big( \frac{1}{1 + e^{\psi_i}} \Big)^{n_i-y_i}
& = 
{n_i \choose y_i}
\frac{e^{\psi_i y_i}}{(1+e^{\psi_i})^{n_i}} \\
& = 
{n_i \choose y_i}
\frac{e^{\psi_i (y_i-n_i/2)}}{(e^{-\psi_i/2}+e^{\psi_i/2})^{n_i}} \\
& = 
{n_i \choose y_i} e^{\psi_i (y_i-n_i/2)} 2^{-n_i} \cosh^{-n_i}(\psi_i/2).
\end{align*}
We can augment this with $PG(n_i, \psi_i)$ to get
\[
{n_i \choose y_i} e^{\psi_i \kappa_i} 2^{-n_i} \cosh^{-n_i}(\psi_i/2) p(\omega_i
|n_i, \psi_i)
= {n_i \choose y_i} e^{\psi_i \kappa_i} 2^{-n_i} e^{-\omega_i \psi_i^2/2}
p(\omega_i | n_i, 0).
\]
The point being that we can do data generation by
\[
\begin{cases}
y_i \sim \text{Binom}(p_i, n_i), & \psi_i = \log \frac{p_i}{1-p_i} \\
\omega_i \sim \PG(n_i, \psi_i).
\end{cases}
\]
Again, $y_i$ and $\omega_i$ are independent given $\psi_i$!

The negative binomial case is slightly different.  There we have
\begin{align*}
{d_i + y_i - 1 \choose y_i} \Big( \frac{e^{\psi_i}}{1 + e^{\psi_i}} \Big)^{y_i}
\Big( \frac{1}{1 + e^{\psi_i}} \Big)^{d_i}
& = 
{d_i + y_i - 1 \choose y_i} \frac{e^{\psi_i (y_i - d_i) /
    2}}{(e^{-\psi_i/2}+e^{\psi_i/2})^{d_i+y_i}} \\
& = {d_i + y_i - 1 \choose y_i} e^{\psi_i \kappa_i} 2^{-(d_i+y_i)}
\cosh^{-(d_i+y_i)}(\psi_i/2). 
\end{align*}
Augmenting with $\PG(n_i = d_i + y_i, \psi_i)$ we have
\[
 {d_i + y_i - 1 \choose y_i} e^{\psi_i \kappa_i} 2^{-(d_i+y_i)}
\cosh^{-(d_i+y_i)}(\psi_i/2) pg(\omega_i | n_i, \psi_i)
= {d_i + y_i - 1 \choose y_i} e^{\psi_i \kappa_i} 2^{-(d_i+y_i)} e^{-\omega_i
\psi_i^2 / 2} p(\omega_i | n_i, 0).
\]
Thus the data is generated by
\[
\begin{cases}
\omega_i \sim \PG(d_i+y_i, \psi_i), & \psi_i = \log(\mu_i) - \log(d_i) \\
y_i \sim \NB(\mu_i, d_i).
\end{cases}
\]
Notice that in this case $\omega_i$ is not independent of $y_i$.  Also, we
worked initially with $p_i$ but that is related to $mu_i$ by $p_i =
\frac{\mu_i}{\mu_i + d_i}$

\section{Sum of Logistics - Normal Mixture}

If we want to do dynamic binomial logistic regression we need to have $n$ latent
$z$'s at each time step.  We can work with the sufficient statistic
$\sum_{i=1}^n z_{it}$ to estimate the log-mean.  But that still creates a lot
more work when estimatign $z$ and when estimating $r$ since there are $nT$
variables instead of just $T$.  We could remedy that by find a normal mixture
for the sum of independent Logistics, which is
\[
\sum_{i=1}^n Lo_i = log(\Ga(n,1)) - \log(Ga(n,1)).
\]

\section{Fussl 2013}

Check out literature-notes.tex.  Let $y_i \sim \text{Binom}(n_i, p_i)$.  Let
$\psi_i = p_i / (1 - p_i) = x_i \beta$.  Then we introduce latent's $z_i^*$ and
$r_i$.

\subsection{$(z_i^*|y_i)$}

Note: MARGINAL.

Draw
\[
\begin{cases}
w_i = \text{rgamma}(n_i, 1) / (\lambda_i^1 + \lambda_i^0) \\
u_i = \text{rgamma}(n_i - y_i, 1) / \lambda_i^0 \\
v_i = \text{rgamma}(y_i, 1) / \lambda_i^1 \\
\end{cases}
\]
where $\lambda_i^0 = 1$ and $\lambda_i^1 = \exp(\psi)$.  Then
\[
(z_i^*|y_i) = - \log(w_i + v_i) + \log(w_i + u_i).
\]

\subsection{$(\beta | z_i^*, r_i)$}

We have that
\[
z_i^* = x_i \beta + \ep_i, \; \ep_i \sim N(0, v_{r_i}).
\]
Let $V_i = v_{r_i}$.  Let $V$ denote this vector or diagonal matrix.  Then we
have
\[
z = X \beta + \ep, \; \ep \sim N(0, V).
\]
So the exponent is 
\begin{align*}
(z - X \beta)' V^{-1} (z - X \beta) & = 
z' V^{-1} z - 2 \beta' X' V^{-1} z + \beta' X' V^{-1} X \beta.
\end{align*}
Thus
\[
P_1 = P_0 + X' V^{-1} X,
\]
and
\[
b_1 = X' V^{-1} z + P_0 m_0.
\]



\section{Mixed Model}

Let's check the model and the posterior.  The model we are working with is
\[
\begin{cases}
y_i \sim \text{binom}(n_i, p_i) \\
p_i = e^{\psi_i} / (1 + e^{\psi_i}) \\
\psi_i = \bbI_{i \in j} \alpha_j + x_i \beta \\
\beta \sim N(m_0, C_0) \\
\alpha_i \sim N(m, 1/\phi) \\
m \sim N(0, \kappa^2 / \phi).
\end{cases}
\]
So the posterior is proportional to
\begin{align*}
& \Big[ \prod_{i=1}^T \frac{{e^{y_i (x_i^{re} \alpha + x_i^{fe} \beta})}}
  {(1+e^{x_i^{re} \alpha + x_i^{fe} \beta})^{n_i}} \Big] \\
& \exp \Big( \frac{-1}{2} (\beta-m_0)' P_0 (\beta-m_0) \Big) \\
& |\phi|^{P_a/2} \exp \Big( \frac{-1}{2} (\alpha - \one m)' I \phi (\alpha -
\one m) \Big) \\
& (\phi^{1/2} / \kappa) \exp \Big( \frac{-1}{2} \phi \frac{m^2}{\kappa^2} \Big) \\
& \phi^{a/2-1} e^{- \phi b / 2}.
\end{align*}

Assume we have a binomial logistic observation.  The llh conditional on $\phi$
and $m$ is
\[
y X (\alpha, \beta) - \sum_i n_i \log (1 + e^{x_i (\alpha, \beta)}).
\]
The prior structure that I am using is
\[
\begin{cases}
\beta \sim N(m_0, C_0) \\
\alpha_i \sim N(m, 1/\phi) \\
m \sim N(0, \kappa^2 / \phi) \\
\phi \sim Ga(s/2, r/2).
\end{cases}
\]
Thus the log posterior is
\begin{align*}
& y X (\alpha, \beta) - \sum_i n_i \log (1 + e^{x_i (\alpha, \beta)}) \\
& - 0.5 (\beta - m_0)' P_0 (\beta - m_0) \\
& + \frac{p_a}{2} \log \phi - 0.5 \phi \sum_{i=1}^{p_a} (\alpha_i - m)^2 \\
& + 0.5 \log \phi - 0.5 \phi \frac{m^2}{\kappa^2} \\
& + (s/2 - 1) \log \phi - \frac{r}{2} \phi
\end{align*}
which becomes
\begin{align*}
& y X (\alpha, \beta) - \sum_i n_i \log (1 + e^{x_i (\alpha, \beta)}) \\
& - 0.5 (\beta - m_0)' P_0 (\beta - m_0) \\
& (0.5 (p_a + s + \one_{\kappa != 0}) - 1) \log \phi - 0.5 (\sum_{i=1}^{p_a}
(\alpha_i - m)^2 + r + m^2 / \kappa^2 \one_{\kappa != 0}) \phi.
\end{align*}
The indicator function shows what terms to delete when $\kappa == 0$, which
corresponds to assuming $m$ is known and equal to zero.

Alternatively, if we parameterize things in terms of $\delta = \alpha - m$ then
we have
\begin{align*}
& y (X,1) (\delta, \beta, m) - \sum_i n_i \log (1 + e^{(x_i,1) (\delta, \beta, m)}) \\
& - 0.5 (\beta - m_0)' P_0 (\beta - m_0) \\
& (0.5 (p_a + s + \one_{\kappa != 0} \one_{\kappa != \infty}) - 1) \log \phi - 0.5 (\sum_{i=1}^{p_a}
\delta_i^2 + r + m^2 / \kappa^2 \one_{\kappa != 0}) \phi.
\end{align*}

Note that we can change variables from $(\delta, m)$ to $\alpha, m$ by
\[
\begin{bmatrix}
I & \one \\
0 & 1
\end{bmatrix}
\begin{bmatrix}
\delta \\ m
\end{bmatrix}
= 
\begin{bmatrix}
\alpha \\ m
\end{bmatrix}.
\]
Further the determinant of the linear transform is $1$.  Thus we have a similar
transformation when including all of the variables.  Now, consider an arbitrary
linear transformation $y = A x$.  Suppose we want to find the gradient of $g(x)
= f(y) = f(A x)$.  Then we have
\[
(\del_x g(x))' h^{x} = (\del_y f(Ax))' A h^{x}.
\]
Similarly, for the Hessian we have
\[
{h^x_2}1' (\del_x \del_x g(x)) h^x_1 = {h^{x}_2}' A' (\del_y \del_y f(Ax)) A h^{x}_1.
\]
Thus we can go between the llh, the gradient of the llh, and the hessian under
each coordinates system by using our knowledge of these transformations.

However, if we are consider the posterior instead of the likelihood, then we
must be slightly more careful.  To go from the posterior in $y$ to the posterior
in $x$ we need to consider the Jacobian.  In particular, suppose we are working
with the posterior $f(y) dy$ and we want to consider the posterior in the $x$
coordiante system.  Then we would need 
\[
f(y) dy = f(Ax) |J(x)| dx
\]
where $J(x) = \frac{\del y}{\del x}(x)$.  For our simple linear transformation
$J(x) = A$ and hence $dy = |A| dx$.  Since $|A|=1$ we just have $dy = dx$.  Thus
the posterior $f(y)$ goes to the posterior $g(x)$ without any apparent
adjustment.  This means that when we calculate the log-likelihood, we may do so
in either coordinate system.  I suppose this is actually true whenever we are
making a linear transformation, however when the transformation is non-linear,
as is the case when changing the coordinate system for $\phi$ below, then we
must take $|J(x)|$ into account.

Let us consider the gradient and the Hessian of the log-likelihood
\begin{align*}
& y X (\alpha, \beta) - \sum_i n_i \log (1 + e^{x_i (\alpha, \beta)}) \\
& - 0.5 (\beta - m_0)' P_0 (\beta - m_0) \\
& (0.5 (p_a + s + \bbI_{\kappa != 0}) - 1) \log \phi - 0.5 (\sum_{i=1}^{p_a}
(\alpha_i - m)^2 + r + m^2 / \kappa^2 \bbI_{\kappa != 0}) \phi.
\end{align*}
\begin{description}

\item[$(\alpha, \beta)$:]
\begin{align*}
\frac{\del}{\del (\alpha, \beta)} (\cdot) 
& = y X h^{(\alpha, \beta)} 
  - \sum_i n_i (1 + e^{x_i(\alpha, \beta)})^{-1} e^{x_i (\alpha, \beta)} x_i h^{(\alpha, \beta)} \\
& - (\beta - m_0)' P_0 h^{\beta}
  - (\alpha -  m)' \phi h^{\alpha}.
\end{align*}

\item[$\phi$:]
\begin{align*}
\graddel{\phi} (\cdot) = (0.5 (p_a + s + \bbI_{\kappa != 0}) - 1) \phi^{-1}
h^{\phi}
+ - 0.5 (\sum_{i=1}^{p_a}
(\alpha_i - m)^2 + r + m^2 / \kappa^2 \bbI_{\kappa != 0}) h^{\phi}.
\end{align*}

\item[$m$:]
\begin{align*}
\graddel{m} (\cdot) = ( (\alpha - \one m)' \one - m / \kappa^2
\bbI_{\kappa != 0}) \phi h^{m}.
\end{align*}

\item[$(\alpha, \beta)$ and $(\alpha, \beta)$:]
\begin{align*}
\hessdel{(\alpha,\beta)}{(\alpha,\beta)} (\cdot)
& = \sum_i \Big[ n_i (1 + e^{x_i(\alpha, \beta)})^{-2} {e^{x_i (\alpha,
    \beta)}}^2 x_i h^{(\alpha,\beta)}_2 x_i h^{(\alpha,\beta)}_1 \\
& - n_i (1 + e^{x_i(\alpha, \beta)})^{-1} e^{x_i (\alpha, \beta)} x_i
  h^{(\alpha,\beta)}_2 x_i h^{(\alpha,\beta)}_1 \Big] \\
& + - h^{\beta}_2 P_0 h^{\beta}_1 - h^{\alpha}_2 \phi h^{\alpha}_1.
\end{align*}
Simplying, this is
\begin{align*}
\hessdel{(\alpha,\beta)}{(\alpha,\beta)} (\cdot)
& = \sum_i \Big[ n_i (p_i^2 - p_i) x_i
  h^{(\alpha,\beta)}_2 x_i h^{(\alpha,\beta)}_1 \Big] \\
& + - h^{\beta}_2 P_0 h^{\beta}_1 - h^{\alpha}_2 \phi h^{\alpha}_1.
\end{align*}

\item[$(\alpha,\beta)$ and $\phi$:]
\[
\hessdel{\phi}{(\alpha, \beta)} (\cdot) = -(\alpha-m)' h^{\alpha}_1 h^{\phi}_2
\]

\item[$(\alpha, \beta)$ and $m$:]
\[
\hessdel{m}{(\alpha, \beta)} (\cdot) = \phi h^{m}_2 1' h^{\alpha}_1.
\]

\item[$\phi$ and $\phi$:]
\[
\hessdel{\phi}{\phi} (\cdot) =  - (0.5 (p_a + s + \bbI_{\kappa != 0}) - 1) \phi^{-2}
h^{\phi}_2 h^{\phi}_1.
\]

\item[$\phi$ and $m$:]
\[
\hessdel{m}{\phi} (\cdot) = ( (\alpha - \one m)' \one - m / \kappa^2
\bbI_{\kappa != 0}) h^{m}_2 h^{\phi}_1.
\]

\item[$m$ and $m$:]
\[
\hessdel{m}{m} (\cdot) = -\one' \one \phi h^{m}_2 h^{m}_1 - 1/\kappa^2 \bbI_{\kappa !=
  0} \phi h^{m}_2 h^{m}_1.
\]

\end{description}

\subsection{Change of variables in prior}

If we let $\phi = e^\theta$ then we get a different prior.  This will change the
posterior mode.  (In general, a transformation does not preserve the mode of a
distribution.  You can see this by considering the gamma distribution.  It's
mode, if it exists, is $(shape - 1) / rate$.  But for a log-gamma distribution
the mode, if it exists, is $shape / rate$.)  In particular, this transformation
will change the form of the log-likelihood, but barely.  Instead of
\[
(0.5 (p_a + s + \bbI_{\kappa != 0}) - 1) \log \phi
\]
we have
\[
0.5 (p_a + s + \bbI_{\kappa != 0}) \theta.
\]

% If you have a bibliography.
% The file withe bibliography is name.bib.
\bibliography{bayeslogit}{}
\bibliographystyle{abbrvnat}

\end{document}

% \begin{algorithm}
% \begin{algorithmic}
% \REQUIRE A real number $z$
% \STATE $z \leftarrow |z|$,
%        $t \leftarrow 0.64$,
%        $K \leftarrow \pi^2 / 8 + z^2 / 2$
% \STATE $p \leftarrow \frac{\pi}{2 K} \exp(-K t)$
% \STATE $q \leftarrow 2 \exp(-|z|) \; \texttt{pigauss}(t, \mu = 1/z, 1.0)$
% % \STATE Set $p, q$ given $z$
% \LOOP
% \STATE Generate $U, V \sim \mathcal{U}(0,1)$
% \IF{$U < p / (p+q)$}
% \STATE $X \leftarrow t + E/K$ where $E \sim \mathcal{E}(1)$
% \ELSE
% %\STATE $X \sim \mathcal{IN}(\mu=1/z, 1.0) \bbI_{(0,t)}$
% \STATE $\mu \leftarrow 1/z$
% \IF {$\mu > t$}
% \REPEAT
% \STATE Generate $1/W \sim \chi^2_1 \bbI_{(t,\infty)}$
% \UNTIL {$\mathcal{U}(0,1) < \exp(-\frac{z^2}{2} W)$}
% \ELSE
% \REPEAT
% \STATE Generate $W \sim \mathcal{IN}(\mu, 1.0) \bbI_{(0,t)}$
% \UNTIL {$W < t$}
% \ENDIF
% % \STATE if $\mu > t$ generate $1/W \sim \chi^2_1 \bbI_{(t,\infty)}$ until $\mathcal{U}(0,1) < \exp(-\frac{z^2}{2} W)$
% % \STATE else generate $W \sim \mathcal{IN}(\mu, 1.0) \bbI_{(0,t)}$
% \ENDIF
% \STATE $S \leftarrow a_0(X)$, $Y \leftarrow VS$, $n \leftarrow 0$
% \LOOP
% \STATE $n \leftarrow n + 1$
% \IF{$n$ is odd}
% \STATE $S \leftarrow S - a_n(X)$; if $Y < S$, then return $X$
% \ELSE
% \STATE $S \leftarrow S + a_n(X)$; if $Y > S$, then break
% \ENDIF
% \ENDLOOP
% \ENDLOOP
% \end{algorithmic}
% \end{algorithm}