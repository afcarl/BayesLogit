\documentclass{article}

\input{commands}
\usepackage{outlines}
\usepackage{parskip}
\usepackage{natbib}
% \usepackage{bibentry}

\newcommand{\qt}{}

% Page Layout
\setlength{\oddsidemargin}{0.0in}
\setlength{\textwidth}{6.5in}
\setlength{\textheight}{8in}
%\parindent 0in
%\parskip 12pt

% Set Fancy Style
\pagestyle{fancy}

%\lhead{Left Header}
%\rhead{Right Header}

\begin{document}

% We need this to place the references in the notes.
% \nobibliography{bayeslogit}{}
%\bibliographystyle{abbrvnat}

% Change Font and Spacing
%\large % change the font size to 12pt
%\linespread{1.1} % change the line spacing

% Set the counter
% \setcounter{section}{0}

\tableofcontents

\section{Albert and Chib, 1993}

Bayesian Analysis of Binary and Polychotomous Response Data:

\begin{outline}
\1 Model: binary response $y_i \sim Bern(p_i)$ where $p_i = H(x_i \beta)$.
\2 We get to chose $H$.  $H =$ cdf of Normal: Probit.  $H=$ logistic: Logit.

\1 Normal approximation to likelihood not good for small $n$.

\1 Key idea: see a function of $x_i \beta$ and make it an integral over some
density.
\2 L: $\prod \Phi(x_i \beta)^{y_i} (1 - \Phi(x_i \beta))^{1-y_i}$.  So
\begin{align*}
\Phi(x_i \beta) 
& = \int_\bbR \bbI_{(-\infty, x_i \beta]}(z) p(z \mid 0, 1) dz 
= \int_\bbR \bbI_{[-x_i \beta, \infty)}(z) p(z \mid 0, 1) dz \\
& = \int_\bbR \bbI_{[0, \infty)}(z) p(z, \mid x_i \beta, 1) dz
\end{align*}
Similarly, $(1-\phi(x_i \beta)) = \int_\bbR \bbI_{(-\infty,x_i \beta]} p(z \mid
x_i \beta, 1)$.
Since
\[
p(\beta | y) = \int p(\beta, z \mid y) dz = \int p(\beta \mid y) p(z \mid \beta,y) dz.
\]
We have
\[
p(\beta, z | y) = \prod p(z_i \mid x_i \beta, y_i)
\]
where $p(z_i \mid x_i \beta, y_i)$ is standard normal with mean $x_i \beta$,
right truncated at zero if $y_i = 0$ and standard normal with mean $x_i \beta$,
left truncated if $y_i=1$.  Equivalently, and this is direct from Albert and
Chib
\[
\prod \Big[ \bbI(z_i > 0, y_i=1) + \bbI(z_i < 0, y_i=0) \Big] p(z_i \mid x_i
\beta, 1).
\]
In this scenario, the posterior conditionals are ``nice'' and you can then use
Gibbs sampling.

\2 You can generalize this further, by mixing the normal $N(z_i \mid x_i \beta,
\lambda)$ over $\lambda$.  They also delve into a hierarchical, e.g. Lindley and
Smith discussion, so that there is a hyperprior on the prior mean for $\beta$.

\1 Also discussion multinomial probit data.

\end{outline}

\section{Held and Holmes, 2006}

Bayesian Auxiliary Variable Models for Binary and Multinomial Regression

\begin{outline}

\1 Overal impression: they do several things: can just tick off the same stuff
with Polson and Scott's auxillery approach.

  \2 Auxillery variable representation

  \2 Joint sampling approach

  \2 Covariate uncertainty (model selection)

  \2 Polychotomous regression

\1 Extension to Alberta and Chib, 1993.

  \2 Joint Update

  \2 Auxillery approach is possible for logistic, using scale mixture of
  Normals.

    \3 Logit model is preferred because you can interpret stuff in terms of
    log-odds.

    \3 Logit link is analytic and has relatively heavy tails

  \2 Their logit work can be generalized to accomodate covariate set uncertainty
  and multinomail response data.

\1 For probit: they suggest a ``joint'' draw, which uses
\[
p(\beta, z |y) = p(z |y) p(\beta | z)
\]
instead of the Gibbs steps 1) $p(\beta | z, y)$ 2) $p(z | beta, y)$.  Joint is
in quotations because they use Gibbs sampling to draw $p(z|y)$.

  \2 Empirically, this is more efficient.

\1 Logit: They follow probit method, but you need another auxillery variable.
\begin{align*}
y_i & = \{ 1, z_i > 0; 0, z_i \leq 0 \} \\
z_i & = x_i \beta + \ep_i \\
\ep_i & \sim N(0, \lambda_i) \\
\lambda_I & = (2 \psi_i)^2 \\
\psi_i & \sim KS \\
\beta & \sim p(\beta)
\end{align*}
Of course, Devroye (1986), knows how to draw a Kolmogorov-Smirnov (KS)
distribution.  Once you have marginalized $\lambda$ you get that
$\ep_i$ is logistic, which Andrews and Mallows discuss.

  \2 Can sample kind of jointly.  A couple options.  They suggest $p(z, \lambda
  | \beta y)$ and $p(\beta | z, \lambda)$.

\1 What is going on in (16)?  Why not just take the $\beta_j$ of interest from
the the posterior?

\1 Check out discussion.  Mentions alternative method by Gamerman (1997) that
uses a ``weighted least squares'' Metropolis-Hastings proposal.

\end{outline}

\section{Andrews and Mallows, 1974}

Scale Mixtures of Normal Distributions

\begin{outline}

\1 Key connection:  If $Z \sim N(0,1) \perp V$ and $X = Z/V$ then $X$ has a
density given by
\begin{equation}
\label{am74:scale-mixture}
f_X(x) = (2 \pi)^{-1/2} \int_0^\infty v 
\exp \big\{ -\frac{1}{2} v^2 x^2 \big\} dG_V(v).
\end{equation}

\1 One can connect this to the Laplace transform of $H$, $h(y) = f_X(\sqrt{y})$
where
\[
H(t) = \int_0^t (u/\pi)^{1/2} dG_v[(2u)^{1/2}].
\]
Thus $f \rightarrow h \rightarrow H \rightarrow G$.

  \2 Similarly, if you know $f_X$ then you can immediately calculate the Laplace
  transform of $G$ (i.e. Moment generating function of $V$) by
  \[
  2 \int_0^\infty f_X(x) \exp \big\{ -\frac{1}{2} t^2 / x^2 \big\}
  =
  \int_0^\infty \exp (-vt) dG_V(v)
  \]
  using the identity
  \[
  \int_0^\infty \exp \Big( -\frac{1}{2} (a^2 u^2 b^2 u^{-2}) \Big)
  =
  (\pi/2a^2)^{1/2} \exp (-|ab|).
  \]

\1 \textbf{Main theorem}: Suppose $f_X$ is symmetric.  Then $g(y) =
f_X(y^{1/2})$ is totally monotone iff you can write $X$ as $Z/V$ where $Z \perp
V$ and $Z$ is standard normal.

  \2 This is all about BERSTEIN's theorem, which is interesting in its own
  right!

  \2 Totally monotone: 
  \[
  (-1)^n \frac{d^n f}{dy} \geq 0 \textmd{ for } y > 0.
  \]

  \2 BERSTEIN's theorem: If $f \in C^\infty\{ [0, \infty) \}$ is totally
  monotone then you can write $f$ as a mixture of exponentials:
  \[
  f(t) = \int_0^\infty e^{-tx} dg(x).
  \]

  \3 This theory characterizes the Laplace transforms of Borel meausres on the
  positive reals.

  \3 This is related to Levy processes and infinitely divisible measures (for
  subordinators, I suppose).

  \2 Their theorem includes a nice application of the MCT.

\1 Several examples: t, double exponential (Laplace), and logistic distribution.

\end{outline}

\section{Fruhwirth-Schnatter and Fruhwirth, 2007}

Auxiliary mixture sampling with applications to logistic models

\begin{outline}

\1 Second paragraph of intro mention ``seminal papers on Bayesian estimation of
logistic regression models''

  \2 First sampling? Zellner and Rossi, 1984

  \2 First MCMC: Zeger and Karim, 1991

  \2 Other MCMC: Gammerman, 1997; Chib et al., 1998, Lenk and DeSarbo, 2000;
  Hurn et al., 2003; Scott, 2004

  \2 Review: Dey et al., 2000.

\1 Mention again that most MCMC techniques require M-H step, which is evidently
not prefered.  Criticize Holmes and Held for using M-H step.

\1 Strategy: use latent utilities approach (McFadden, 1974) and (Scott, 2004) to
turn into a (conditionally) linear model; use normal mixture to approximate type
I extreme value distribution.

  \2 Latent utility thing is confusing economic-speak.  Just think of as latent
  varibles.

  \2 Equivalent to logit: 
  \begin{displaymath}
  \begin{cases}
    y_i = \bbI \big\{y_i^u > y_{i0}^u \big\} & \\
    y_i^u = x_i \beta + \ep_i & \\
    \ep_i, y_{i0}^u \sim \textmd{Type I Extreme Value Distributions} &
  \end{cases}
  \end{displaymath}

  Use normal mixture to approximate $\ep_i \approx N(m_{r_i},
  V_{r_i})$.  Then one can estimate $\beta$, in a Gibbs sampler using
  \[
  y_i^u = x_i \beta + m_{r_i} + N(0, V_{r_i}).
  \]

\1 Notes on Normal Mixture:

  \2 Pick values for normal mixture by minimizing Kullback-Leibler distance or
  Maximum absolute deviation.

  \2 Test how well normal mixture approximates true density by running a MCMC
  Met. Hast. simulation and measuring acceptance rate.  (SOMETHING TO REMEMBER.)

\1 Key link between type I extreme value distribution and exponentials.  In
particular,

  \2 $z \sim EV(I) \implies \exp(-z) \sim Ex(1) \sim Ex(rate=\exp(z))$.  Thus
  one can work with exponentials instead of EV(I) since we have a monotone
  transform which preserves the inequality.

  \2 Several key facts.

  \3 $\exp(-y_i^u) = \exp(-x_i \beta - \ep_i) = \frac{1}{\lambda_i} \exp(-\ep_i) \sim
  E(\lambda_i).$

  \3 $x_1 \sim E(\lambda_1)$ and $x_2 \sim E(\lambda_2)$ implies 
  $x_3 = \min(x_1, x_2) \sim E(\lambda_3 = \lambda_1 + \lambda_2)$
  And
  \[
  \begin{cases}
  p(x_1 | x_1 < x_2) \sim E(\lambda_3) \\
  p(x_1 | x_1 > x_2) \sim x_2 + E(\lambda_1).
  \end{cases}
  \]

  For my own sanity: The trick to use: think about $P(X_1 > x_1, etc.)$.  To
  show the min: $P(M:=\min\{X_1,X_2\} > x) = P(X_1 > x, X_2 > x)$ to get cdf.  To
  show conditionals, use similar trick:
  \begin{align*}
  P(X_1 > x, M > y) & = P(X_1 > x, X_1 > y, X_2 > y) \\
  & =
  \begin{cases}
    P(X_1 > x, X_2 > y) = e^{-\lambda x} e^{-\gamma y} &, x \geq y \\
    P(X_1 > y, X_2 > y) = e^{-(\lambda + \gamma)x} &, y > x.
  \end{cases}
  \end{align*}
  To get the density we apply $\frac{-\del}{\del x} \frac{-\del}{\del y}$ above
  to get the density
  \[
  p(x,y) =
  \begin{cases}
    \lambda e^{ -\lambda x} \gamma e^{ - \gamma y} &, x \geq y \\
    0 &, y > x.
  \end{cases}
  \]
  The kernel for the conditional is then
  \[
  \begin{cases}
    e^{-(\lambda + \gamma)x} &, x=y \\
    e^{-\lambda x} &, x > y.
  \end{cases}
  \]

\1 \textbf{Everything you need to implement code} is on pp. 3512 and 3513.

  \2 Key thing!  You must sample $(y^u, r \mid, e.e.)$ jointly!

\1 Multinomial logit

  \2 Direct extension of logistic regression for binary outcomes.

  To remind myself: One view of the multinomial case: $\mb{y}_i = (y_{i1},
  \ldots, y_{im})$ with each componenent representing the number of instances of
  case $j$ on the $i$th trial.

  I believe in FS\&F we must restrict our attention to ``one roll of the
  die,'' in which case we just get only one observation be trial.  In that case
  we can identify $y_i$ by the catogory $k$ we observe.

  We now have $m$ latents with
  \[
  y_{ji}^u = x_i \beta_j + \ep_{ji}, \ep_{ji} \sim EV(I), j = 1, \ldots m
  \]
  as well as $y_{0i}^u \sim EV(I)$.  Then
  \[
  y_i = k \iff y_{ki}^u = \max_{\ell} y_{\ell i}^u.
  \]
  This is due to McFadden, 1974.

  They say ``the category 0 for observation $y_i$ is indenpendent of any
  covariates for reasons of identifiability.''  Thinking along previous lines,
  one way to identify model is to chose $\lambda_0 = 0$, i.e. $\beta = 0$, which
  can be seen as $y_{0i}^u = \ep_{0i}$.

  In the case above we have two outcomes. Here it would seem that really we have
  $m+1$ outcomes.

  \2 Posterior estimation in essentially the same way:

    \3 Use normal mixture trick to represent some of the EV(I)'s.

    \[
    y_{ki}^u = x_i \beta_k + m_{r_{ki}} + N(0, s_{r_{ki}}^2).
    \]

    \3 Use exponential trick to sample $y_i^u | y_i, \beta)$.

    We know $\exp(-y_{ki}^u) = Ex(\lambda_{ki})$ with the convention that
    $\lambda_0 = 1$.  Further,
    \[
    \min_k \Big\{ \exp(-y_{ki}^u) \Big\} \sim Ex( \sum_{k=0}^m \lambda_{ki} ).
    \]
    
    When $y_i = k$ we know that 

    \4 $\exp(-y_{ki}^u) = \min \cdots$ with the above distribution and that
    
    \4 $\exp(-y_{\ell i}^u) = \exp(-y_{ki}^u) + Ex(\lambda_{\ell i})$.

    \3 Sample the normal mixture in the same way.

\1 Two extensions to more complex models:

  \2 State space modeling of binary data

  \2 Multinomial logit models with random effects

\1 Applications...

\end{outline}

\section{Fruhwirth-Schnatter and Wagner, 2006}

Title: Auxiliary mixture sampling for parameter-driven models of time series of counts
with applications to state space modelling

Citation: \cite{fruhwirth-schnatter-wagner-2006}

BibKey: fruhwirth-schnatter-wagner-2006

\begin{outline}

\1 Poisson regression for counts with mean changing in time

\[
y_t \mid \lambda_t \sim Po(\lambda_t), \; \log \lambda_t = x_t \beta.
\]

  \2 Usually, observations are assumed to be independent.

  \2 Account for dependence by (1) parameter-driven or (2) observation-driven
  models.  (See Cox, 1981).

  \2 They consider \emph{parameter} driven, in which case dependence is
  introduced by latent process.

\1 p.2, 2nd paragraph, references.

\1 There general model is
\[
\begin{cases}
y_t \mid \lambda_t \sim Po(\lambda_t) \\
\log \lambda_t = x_t^1 \alpha + x_t^2 \beta_t \\
x_t \sim \textmd{ some time series } \sim p(x\mid\theta).
\end{cases}
\]
The model for $x_t$ is what ``ties'' the means $\lambda_t$ together across
observations.

  \2 Two tricks: latent exponentials and normal mixtures.

  Consider the density for $y_t$.  Check Sato: You can construct the Poisson procoess
  $X_t$ using the random walk $W_n = T_n + W_{n-1}$ where $T_n \sim E(\lambda)$
  and $X_u = n$ iff $W_n \leq u < W_{n+1}$.  $X_u$ as marginal distribution
  $Po(\lambda u)$.  Thus if $y_t \sim Po(\lambda_t)$ then we must have that
  $P(y_t = n \mid \lambda_t) = P(W_{n} \leq 1 < W_{n+1} \mid \lambda_t)$.

  More specifically, it would be something like for each day $t$ we have
  $y_{t,u}$, a Poisson process in $u$ with parameter $\lambda_t$.  You can then
  say that $y_{t,1}$, which we denote $y_{t}$ for short, has
  \[
  P(y_t = n | \lambda_t) = P(W_{n}^{(t)} \leq 1 < W_{n+1}^{(t)} | \lambda_t)
  \]

  Analogous to $W_n^j$ above consider the latents $\tau_{tj} \sim E(\lambda_t),
  j = 1, \ldots, y_t + 1$.  Then
  \[
  \tau_{tj} \sim \frac{\xi_{tj}}{\lambda_t}, \; \xi_{tj} \sim Ex(1).
  \]
  You can do inference on $\log \lambda_t = \alpha x^1_t + \beta_t x^2_t$ by
  considering the log transformation,
  \[
  - \log \tau_{tj} \mid \alpha, \beta_t = x_t^1 \alpha + x_t^2 \beta_t + \ep_{tj}
  \]
  \textbf{ where }
  \[
  \ep_{tj} = - \log \xi_{tj} \sim \textmd{Extreme Value, model by exponential mixture.}
  \]
  Presumably $\beta_t$ is coming from a random walk, AR(1), etc.

  \2 Inference (my impression)

    \3 Sample $\alpha, \{\beta_t\}$ given $\{\tau_{ij}\}, \theta$ using DLM stuff.
    \3 Sample $\theta$ (from $\beta_t \sim AR(1|\theta)$) using time series stuff.
    \3 Sample normal mixture indicators $r_{tj} | \alpha, \beta_t, \tau_{tj}$.
    \3 Recall that $\log \lambda_t = \alpha x^1_t + \beta_t x_t^2$.
    \3 Sample $\tau_{tj} | \{y_{t}\}, \lambda_t$ using knowledge about
    conditional distribution of $\tau_{tj}$, i.e. $W_{y_t} = \sum_{i=1}^{y_t}
    \tau_{tj}$ has $W_{y_t} \leq 1 < W_{y_t+1}$.  Evidently, you can do this
    using order statistics... see Robert \& Casella, 1999, p. 47.


  \2 FS\& say to do the following:

  Select starting values for $\tau, \theta$ and $S = \{r_{tj}, j=1, \ldots, y_t,
  t=1, \ldots, T\}$.  (Aside: from my experience you want to be smart about
  seeding the indicators $\{r_{tj}\}$.  In particular, you DO NOT want to start
  the seed at a highly unlikely spot.  That can throw your entire simulation off
  into some other local mode.  [Or something like that.  I'm not exactly sure
  what happens.]  You could start by randomly picking $\{r_{tj}\}$ be prior
  distribution.)

    \3 Sample $\alpha, \{\beta_t\} | \ldots$.
    \3 Sample $\theta | \ldots$.
    \3 Sample $\{\tau_{tj}\} | \ldots$.
    \3 Sample $\{r_{tj}\} | \ldots$.
  
\1 Example: Application to road safety.

\end{outline}

\section{Geyer, 1992}

Title: Practical Markov Chain Monte Carlo

Citation: \cite{geyer-1992}

BibKey: geyer-1992

\begin{outline}

\1 Lots of good references in the beginning.

\1 If you run several chains and they disagree, then you haven't run your chains
for a long enough period.

  \2 But ``no comfort should be taken from the agreement of multiple runs.''

  \2 N. Polson, ``witch's hat'' distribution.

\1 p. 474 right side: ``Metropolis-rejected swapping'' (Geyer, 1991a) and other variants on
Metropolis methods.

\1 Theorem due to Kipnis and Varadhan, 1986: you can show that the moment of the
chain $\hat \mu_n$ and the moment you are interested in satisfy some central
limit result (p. 475).

In particular,
\[
\sqrt{n} (\hat \mu_n - \mu) \ra_{\mcD} N(0, \sigma^2).
\]

However, for this to be useful you need an estimate of $\sigma^2 =
\sum_{t=-\infty}^\infty \gamma_t$, the sum of the autocorrelations.

Just using the corresponding sample quantities does not work though.  However,
$\Gamma_m = \gamma_{2m} + \gamma_{2m+1}$ possess special properties (positive,
monotone decreasing, and convex), which can be used to select some ``good''
terms to generate an upper bound for $\sigma^2$.

\end{outline}

\section{West et al. 1985}

Title: Dynamic Generalized Linear Models and Bayesian Forecasting.

Citation: \cite{west-etal-1985}

BibKey: west-etal-1985

\begin{outline}

\1 Brief history of genesis: static to dynamic; formalize management by exception.

\1 Exponential family:
\[
p(Y_t \mid \eta_t, \phi) = \exp \Big[ \phi \Big\{Y_t \eta_t - a(\eta_t)\Big\}
\Big]
b(Y_t, \phi).
\]
$\eta_t$ is called the natrual parameter and $\phi$ is a scale parameter.  The
mean and variance are described by
\[
\bbE[Y_t \mid \eta_t, \phi] = \mu_t = a'(\eta_t)
\]
and
\[
\Var[Y_t \mid \eta_t, \phi] = a''(\eta_t) / \phi.
\]
If the prior for $\eta_t$ takes a specific form then closed form Bayesian
forecasting is possible.  That form is
\[
p(\eta_t | D_{t-1}) = c(a_t, \beta_t) 
\exp \Big[ a_t \eta_t - \beta_t a(\eta_t) \Big].
\]

\1 You can find a recapitulation of this work in \cite{harrison-west-1997}.

The idea there is to proceed using clever approximations.  With a generalized
linear model you want to model the conditional mean.
\[
\bbE[Y_t | \beta] = \mu_t(\beta) = a'(\eta_t).
\]
But the conditional distribution $p(Y_t | \beta)$ is not normal.  One does not
want to relate $\mu_t$ to $\beta$ in a linear way because that will probably
force $\mu_t$ to potentially take on some values that should have zero
probability.  Instead, one wants to use some transform taking $\mu_t$ to
$\lambda_t$ and to model $\lambda_t = x_t \beta$ linearly.  I believe $a'$ is
usually (always?) strictly increasing.  Let $\lambda_t = h(\mu_t)$ and
$\lambda_t = g(\eta_t)$ be bijections.  Then we can go between $\mu_t, \eta_t$,
and $\lambda_t$ with $a', h$, and $g$.  In binary logistic regression the
natural paramter is the log-odds so $g$ is the identity in that case.

\2 On p. 521 you see that
\[
\begin{cases}
p(Y_t | \eta_t) \; \text{ and } \; g(\eta_t) = \lambda_t = F_t' \theta_t \\
\theta_t = G_t \theta_{t-1} + \omega_t \; \text{ with } \; \omega_t \sim [0, W_t].
\end{cases}
\]
Notice that now the observatione equation is not just a normal perturbation and
that the distribution of $\omega_t$ is unspecified; instead, the first and
second moments are given.

If the time $t-1$ posterior moments are
\[
(\theta_{t-1} | D_{t-1}) \sim [m_{t-1}, C_{t-1}]
\]
then one evolves forward to
\[
(\theta_t | D_{t-1}) \sim [a_t, R_t]
\]
just like in the normal case.  But now we are just working with moments instead
of distributions.  The joint moments of $lambda_t$ and $\theta_t$ are then
\[
\begin{pmatrix} \lambda_t \\ \theta_t \end{pmatrix} | D_{t-1}
\sim
\Big[
\begin{pmatrix} f_t \\ a_t \end{pmatrix},
\begin{pmatrix}q_t & F_t' R_t \\ R_t F_t & R_t \end{pmatrix} \Big].
\]
One may take these moments and produce the regression of $\theta_t$ upon
$\lambda_t$ to calculate $\hat \bbE[ \theta_t | \lambda_t, D_{t-1}]$ and $\hat
\bbV[ \theta_t | \lambda_t, D_{t-1}]$.  I think you can derive these quantities
using linear predictors and quadratic loss.  (See section 4.9.)  In that case
I'm not sure if $\bbV(\theta_t | \lambda_t, D_{t-1})$ has the meaning that one
would initially give it.

They suggest using a conjugate distribution for $\eta_t$ so that one may go from
$(\eta_t | D_{t-1})$ to $(\eta_t | D_t)$ in a convenient way.  Think of this in
the following way.  The distribution of $(\eta_t | D_{t-1})$ is described by
$(r_t, s_t)$.  You update with new data to get new parameters $(r_t^*, s_t^*)$.
So the program to this point would be:

\begin{enumerate}
\item Get moments of $\lambda_t | D_{t-1}$.  Use those moments to calculate
  $r_t$ and $s_t$.

\item Use $r_t$ and $s_t$ to get $r_t^*$ and $s_t^*$.

\item Use $r_t^*$ and $s_t^*$ to get $f_t^*$ and $q_t^*$ where
\[
f_t^* = \bbE[\lambda_t | D_t] \; \text{ and } \; q_t^* = \Var[\lambda_t | D_t].
\]  
\end{enumerate}

Evidently, one can show that
\[
p(\theta_t | D_t) = \int p(\theta_t | \lambda_t, D_{t-1}) p(\lambda_t | D_t) d \lambda_t.
\]
This shows that $p(\theta_t | \lambda_t, D_t) = p(\theta_t | \lambda_t D_{t-1})$
I think.  That lets us then do
\[
\bbE[\theta_t | D_t] = \bbE[ \bbE[ \theta_t | \lambda_t, D_{t-1}] ]
\]
and
\[
\bbV[\theta_t | D_{t}] = 
\bbV[ \bbE[ \theta_t | \lambda_t, D_{t-1}] | D_{t} ] + 
\bbE[ \bbV[\theta_t | \lambda_t, D_{t-1}] | D_t ].
\]
We have linear estimates $\hat \bbE[\ldots] \simeq \bbE[\theta_t | \lambda_t,
D_{t-1}]$ (as an affine function of $\lambda_t$, and a function of $\lambda_t$,
$f_t$, and $q_t$) and $\hat \bbV[\ldots] \simeq \bbV[\theta_t | \lambda_t,
D_{t-1}]$ (as a function of $q_t$).  We can use those to get approximates
versions of $\bbE[\theta_t | D_{t}]$ and $\bbV[\theta_t | D_t]$.  Calculating
the approximation yields $(\theta_t | D_t) \simeq [m_t, C_t]$.

So the two places we must fiddle with things: (1) when we assume $\eta_t$ is
conjugate to $Y_t$, which I suppose isn't that crazy, except that I don't see
why it should be conjugate \emph{and} the transformed parameters should follow a
certain process; (2) we use the linear moments instead of the actual moments.

Thus, this procedure is really moving between \emph{approximations} of the first
and second moment.

\end{outline}

\section{Ravines Et al., 2006}

Title: An Efficient Sampling Scheme for Generalized Dynamic Models

Citation: \cite{ravines-etal-2006}

Bibkey: ravines-etal-2006

Building upon the work of \cite{west-etal-1985}, \cite{ravines-etal-2006} take
their sampling scheme and then add an MCMC step to get an exact draw from the
joint distribution of the states.

In particular, one assumes that the approximate moments produced by the above
procedure are exact and then backwards samples as in the normal case using
\[
p(\theta_T | D_T) \prod_{i=1}^T p(\theta_{t-1} | \theta_{t}, D_{t-1})
\]
where, I believe one uses the approximate moments $m_{t-1} = m_{t-1}(D_{t-1})$
and $C_{t-1} = C_{t-1}(D_{t-1})$ to evolve $\theta_{t-1}$ and then calculate the
regression coefficients needed for the backwards sampling.

Papers they mention:

\begin{outline}

\1 A review: Ferreira and Gamerman (2000).

\1 Prior to MCMC

\2 West et al. (1985)
\2 Kitagawa (1987)
\2 Farrmeir (1992)

\1 Migon (2005) has a survey?

\1 MCMC 
\2 Gamerman (1998) - studied MH methods.  Found single move samplers to be better.

\2 Geweke and Tanizaki (2001) - proposed some extensions.  Focused on good
proposals.

\end{outline}

Their contribution: proposing a multimove sampler.

\section{Fruhwirth-Schnatter and Fruhwirth, 2010}

Title: Data Augmentation and MCMC for Binary and Multinomial Logit Models. 

Citation: \cite{fruhwirth-schnatter-fruhwirth-2010}

BibKey: fruhwirth-schnatter-fruhwirth-2010

Good review of Bayesian methods for estimation in binary and multinomial logit.

\begin{outline}

\1 Review Holmes and Held and FS\&F along with several MH methods

\1 Scott's independnet MH methods is the best.

\2 I think the advantages of independent MH are going to go away when we are
working in the dynamic case.

\1 FS\&F present a normal mixture for the logistic distribution, a la the HH
auxiliar representation, which does much better than the FS\&F representation in
terms of extreme values.

\2 They call representation with logistic error DRUM and the representation with
EV error RUM (random utility).  You can derive the DRUM representation from RUM:
the logistic distribution can be written as the difference of two extreme value
distributions.

\1 From this point, it would seem that our challenge is to beat MH.  That may
not be possible.

\1 HH does horrible in their paper.  I am guessing that they did not implement the
HH stuff in C.

\end{outline}

\section{Fr\"{u}wirth-Schnatter and Wagner, 2008}

Title: Marginal liklihoods for non-Gaussian models using auxiliary mixture
sampling.

Citation: \cite{fruhwirth-schnatter-wagner-2008}

BibKey: fruhwirth-schnatter-wagner-2008

This will be of interst to James maybe.  It is about model selection for GLMs.
Use the complete-data likelihood estimator.  I need to read this more closely.

\section{Fr\"{u}wirth-Schnatter et al., 2009}

Title: Improved auxiliary mixture sampling for hierarchical models

Citation: \cite{fruhwirth-schnatter-etal-2009}

BibKey: fruhwirth-schnatter-etal-2009

The idea is that it is annoying to have to calculate all of those latent
variables in the Poisson case.  She remedies that by using a different latent
variable augmentation, though now she must approximate the $\log \text{Ga}(a,
1)$ distribution for various $a$.  She does that by generating the finite
mixture approximations for various $a$.  It appears that one could preprocess to
find the approximations.

She gives a presentation for negative binomial regression.  It appears that one
should want to select the degrees of freedom parameter.  She provides a way to
do that.  It seems to me that she is saying that she can sample from $\log
\text{Ga}(a, 1)$ for positive integers $a=1, \ldots, 20$, and then for
increasingly less frequent positive integers.  She has a way to interpolate
between those positive integers she hasn't precomputed a mixture representation
for directly.

\section{O'Brien and Dunson, 2004}

Title: Bayesian Multivariate Logistic Regression

Citation: \cite{obrien-dunson-2004}

BibKey: obrien-dunson-2004

Their moitivating example is binary logistic regression: neurtoxicology study in
rat pups.  Give rats pesticide at various levels and then observe normal
activity level (0) or elevated activity level (1).  This is odd since they are
designing a multivariate procedure.

Common frequentist approaches they mention: marginal logistic regression via
generalized estimating equations \qt{(GEE's; Zeger and Liang, 1986; Pretice, 1988,
Lipsitz, Laird, and Harrington, 1991; Carey and Zeger, 1993, among others) and
mixed effects logistic regression (Stiratelli, Laird, and Ware, 1984).}

They mention Albert and Chib-like method when working in univariate case.  Then
they say that this approach is problematic when trying to extend things to the
multivariate setting.

Their approach: Take i.d. $e_j$ with CDF $F$.  Then $\ell_j = \log \{ F(e_j) /
(1 - F(e_j)) \}$ is logistically distributed and hence $z_j = \mu_j + \ell_j$ is
logistically distributed with mean $\mu_j$.  They claim that because each $z_j$
is marginally logistic that the joint distribution of $z = \{z_j\}$ is
multivariate logistic.  I need to convince myself of that.  (I suppose you could
take Andrews and Mallows and then apply a transformation.  $X = N(0, \Lambda)$
where $\Lambda_{ii}$ is $4KS^2$.  Then define $AX$ to be multivariate logistic?
Maybe not.  Does this have logistic marginals?  It does if the sum of
independent logistic rvs is logistic.  But that does not seem to be the case
looking at the CF.  Thus it must be the multivariate logistic is defined in
terms of havign marginal logistic distributions?  Or it could be defined via the
multinomial logistic transformation--identifying that with a CDF.)

It isn't clear to me that that are doing multinomial logistic regression in the
sense that they are constructing a transformation for a generalized linear
model, but they do not show, at least I haven't found yet, that their
construction implies a likelihood invovlving
\[
\frac{e^{\psi_{ij}}}{\sum_{j=1}^J e^{\psi_{ij}}}.
\] 
Their model has the property that marginally one recovers a logistic
transformation.  I suppose when one choses a t-disribution with diagonal scale
parameter that things become independent and so you should get something like
that.  They mention multiple binary observations, which is different than
rolling a dice.

See equation (6) and discussion near by.  Theirs is slightly different.

In any event their construction of multivariate link function is interesting.
It is induced by the augmentation.  They suggest using $e \sim T_p$, a
multivarite t-distribution to construct $z$.  This induces a distribution on
$z$, and that can then be used for modeling.  Decomposing the t-distribution as
a scale-mixture of normals gets one back to a normal distribution, I am
guessing, for the regression coefficient.  (Aside, what if you use Andrews and
Mallows as suggested above with Holmes and Held to generate a link function?  Or
with our method for that matter.)

Check out Chib and Greenberg 1998 to see where they get some inspiration.

We can use there method to do a direct comparison with binary logit.  We need to
add a Metropolis step.  This is essentially in Albert and Chib 1993---use a $t$
to approximate the logistic.

If we think about a single draw from a multinomail distribution with $K$
categories, we have
\[
p_{ik} = \frac{e^{\psi_{ik}}}{1 + \sum_{k=1}^{K-1} e^{\psi_{ik}}} = F(\alpha_k)
- F(\alpha_{k-1})
\]
where the last term comes from O'Brien and Dunson.  The parameters $\alpha_k$
partition $\bbR$ to determine the probability of selecting something from group
$k$.  Thus it seems like it should be the $\alpha_k$ that are associated with
the $\beta_k$ in the multinomial case.

\section{Scott, 2011}

Title: Data augmentation, frequentist estimation, and the Bayesian analysis of
multinomial logit models

Citation: \cite{scott-2004}

BibKey: scott-2004 (I think everyone refers to a working paper from 2004)

Key papers he mentions
\begin{itemize}
\item Tanner and Wong (1987)
\item Albert and Chib (1993)
\item McCulloch and Rossi (1994)
\item Meng and van Dyk (1999)
\item Liu and Wu (1999)
\item van Dyk and Meng (2001)
\end{itemize}

He suggests three different algorithms.

\begin{description}

\item[DAFE] (Constrain $\beta_0=0$ for identifiabiltiy.)

This is Gibbs sampling--MH within Gibbs.

1. Sample $z \sim p(z | \beta^{(t)}, y)$

2. Use a MH step to sample $\beta \sim p(\beta | y,z)$.

In particular, Let your proposal be
\[
q_t(\beta) = p_\infty(\beta | \hat \beta^{(t)})
\]
where $\hat \beta^{(t)}$ is a least squares estimate of $\beta$ given $z^{(t)}$
and normal prior; and the target distribution is $p(\beta | y, z)$.

Scott uses the EV representation.  FSF say using the logistic representation is
better!!!

\item[DAFE-R]

This RW metropolis.

Just do MH.  Use the least squares estimate for the scale (I think) in the
proposal.

\item[DAFE-WP] Do DAFE but now do not constrain $\beta_0$.  Correct that in
  post-processing for identifiability.

\end{description}

\section{Hahn et al., 2010}

Title: A sparse Factor-Analytic Probit Model for Congressional Voting Patterns

Citation: \cite{hahn-etal-2010}

Bibkey: hahn-etal-2010

\section{Chib and Jeliazkov, 2006}

Title: Inference in Semiparametric Dynamic Models for Binary Longitudinal Data

Citation: \cite{chib-jeliazkov-2006}

BibKey: chib-jeliazkov-2006

\section{Chib and Greenberg, 1998}

Title: Analysis of multivariate probit models

Citation: \cite{chib-greenberg-1998}

BibKey: chib-greenberg-1998

They model \emph{correlated binary responses}.  This is not a multinomial
model.  In particular, given $Y_i = (Y_{i1}, \ldots, Y_{iJ})$,
\[
P(Y_i = y_i | \beta, \Sigma) = p(y_i | \beta, \Sigma) = 
\int_{A_{i1}} \ldots \int_{A_{iJ}} \phi_J(t | 0, \Sigma) dt
\]
where $\phi_J$ is the normal density, $\Sigma$ is a correlation matrix (for
identifiability), and
\[
A_{ij} = 
\begin{cases}
(-\infty, x_{ij} \beta_j), & y_{ij} = 1 \\
[x_{ij} \beta_j, \infty), & y_{ij} = 0.
\end{cases}
\]
Here $x_{ij}$ is a row of covariates.  I am used to working with the same
covariates for each $\beta_j$, in which case we would drop the $j$ from
$x_{ij}$.  Thus there setup is more general than I am used to.

Equivalently, we can let $y_{ij} = \one (z_{ij} > 0)$, and $B_{ij} = (0,
\infty)$ and then let
\[
P(Y_i = y_i | \beta, \Sigma) = \int_{B_{i1}} \ldots \int_{B_{iJ}} \phi_j(z_i |
x_i \beta, \Sigma).
\]

They sample $z_i$, which is a truncated multivariate normal, by using Geweke.
They sample $\Sigma$ using a Metropolis-Hastings step.

\section{Miscellaneous}

\begin{outline}

\1 Talking with Larry Carin: he said that logit is preferable to probit because
you can sample $p$ (the probability) directly whereas within probit you can only
easily sample 1,0.  I don't know what he meant.

\1 Think about Poisson Binomial

\end{outline}

\section{Things to look up:}

\begin{outline}

\1 Peskun ordering (p. 153 Holmes and Held, 2006)

  \2 After looking that up revisit that the aforementioned section in Holmes and
  Held.

\1 Rao-Blackwellization.  Look up in Gelfand and Smith, 1990.

\2 Rao-Blackwellization refers to conditioning on a sufficient statistic to
reduce variance of estimate.  It relies upon identity
\[
\Var(X) = \Var(\bbE[X|Y]) + \bbE[\Var(X|Y)].
\]
You can find a discussion of this in Berger and Casella.  First, $\bbE[X|Y]$ is
an unbiased estimator as well and it has small variance than $X$.  Second, and
more subtle, is that we need $Y$ to be a sufficient statistic.  If it is not a
sufficient statistic, then the esitmator $\bbE[X|Y]$ may depend upon the
underlying parameter $\theta$, which isn't helpful since we don't know $\theta$.
In other words, when $Y$ is sufficient then $p(x | y, \theta) = p(x | y)$, which
may not otherwise be the case.


\1 What is a quantile-quantile (QQ?) plot good for?

\1 Mixed effects models.

\1 Check out the book Generalized Linear Models by McCullagh and Nelder, 1999.

\end{outline}

% If you have a bibliography.
% The file withe bibliography is name.bib.
\bibliography{bayeslogit}{}
\bibliographystyle{abbrvnat}

\end{document}
