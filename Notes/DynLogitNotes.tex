\documentclass{article}

\input{commands}
\usepackage{outlines}
\usepackage{natbib}
\usepackage{parskip}

\newcommand{\Polya}{P\'{o}lya}
\newcommand{\Pois}{\text{Pois}}
\newcommand{\Ga}{\text{Ga}}
\newcommand{\NB}{\text{NB}}

% Page Layout
\setlength{\oddsidemargin}{0.0in}
\setlength{\textwidth}{6.5in}
\setlength{\textheight}{8in}
%\parindent 0in
%\parskip 12pt

% Set Fancy Style
%\pagestyle{fancy}

%\lhead{Left Header}
%\rhead{Right Header}

\begin{document}

% Change Font and Spacing
\large % change the font size to 12pt
\linespread{1.1} % change the line spacing

% Set the counter
\setcounter{section}{0}

\section{Dynamic / GP Logit}

Imagine a scenario where the coefficients in a regression may be slowly changing
in time.  The idea, I think, is that somehow the population is slowly evolving
via changes in the parameters of the regression.  For instance, we could regard
$\psi_{it}$ as the log-odds of success for observation $i$ at time $t$.  This
would be something like
\[
P(y_{it}=1) = p_{it} \; \textmd{ where } \psi_{it} = \log \frac{p_{it}}{1-p_{it}}.
\]
I am imagining a scenario where at each time $t$ we are doing a logistic
regression and these logistic regressions are tied together by some sort of
correlation / covariation structure.  This could be modeled by something like
\[
\begin{cases}
\psi_{it} = x_{it} \beta_t \\
\beta_t = \beta_{t-1} + \omega_t, & \omega_t \sim N(0, W).
\end{cases}
\]
Above is the basic idea used by West et al. (1985) (or check out West and
Harrison 1997) for modeling dynamic generalized linear models.

Imagine fixing $t$ and looking at a single term in the likelihood using the
Polya-Gamma trick.  You would have something like (where $y_i$ is the number of
successes for covariate $i$--though for the PG code we have $y_i$ is the
proportion of successes!)
\[
\exp \Big[ (y_{i} - n_{i}/2) x_{i} \beta - \frac{1}{2} \omega_{i} \psi_i^2
\Big] 
p(\omega_i | n_i, 0). 
\]
Collecting all the observations this becomes
\[
\exp \Big[ \alpha' X \beta - \frac{1}{2} \beta' X' \Omega X \beta \Big]
p(\omega | n, 0) \;
\textmd{ where } \;
\alpha_i = (y_i - n_i/2) \; \textmd{ and } \; X = [x_i].
\]
Assuming that $\omega$ is fixed.  We want to find a quadratic form
\[
(z - X \beta)' \Omega (z - X \beta).
\]
Expanding the quadratic form we get
\[
z' \Omega z - 2 z' \Omega X \beta + (X \beta)' \Omega (X \beta).
\]
Since $\omega$ is fixed we can just multiply the likelihood by $\exp
\frac{-1}{2} z' \Omega z$ where $\Omega z = \alpha$ to generate this expression.
Thus, for fixed $\omega$ the likelihood of $\beta$ can be interpreted as coming from
\[
z = X \beta + \ep, \; \ep \sim N(0, \Omega^{-1})
\]
where
\[
\Omega z = \alpha.
\]
(Waving my hands: We have $\Omega z = \alpha = y - n/2$.  Thus we should
have $y = \Omega z + n/2$.  Use for generating synthetic data?)  Thus,
reincorporating the time index we have
\[
\begin{cases}
z_t = X_t \beta_t + \ep_t, & \ep_t \sim N(0, \Omega_t^{-1}) \\
\beta_t = \beta_{t-1} + \omega_t, & \omega_t \sim N(0, W).
\end{cases}
\]
It is easy to calculate $z_t$ since $\Omega_t$ is diagonal.  The one place that
we need to be careful is that there may be a variable number of observations at
each point in time.  Thus the number of rows in $z_t$ and $X_t$ is $m_t$, which
may vary.  Also, it may be the case that there are no observations, in which
case we have the equivalent of missing data.  I believe the dynamic linear model
shouldn't have a problem with that.  The sampling procedure is then
\begin{enumerate}
\item Set $\beta_t = 0$ for all $t$.  This is a reasonable seed since it
  represents even odds.
\item Calculate $\psi_t = X_t \beta_t$.
\item Sample $\omega_{it} \sim PG(n_{it}, \psi_{it})$.
\item Sample $\beta_t$ using the dynamic linear model above.
\item Repeat, starting at (2).
\end{enumerate}

An equivalent, off-line, method would be to assume that $\beta(t)$ follows a
Gaussian process or some other prior that ties $\beta_t$ together.  Then we
would have something like
\[
\begin{cases}
z_{i} = x_{i} \beta(t_i) + \ep_{i}, & \ep_{i} \sim N(0, \omega_{i}^{-1}) \\
\beta \sim GP(0, K) \\
\end{cases}
\]
where $K(t_1, t_2)$ is the covariance function of the Gaussian Process.  We have
re-indexed so that each observation has its own index and time is a covariate.
In fact, this shows how you can take fancy linear model and apply it to
categorical data using the Polya-Gamma trick.

Maybe we should try to find categorical data that include spatial component.
One could then use try Gaussian Process regression to estimate with coefficients
changing smoothly in space.  In that case we would have,
\[
\begin{cases}
z_i = f(x_i) + \ep_i, \; \ep \sim N(0, \omega_i^{-1}) \\
f \sim GP(0, K).
\end{cases}
\]
This could be a computational burden because we will have to invert a relatively
large matrix.

Check out Chapter 3 of Williams and Rasmussen's book on Gaussin Processes.  You
must use approximation in that case because you do not have a conjugate
situation.  This is not the case with the Polya-Gamma method.  Though, again,
you may be burdened large amounts of data making it time consuming to invert
matrices.  They have an example using a database of hand written characters
provided by the USPS.  I wonder... instead of using a Gaussian process can you
use a Dirichlet process... You should have clusters, i.e. characters, or several
variants of characters and you want to assign each letter to a cluster.  Maybe
you can do a cluster of clusters.  I think Jordan had something similar to this.
This may relate to his beta process.  Basically, I am thinking of a Gaussian
process of normal mixtures, where the location of the normal mixtures is random.
This seems like it might be computationally intensive.

We can follow a program similar to above following Fr\"{u}hwirth-Schnatter.
This is kind of a hybrid between the Poisson and Logistic regression.  The
log-odds are still expressed as $\psi_{it} = x_i \beta_t$.  If, again, we think
of fixing $t$, then we sample using the latent utiltiy representation
\[
y_{i}^u = x_i \beta + \ep_{i}, \; \ep_i \sim N(m_{r_i}, s_{r_i}^2).
\]
Given $\beta$ we sample each $y_i^u$, $y_{i}^0$, and $r_i$ independently.
Further, we do not assume any of these have temporal dependence when
conditioning on $\beta_t$.  Thus the sampling procedure, I think, should be a
simple extentions where now we estimate $\beta_t$ using
\[
\begin{cases}
y_t = X_t \beta_t + \ep_t, & \ep_t \sim N(m_{r_t}, \diag(s_{r_t}^2)) \\
\beta_t = \beta_{t-1} + \omega_t, & \omega_t \sim N(0, W).
\end{cases}
\]
Again, we must pay attention to the dimension of $y_t$, $X_t$, and $r_t$, which
changes in time.  (The number of columns in $X_t$ does not change in time.)  As
you can see, this is (essentially) identical to the model above.

In either case, we can write either model as
\[
\begin{cases}
y_t = X_t \beta_t + \ep_t, & \ep_t \sim N(b_t, V_t) \\
\beta_t = (I-\Phi)\mu + \Phi \beta_{t-1} + \omega_t, & \omega_t \sim N(0, W).
\end{cases}
\]
For the moment let's restrict $\Phi$ to be diagonal.  I am writing this slightly
differently than Fr\"{u}wirth Schnatter because she includes the mean term in
$\beta_t$ (essentially).  I suppose I could do that here as well by accepting
singular $W$.  The mean and variance, $b_t$ and $V_t$ are known while $W$ is not
known.  (Aside: we could store $y_t$ and $X_t$ as lists.  But we could also
store them as vectors or matrices and keep track of the indexing using another
variable.)

\section{DLM}

It turns out that we can represent both the PG DLMs and the FS DLMs using the
same model.  FS includes a static term, which we can do and which we discuss
below, though not much changes when that is the case.  \textbf{What we discuss
  here informs our implementation of the FFBS in the BayesLogit package.}  The
basic model is
\[
\begin{cases}
z_t = x_{s,t} \alpha + x_{d,t} \beta_t + \ep_t, & \ep_t \sim N(0, V_t) \\
\beta_t = (1-\phi) \mu + \Phi \beta_t + \omega_t, & \omega_t \sim N(0, w).
\end{cases}
\]
Thus we have covariates that are associated with a static coefficient and
covariates that are associated with a dynamic coefficient. 

This encapsulates all of the models we encounter.  In the PG case, $V_t =
1/\omega_t$.  In the FS logistic case, $V_t = V_{r_t}$.  In the FS NB case $V_t
= V_{r_t}$ and we must use $z_t = \log(\lambda_t) + \log(d) + m_{r_t}$. where
$m_{r_t}$ and $V_{r_t}$ in this case from the representation of $-\log \Ga$.

We may further clarify what is going on by rewriting the system letting $x_t =
[x_{s,t}, x_{d,t}]$ (row vector) and $\theta_t = [\alpha_t; \beta_t]$ (col
vector) and then using the DLM
\[
\begin{cases}
z_t = x_t \theta_t + \ep_t, & \ep_t \sim N(0, V_t) \\
\theta_t = (1-\Phi) \mu + \Phi \theta_t + \omega_t, & \omega_t \sim N(0, W) \\
\Phi = 
\begin{pmatrix}
I & 0 \\
0 & \phi
\end{pmatrix}
\text{ and }
W = 
\begin{pmatrix}
0 & 0 \\
0 & w
\end{pmatrix}.
\end{cases}
\]
When there is no $\alpha$, then this just reduces to $\Phi = \phi$ and $W = w$.
We can filter forward as normal using $\Phi$ and $W$.  To backwards sample we
draw $\Theta_T | D_T$ and let $\alpha = \Theta_{0:P_a-1}$ if $P_a > 0$ (the
dimension of $\alpha$) and we let $\beta_T = \theta[P_a:P_b-1,T]$ where $P_b$ is
the dimension of $\beta_T$.  At this point, we may consider $\alpha$ a fixed
quantity.  When backwards sampling we just need to look at the joint
distribution of $(\beta_t, \beta_{t-1})$, which we can calculate using the
submatrices of $a_t, R_t, m_t, C_t$ from when we filtered forward.  In
particular, we just let (\texttt{il} stands for $i$ lag)
\begin{itemize}
\item \texttt{a.i  = a[b.idc,i]}
\item \texttt{m.il = m[b.idc,i-1]}
\item \texttt{R.i  = R[b.idc, b.idc, i]}
\item \texttt{C.il = C[b.idc, C.idc, i-1]}.
\end{itemize}

Thus, given these small caveats, we just need the FFBS described below with the
adjustments discussed above.  Think of $\beta_t$ as $\theta_t$ when including
$\alpha$.
\begin{outline}
\1 Time $t-1$ posterior:
\[
\beta_{t-1} \mid D_{t-1} \sim N(m_{t-1}, C_{t-1}).
\]
\1 Evolution / Prior:
\begin{align*}
a_t & = (I-\Phi) \mu + \Phi m_{t-1} \\
R_t & = \Phi C_{t-1} \Phi' + W \\
\beta_t & \mid D_{t-1} = N(a_t, R_t).
\end{align*}
\1 Observation:
\begin{align*}
f_t & = X_t a_t + b_t \\
Q_t & = X_t R_t X_t' + V_t \\
y_t & \mid D_{t-1} \sim N(f_t, Q_t).
\end{align*}
\1 Covariance:
\[
\cov(y_t, \beta_t | D_{t-1}) =: \rho_t = X_t R_t.
\]
\1 Joint:
\[
\begin{bmatrix}
y_t \\ \beta_t
\end{bmatrix}
\mid D_{t-1}
\sim
N
\Big(
\begin{bmatrix}
f_t \\ a_t
\end{bmatrix},
\begin{bmatrix}
Q_t & \rho_t \\
\rho_t' & R_t
\end{bmatrix}
\Big)
\]
\1 Regression matrix, error:
\begin{align*}
A_t & = \rho_t' Q_{t}^{-1} \\
e_t & = y_t - f_t.
\end{align*}
\1 Posterior
\begin{align*}
m_t & = a_t + A_t e_t \\
C_t & = R_t - \rho_t' Q_t^{-1} \rho_t \\
\beta_t & \mid D_t \sim N(m_t, C_t).
\end{align*}

\end{outline}

When calculating $Q_t^{-1}$ you want to consider the dimension of $X_t$, which
we will call $N \times P$.  In the PG/FS Logit/NB case, $X_t$ is $1 \times P$
and $Q_t$ is $1 \times 1$!  Thus just do $1/Q_t$ if in \textbf{PG/FS Logit/NB}
case.  If $N > P$ then you might want to use SWM, i.e.
\[
(V + XRX')^{-1} = V^{-1} + V^{-1} X(R^{-1} + X'V^{-1}X)^{-1}X' V^{-1}.
\]
If $P > N$ then you might just want to invert $Q$ directly.  It is possible that
$P > N$ since we may have only an observation or two for a given time point.
NOTICE that in former case, you do not need to calculate $Q$.  Thus $Q$ is $N
\times N$ ($N$ is the number of ``data points'' and $P$ is the dimension of
$\beta_t$) and
\begin{outline}
\1 $P \geq N$ and $P > 1$

  \2 Note: $A_t' = Q_t^{-1} \rho_t \iff Q_t A_t' = \rho_t$.
  \2 $L L' = Q_t$ so $LL' A_t' = \rho_t$.
  \2 $\xi = L^{-1} \rho_t \iff L \xi = \rho_t$.
  \2 $A_t' = L^{'-1} \xi \iff L'A_t' = \xi$.
  \2 $m_t = a_t + A_t e_t$.
  \2 $C_t = R_t - \xi' \xi$.

\1 $N > P$ (Maybe this should be something like $N > c P$ where $c > 1$.)

  \2 $T = R_t^{-1} + X_t'V_t^{-1}X_t$
  \2 $LL' = T$
  \2 $L^{'-1} X'V^{-1} = \xi \iff X'V^{-1} = L' \xi$.
  \2 $Q_t^{-1} = V_t^{-1} + \xi' \xi$.
  \2 $A_t = \rho_t' Q_t^{-1}$.
  \2 $m_t = a_t + A_t e_t$
  \2 $C_t = R_t - \rho_t' Q_t^{-1} \rho_t$.

\end{outline}

To backward sample we need $p(\beta_{t-1} \mid \beta_t, D_{t-1})$.
\begin{outline}
\1 Covariance:
\[
\cov(\beta_t, \beta_{t-1} \mid D_{t-1}) = \Phi C_{t-1}
\]
\1 Joint:
\[
\begin{bmatrix}
\beta_t \\ \beta_{t-1}
\end{bmatrix}
\mid D_{t-1}
\sim
N
\Big(
\begin{bmatrix}
a_t \\ m_{t-1}
\end{bmatrix},
\begin{bmatrix}
R_t & \Phi C_{t-1} \\
(\Phi C_{t-1})' & C_{t-1}
\end{bmatrix}
\Big)
\]
\1 Regression matrix, error:
\begin{align*}
B_t & = (\Phi C_{t-1})' R_t^{-1} \\
e_t & = \beta_t - a_t
\end{align*}
\1 Draw:
\begin{align*}
\ell & = m_{t-1} + B_t e_t \\
U & = C_{t-1} - (\Phi C_{t-1})' R_t^{-1} (\Phi C_{t-1}) \\
\beta_{t-1} & \mid \beta_t, D_{t-1} \sim N(\ell, U).
\end{align*}
\end{outline}

If you have missing data then
\[
\begin{cases}
m_t = a_t = \Phi m_{t-1} \\
C_t = R_t = \Phi C_{t-1} \Phi' + W.
\end{cases}
\]
This won't (I don't think) affect backwards sampling.  Whereas before you get a
reduction in variance in $C_t$ compared to $R_t$, now you do not.

\section{Static Probit vs. Static Logit}

Effective sample size.  These are the averages for 10 simulations.  Each
simulation was run 10000 time and the first 1000 samples were discarded.  For
both the logit and probit models a non-informative $N(0, 100)$ prior was used.
The simulations were run on a workstation with an Intel Xeon 2.40 GHz CPU and 4
GB of RAM.  Below, $N$ is the number of observations and $P$ is the number of
covariates.

You will notice that the running times for the PG method are drastically lower
than found in the PG paper.  This is because the code used in that paper drew
each $\omega_i$ one at a time.  It is much faster to use the \texttt{rpg}
functions ability to draw the entire vector $\omega$ at once.

I believe the reason the Probit model does better than the Logit model when
$\beta$ has many dimensions is because you can precompute the posterior variance
(and Cholesky factor) of $p(\beta | y, z)$ before Gibbs sampling.  You cannot do
that in the Logit case.

We also calculate the average ESS over observations and simulations of the
posterior probability of each success probability.  The posterior probability
can be calculated from the log odds $\psi_i = x_i \beta$ via $p_i = \exp(\psi_i)
/ (1 + \exp(\psi_i))$ for the logit model and from $p_i = \Phi(x_i \beta)$ for
the probit model.  We may encounter values of $x_i \beta$ in the probit model
that result in $p_1 = 1$ or $0$.  In that case the ESS calculation is degenerate
and we remove it from the average.

\begin{table}
\centering
\begin{tabular}{l c c c c c c}
          & $N$  & $P$ & $P^*$ & PG.ESS   &  Pro.ESS  & BL.to.Pro \\
\hline
Diabetes  & 768  & 8   & 9  & 4849.392 & 2547.9075 & 1.903284  \\
Heart     & 270  & 13  & 19 & 3203.718 & 1470.3308 & 2.178910  \\
Australia & 690  & 14  & 35 & 3218.647 &  995.6811 & 3.232609  \\
Germany   & 1000 & 20  & 49 & 4959.180 & 2485.7040 & 1.995081  \\
\end{tabular}
\caption{The average effective sample size of $\beta$ over components of $\beta$ and 
  simulations.  $N$ is the number of observations, $P$ is the
  number of covariates, and $P^*$ is the dimension of $\beta$.}
\end{table}

\begin{table}
\centering
\begin{tabular}{l c c c c c}
           & PG.Time & Pro.Time & PG.ESS.per.Time & Pro.ESS.per.Time &
           PG.Pro.ratio \\
\hline
Diabetes   & 12.290  &    8.435 &        394.5803 &        302.06372 & 1.3062816    \\
Heart      &  7.571  &    4.547 &        423.1565 &        323.36283 & 1.3086121    \\
Australia  & 24.039  &   10.217 &        133.8927 &         97.45338 & 1.3739158    \\
Germany    & 47.668  &   16.813 &        104.0358 &        147.84417 & 0.7036857      
\end{tabular}
\caption{The average effective sample sizes per execution time.}
\end{table}

\begin{table}
\centering
\begin{tabular}{l c c}
          & PG       & Pro      \\
\hline
Diabetes  & 4719.281 & 2499.716 \\
Heart     & 3109.270 & 1568.471 \\
Australia & 2681.208 & 1363.054 \\
Germany   & 4736.249 & 2401.344
\end{tabular}
\caption{Average effective sample size of the posterior probability of success
  over observations and simulations.  Degenerate ESS removed.}
\end{table}

\section*{FS\&F for NB}

See FS 2009.

The distribution for $k$ is
\[
{k+d-1 \choose k} (1-p)^d p^k.
\]
We could think of $p$ as the probability of success and $d$ as the number of
failures, in which case $k$ is the number of successes before there are $d$
failures; but that is not the best interpretation for our purposes, which we
will get to in a moment.  You can write $k \sim NB(d, p)$ as a Poisson-Gamma
mixture
\begin{align*}
& k \sim \text{Pois}(\lambda) \\
& \lambda \sim \text{Ga}(d, \text{scale}=\alpha) \; \text{ where } \alpha = \frac{p}{1-p}.
\end{align*}
This is all from Wikipedia.  

This arises because one wants to use a Poisson like model, but to have greater
variance in the number of counts.  The mean and variance of $X \sim
\text{Pois}(\lambda)$ is $\lambda$.  The mean of the negative binomial is
\[
\bbE[k] = \bbE[\bbE[k|\lambda]] = \bbE[\lambda] = d \alpha
\]
and the variance is
\begin{align*}
\Var(k) & = \bbE[\Var(k|\lambda)] + \Var(\bbE[k|\lambda]) \\
& = \bbE[\lambda] + \Var(\lambda) \\
& = d \alpha + d \alpha^2.
\end{align*}
Letting $\mu = d \alpha$ we have
\[
\bbE[k] = \mu \; \text{ and } \; \Var(k) = \mu + \frac{\mu^2}{d}.
\]
This is like the Poisson distribution, but with greater dispersion.

All of this is great since we have a scale-mixture of exponential-like things,
which is the key to FS\&F.  In particular,
\[
\lambda \sim \alpha \text{Ga}(d, 1)
\]
so that taking the $\log$ of both sides you have
\[
\log(\lambda) = \psi + \log(\text{Ga}(d,1))
\]
(or alternatively $\log (\lambda) = \psi - - \log(\text{Ga}(d,1))$) where
\[
\psi = \log \alpha, \; \text{ the ``log odds.''}
\]
I put this in quotes, because it isn't exactly the quantity we desire, but this
is what we model when using the \Polya-Gamma trick.  We can also write
\[
\psi = \log \mu - \log d,
\]
which shows that are choice of $\psi$ above is just a scaled version of the log
mean counts.  So long as we include an intercept in our regression we will have
the same posteriors for all predictors and produce the same forecasts.

Now we can use the normal mixture trick.  When $d=1$, we have an exponential, so
we can use FS\&F mixture where the mixture is $N(-m_r, \sigma_r^2)$ since she
looks at $-\log(\text{Ex}(1))$.  Everything else is conjugate.  In particular,
\[
p(k | \alpha, \lambda) = \frac{e^\lambda \lambda^k}{k!} 
\frac{\alpha^{-d} \lambda^{d-1}}{\Gamma(d)} e^{-\lambda / \alpha}.
\]
where $\alpha = \mu / d = e^\psi / d$.  To sample $\lambda$, which is just an
intermediate auxilliary, we have
\[
p(\lambda | \alpha, k) \propto \lambda^{k+d-1} e^{-\lambda (1+1/\alpha)} \propto
\text{Ga}(k+d, \text{rate}=1+1/\alpha).
\]
To sample $r$ we have a discrete mixture,
\[
p(r|\lambda, \psi) \propto \sigma_{r}^{-1} \exp 
\Big[ \frac{-\sigma_r^{-2}}{2} \Big(\log(\lambda) - \psi + m_r \Big)^2 \Big].
\]
This seems way too easy.  This would show that it is sometime easier to sample
what appears to be a more complicated distribution (a mixture vs. a non-mixture).

\subsection*{Estimating $d$}

From above it seems like we might be able to estimate $d$.  But this does not
work for FS\&F, since we need to know $d$ for the normal mixture.  However, I
think we might be able to use the Polya-Gamma trick. 

We know a few things.  First we know that $p(k | \alpha, d)$ is negative
binomial, which means that we can use the PG trick.  We also know that we can
sample $p(\lambda | \alpha, k, d)$.  Thus we should be able to sample
\[
p(\alpha, \lambda | k, d) =  p(\alpha | k, d) p(\lambda | \alpha, k, d)
\]
Furthermore, it looks to me like we can sample
\[
p(d | \alpha, \lambda).
\]
Thus we could maybe Gibbs sample so that
\begin{enumerate}
\item Sample $(\omega | \alpha, k, d)$.
\item Sample $(\alpha | \omega, k, d)$.
\item Sample $(\lambda | \alpha, k, d)$.
\end{enumerate}
Actually, \textbf{THIS IS WRONG}.  We need to sample
\[
(\lambda | \alpha, k, d, \omega).
\]
We could sample $\alpha$, approximately, be doing several PG steps.  What
happens when you do the above.  Can you take only one auxiliary step and still
guarentee covergence of the chain?  If you are going to do several auxiliary
steps, you might as well just look at the likelihood for $d$
\[
\frac{\Gamma(k+d)}{\Gamma(d)} (1-p)^d.
\]
I think it is possible to do something with this, but it is going to get ugly.
It is like a sum of gamma densities or something.

\section{Dynamic Principle Componenent Regression}

See Christiansen p. 385 for a more thorough discussion of principle component
regression.  Principle component regression is different than principle
component analysis.

With PCA you find orthogonal latent factors.  Given knowledge of the population
\[
x \sim N(0, \Sigma)
\]
then we know that
\[
x = Vy, \; y \sim N(0, D^2)
\]
where $\Sigma = VD^2V'$.  When working with a population $X \sim MN(0, I,
\Sigma)$ (where we have i.i.d rows) then we can take the SVD $X = \hat U \hat D
\hat V'$ so that $\hat V$ is an estimate of $V$ and $\hat D^2$ is an estimate of
$n D^2$ (or something like that).

In PC regression we take the design matrix $X = U D V'$ and then transform
\[
y = X \beta + \ep
\]
into
\[
U_*' y = 
\begin{bmatrix}
D \\ 0
\end{bmatrix}
\gamma + \ep.
\]
where $U_*$ has been made full rank by adding additional orthonormal columns.
The variance of $\gamma_i$ is $\sigma^2 / d_i^2$ thus if some of the $d_i$ are
small then it may be reasonable to drop the corresponding variables $\gamma_i$
from the regression.

James wants to use princple component regression in the dynamic case.  Suppose
that $X = UDV$ and that $x_t$ is a row of $X$ and $\beta_t$ as a column of
$\beta$.  Then our dynamic regression involves
\[
y_t = x_t \cdot \beta_t + \ep_t.
\]
Specifically
\[
y_t = \sum_{i} X_{ti} \beta_{it} + \ep_t
\]
The SVD of $X$ can be written as
\[
X_{ti} = \sum_{j} d_j u_{tj} v_{ij}.
\]
Thus we have
\[
y_t = \sum_i \sum_j d_j u_{tj} v_{ij} \beta_{it} + \ep_t.
\]
Letting
\[
\gamma_{jt} = \sum_i v_{ij} \beta_{it}
\]
that is
\[
\gamma_t = V' \beta_t.
\]
Thus we have
\[
y_t = \sum_{j} d_j u_{tj} \gamma_{jt} + \ep_t
\]
or
\[
y_t = z_t \cdot \gamma_t + \ep_t
\]
where
\[
z_{tj} = d_j u_{tj} \text{ or } z_t = u_t D \text{ or } Z = UD.
\]
Just assuming that the logic carries over, if $d_j$ is small, then remove it
from the regression to reduce the dimensionality of the problem.

Actually, this may be slightly different than princple components regression.
It appears that principal components removes the mean in each column of $X$.
This would align with the initial assumptions we mentioned.

I think the same is going to hold here.  The only thing that matters is the
column space of $X$ so we can do a change of variables so that
\[
y = \alpha 1 + X \beta + \ep
\]
produces the same fitted values and residuals as
\[
y = \kappa 1 + (X - 1 m_x') \gamma + \ep.
\]
The same argument holds as above, but now we do the SVD on $X- 1 m_x'$.  But now
we ensure that we can have an intercept.  In that case we have
\[
y_t = \kappa + z_t \gamma_t + \ep_t.
\]
After $X$ has been normalized, take the SVD to get $UDV'$.  $Z=UD$ is called the
scores I believe.

\section{Dynamic Negative Binomial}

Unfortunately, it appears that FS may be competitive.  I have yet to do the
actually simulations; but it appears that \cite{fruhwirth-schnatter-etal-2009}
may have a better ESR than the PG method.  This is due to the relative
inefficieny when calculating $PG(y+d, \psi)$ random variables.  Whenver the
count $y$ is high enough it is going to take a while to simulate all of the
random variables we need.  Yet more motivation for finding a better $PG(n, z)$
sampler.

See \texttt{notes.tex} for more discussion of negative binomial.

In those models we have
\[
\begin{cases}
y_t \sim \Pois(\lambda_t) \\
\lambda_t \sim \mu_t \Ga(d, \text{scale}=1/d) \\
\log(\mu_t) = x_t \beta.
\end{cases}
\]
We are modeling the log-mean (or log-odds).  We can make this dynamic by In
those models we have
\[
\begin{cases}
y_t \sim \Pois(\lambda_t) \\
\lambda_t \sim \mu_t \Ga(d, \text{scale}=1/d) \\
\lambda_t \sim \alpha_t \Ga(d, 1) \\
\log(\mu_t) = x_t \beta_t \\
\log(\alpha_t) = x_t \beta_t - \log(d) \\
\beta_t = \beta_{t-1} + \omega_t, & \omega_t \sim N(0, W).
\end{cases}
\]
Or if we are following the principle components regression above we have
\[
\begin{cases}
y_t \sim \Pois(\lambda_t) \\
\lambda_t \sim \mu_t \Ga(d, \text{scale}=1/d) \\
\lambda_t \sim \alpha_t \Ga(d, 1) \\
\log(\mu_t) = \hat \kappa + z_t \beta_t \\
\log(\alpha_t) = \kappa + z_t \beta_t, & \kappa = \hat \kappa - \log(d) \\
\beta_t = \beta_{t-1} + \omega_t, & \omega_t \sim N(0, W).
\end{cases}
\]
where $z_t$ now does not include an intercept.  It is convenient to have several
quantities to work with so 
\[
\begin{cases}
\zeta_t = x_t \beta_t \\
\psi_t = \zeta_t - \log(d) \\
\psi_t = \log(\alpha_t) \\
\alpha_t = \frac{p_t}{1-p_t}.
\end{cases}
\]
We could model the log-odds directly.  There is an R-script
\texttt{NBPG-logodds.R} that does that in the non-dynamic case.  But it is
traditional to model the log-mean.

For FSF, I think pretty much nothing changes between the static and dynamic
case:

\begin{enumerate}

\item Sample $(d, r, \lambda | \{\beta_t\})$ via

\begin{enumerate}
\item $p(d | \mu, y)$ where $\mu = \{\mu_t\}$
\item $p(\lambda | \alpha(d), d, y)$ where $\alpha = \alpha_t(d)$
\item $p(r | \lambda, \alpha, d, y)$.
\end{enumerate}

\item Sample $( \{\beta_t\} | r, \lambda, d, y )$.

  We need to FFBS now instead of LS using our normal-mixture term for the
  observation.

  Also we need to calculate $\mu_t$ differently.

  We also must deal with sampling the intercept.

\end{enumerate}

For the PG method we again sample $p(d, \omega | y, \beta)$ by $p(d | y, \beta)
p(\omega | d, y, \beta)$.  The likelihood for $\{\beta_t\}, \omega$ is
\[
\propto \prod_{t=1}^T e^{\psi_i \kappa_i} e^{-\omega_i \psi_i^2 / 2} p(\omega_i | y_i + d_i).
\]
where $\psi_t = \hat \iota + x_t \beta_t - \log(d)$.  In this case we see that
it really isn't a big deal to model the log-odds instead of the log-mean.  If we
model $\psi_t = \iota + x_t \beta_t$ where $\iota = \hat \iota - \log d$ then we
can recover $\mu_t = \psi_i + \log d$.  In either case though it doesn't really
matter.  We can always chose to roll $\log d$ into $z$.

Following the identical logic as above if we multiply the likelihood by
$\exp - \frac{1}{2} z \Omega z$ where $\Omega z = \kappa$ then we have
\[
\exp \Big( -\frac{1}{2} (z - \psi)' \Omega (z- \psi) \Big).
\]
Thus we have
\[
\begin{cases}
z = \psi + \ep, \; \ep \sim N(0, \Omega^{-1}) \\
\Omega z = \kappa
\end{cases}
\]
which becomes
\[
\begin{cases}
z_t = \iota + x_t \beta_t + \ep_t, & \ep_t \sim N(0, 1/\omega_t) \\
z_t = \kappa_t / \omega_t \\
\beta_t = \phi \beta_{t-1} + \nu_t, & \nu_t \sim N(0, W).
\end{cases}
\]
Or in the case of log-mean we have
\[
\begin{cases}
\hat z_t = z_t + \log(d) = \hat \iota + x_t \beta_t + \ep_t, 
  & \ep_t \sim N(0, 1/\omega_t) \\
z_t = \kappa_t / \omega_t \\
\beta_t = \phi \beta_{t-1} + \nu_t, & \nu_t \sim N(0, W).
\end{cases}
\]
I don't think it makes sense to have an intercept $\iota$ when we are using a
random walk.  If we have some covariates then I think it is okay.  But if we
just had
\[
\begin{cases}
z_t = \iota + \beta_t + \ep_t \\
\beta_t = \beta_{t-1} + \nu_t
\end{cases}
\]
then $\beta_t$ is not identified.  (Recall however, that Scott 2011 mentions
something about working parameters.  Evidently it is sometime okay to let
parameters not be identified and then to later identify them.)

I believe in either case we have
\[
\begin{cases}
z_t = \iota + x_t \beta_t + \ep_t, \; \ep_t N(0, V_t) \\
\beta_t \sim GP
\end{cases}
\]
where GP refers to Gaussian process, in place of AR(1) or a random walk or
whatever it is that we have.

We need to be somewhat careful about whether we are using covariates or not.
When we have
\[
\begin{cases}
z_t = \iota + x_t \beta_t + \ep_t \\
\beta_t \sim AR(1)
\end{cases}
\]
it is okay where $\beta_t$ has a non-zero mean.
When we have
\[
\begin{cases}
z_t = \iota + \beta_t + \ep_t \\
\beta_t \sim AR(1)
\end{cases}
\]
then $\beta_t$ should not have a mean.  When $\beta_t$ is a random walk then
there should be no mean level $\iota$.

\section{Dynamic Binary Logistic Regression}



% If you have a bibliography.
% The file withe bibliography is name.bib.
\bibliography{bayeslogit}{}
\bibliographystyle{abbrvnat}

\end{document}
