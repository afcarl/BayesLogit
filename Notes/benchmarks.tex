\documentclass{article}

\usepackage{amsmath}
\usepackage{amssymb}
\usepackage{amsthm}
\usepackage{fancyhdr}
\usepackage{makeidx}
\usepackage{url}
\usepackage{graphicx}
\newcommand{\one}{\bold{1}}
\newcommand{\ep}{\varepsilon}

%\input{commands}
\usepackage{natbib}
% \usepacakge{parskip}
\usepackage{outlines}

\newcommand{\Polya}{P\'{o}lya}
\newcommand{\PG}{\text{PG}}

% Page Layout
\setlength{\evensidemargin}{0.0in}
\setlength{\oddsidemargin}{0.0in}
\setlength{\textwidth}{6.5in}
\setlength{\textheight}{8in}

% Set Fancy Style
%\pagestyle{fancy}
%\lhead{Left Header}
%\rhead{Right Header}

\begin{document}

\section{Benchmarks}

We benchmark the \Polya-Gamma method against several alternatives for binary
logistic regression and negative binomial regression for count data to measure
its relative performance.  All of these benchmarks are empirical and hence some
caution is urged.  Our primary metric of comparison is the effective sampling
rate, which is the effective sample size per second.  This is the most practical
metric since it tells us how quickly one can produce independent draws from the
posterior distribution.  However, this metric is also sensitive to numerous
idiosyncrasies relating to the implementation of the routines.  We generate
these benchmarks using R, though some of the routines make calls to external C
code.  The specifics of each method are discussed in further detail below.  In
general, we found that the \Polya-Gamma technique compared favorably to other
data augmentation methods; specifically, the \Polya-Gamma technique performed
better than the methods of Gramacy and Polson (2011) and Fr\"{u}wirth-Schnatter,
et al. (2010).  The latter paper itself provides a detailed comparison of
several methods itself.  For instance, the authors found that method of Holmes
and Held (2006) did not beat their discrete mixture of normals, something that
we found as well and hence omitted from the comparisons below.  For the simple
case of binary logistic regression we found that often an independence
Metropolis sampler performed well.  We draw the following conclusions from these
numerical experiments:

\begin{itemize}

\item In binary logistic regression, the \Polya-Gamma method performs better than
other data augmentation techniques.

\item In negative binomial regression, the \Polya-Gamma method performs well
  when the count sizes are not too high or when the regression coefficient has a
  complex underlying structure, for instance, when it is a Gaussian process.

\item Independence Metropolis works well for binary logistic regression.  This
  is not surprising as the asymptotics suggest that a normal approximation may
  be reasonable.  Even when the \Polya-Gamma method has a superior effective
  sample size, it may be that the independence Metropolis sampler does better in
  terms of effective sampling rate since there is very little to compute at each
  iteration.  Nonetheless, there are times in which the \Polya-Gamma method is
  better in terms of effective sampling rate.

\item An important advantage of data augmentation, and the \Polya-Gamma
  technique in particular, is that it is easy to construct and code more
  complicated models, whereas independence Metropolis only works well in the
  simplest case.  For instance, the \Polya-Gamma method easily accommodates more
  sophisticated models that produce a Gaussian ``prior'' distribution for
  $\beta$.  

\end{itemize}

\subsection{Benchmarking Procedure}

For each benchmark stated below, we ran 10 MCMC simulations with 12,000 samples
each, discarding the first 2,000 as burn in to leave a total of 10 batches of
10,000 samples each.  The effective sample size for each regression coefficient
was calculated using the \texttt{coda} package and then averaged across the 10
batches.  We then calculated the minimum, median, and maximum of these average
effective sample sizes.  A similar calculation was performed for the effective
sampling rates.  The time and average acceptance rate are averaged across the 10
batches as well.  The time refers to the time taken to produce the 10,000
samples, not the total run time.  Thus any preprocessing and burn-in are not
included in the total time reported.  When discussing the various methods the
primary metric we refer to is the median effective sampling rate, following the
example of \cite{fruhwirth-schnatter-fruhwirth-2010}.

\subsection{Binary Logistic Regression}

\subsubsection{The Data Sets}

We benchmark against the data sets used by \cite{holmes-held-2006} and
\cite{fruhwirth-schnatter-fruhwirth-2010}.

\begin{outline}

\1 Nodal (part of the \texttt{boot} R package) :

The response indicates if cancer has spread from the prostate to surrounding
lymph nodes.  There are 53 observations and 5 binary predictors.

\1 Pima Indian:

\url{http://archive.ics.uci.edu/ml/datasets/Pima+Indians+Diabetes}

There are 768 observations and 8 continuous predictors.  It is noted on the UCI
website that there are many predictor values coded as 0, which correspond to a
physical measurement that should be non-zero.  We have removed all of those
entries to generate a data set with 392 observations.  The marginal mean
incidence of diabetes is roughly 0.33 before and after removing the data.

\1 Heart:

\url{http://archive.ics.uci.edu/ml/datasets/Statlog+(Heart)}

The response represents either an absence or presence of heart disease.  There
are 270 observations and 13 attributes of which 6 are categorical or binary and
1 is ordinal.  The ordinal covariate has been stratified by dummy variables.

\1 Australian Credit: 

\url{http://archive.ics.uci.edu/ml/datasets/Statlog+(Australian+Credit+Approval)}.

The response represents either accepting or rejecting a credit card application.
The meaning of each predictor has been removed to protect the propriety of the
original data.  There are 690 observations and 14 attributes, of which 8 are
categorical or binary.  There were 37 observations with missing attribute
values.  These missing values have been replaced by the mode of the attribute in
the case of categorical data and the mean of the attribute for continuous data.
This dataset is linearly separable and results in some divergent regression
coefficients.

\1 German Credit: 

\url{http://archive.ics.uci.edu/ml/datasets/Statlog+(German+Credit+Data)}

The response represents either a good or bad credit risk.  There are 1000
observations and 20 attributes including both continuous and categorical data.
We benchmark two scenarios.  In the first, the ordinal covariates have been
given integer values and have not been stratified by dummy variable, yielding a
total of 24 numeric predictors.  In the second, the ordinal data by dummy
variables, yielding a total of 48 predictors.

\end{outline}

\subsubsection{The Methods}

All of these routines are implemented in R, though some of them make calls to C.
In particular, the independence Metropolis samplers do not make use of any
non-standard calls to C, though their implementation's have very little R
overhead in terms of function calls; the \Polya-Gamma method calls a C routine
to sample the \Polya-Gamma random variates, but otherwise only uses R.  We think
this is fair since other basic random variate generators in R call compiled
code.  As a check upon our independence Metropolis sampler we also used Rossi,
Allenby, and McCulloch's independence Metropolis sampler, which may be found in
the \texttt{bayesm} package, though their sampler uses a $t_6$ proposal while
ours uses a normal proposal.  The suite of routines in the \texttt{binomlogit}
package implement the techniques discussed in
\cite{fruhwirth-schnatter-fruhwirth-2010}.  The benchmark labeled FS refers to
the discrete mixture of normals used in
\cite{fruhwirth-schnatter-fruhwirth-2010} and makes use of single C wrapper.
Specifically, the methods FS and dRUMAuxMix are different implementations of the
same method; we see that they have similar effective sample sizes.  Gramacy and
Polson's R package, \texttt{reglogit}, also calls external C code.  For every
data set we used at diffuse $N(0, 0.01 I)$ prior.

\begin{outline}

\1 PG: The \Polya-Gamma method described above.

\1 FS: \cite{fruhwirth-schnatter-fruhwirth-2010} follow Holmes and Held (2006)
and use the \cite{albert-chib-1993} like representation,
\begin{equation}
\label{eqn:hh-rep}
\begin{cases}
y_i = \one \{ z_i > 0 \} \\
z_i = x_i \beta + \ep_i, & \ep_i \sim \text{Lo}
\end{cases}
\end{equation}
where $\text{Lo}$ is the standard logistic distribution.  They approximate the
distribution of $\ep_i$ using a discrete mixture of normals.

\1 IndMH: Independence Metropolis with a normal proposal.  Calculate the
posterior mode and the Hessian at the mode to pick the mean and variance of the
proposal.

\1 RAM: Independence Metropolis with a $t_6$ proposal from the R package
\texttt{bayesm}.  Calculate the posterior mode and the Hessian at the mode to
pick the mean and scale matrix of the proposal.

\1 dRUMAuxMiX: Fr\"{u}wirth-Schnatter's discrete mixture of normals method from
the R package \texttt{binomlogit}.

\1 dRUMIndMH: Exploit the representation (\ref{eqn:hh-rep}), but use a single
normal to approximate the logistic distribution and then correct using a
Metropolis-Hastings step.  From the R package \texttt{binomlogit}.

\1 IndivdRUMIndMH: This is the same as dRUMIndMH, but specific to binary
logistic regression.  From the R package \texttt{binomlogit}.

\1 dRUMHAM: Again, use the representation (\ref{eqn:hh-rep}), but now
approximate the logistic distribution using a discrete mixture of normals, the
number of components to mix over being determined by $y_i / n_i$.  From the R
package \texttt{binomlogit}.

\1 GP: Gramacy and Polson (2011).  Another data augmentation scheme with only a
single layer of latents.  This routine uses a double exponential prior.  We set
the scale of this prior to agree with the scale of the normal prior we used in
all other cases above.

\end{outline}



\begin{table}
\begin{tabular}{l r r r r r r r r } 
          Method  &     time &    ARate &  ESS.min &  ESS.med &  ESS.max &  ESR.min &  ESR.med &  ESR.max \\ 
              PG  &     2.84 &     1.00 &  3236.18 &  4694.08 &  5471.59 &  1139.03 &  1652.40 &  1925.36 \\ 
           IndMH  &     1.77 &     0.66 &   803.41 &  1327.05 &  1600.46 &   454.98 &   753.52 &   908.99 \\ 
             RAM  &     1.28 &     0.64 &  3057.88 &  3629.46 &  3906.55 &  2398.51 &  2845.99 &  3061.71 \\ 
       dRUMIndMH  &     1.40 &     0.61 &   655.02 &  1076.01 &  1305.05 &   466.95 &   767.64 &   931.47 \\ 
  IndivdRUMIndMH  &     1.29 &     0.61 &   668.46 &  1072.60 &  1312.53 &   519.49 &   833.70 &  1021.08 \\ 
         dRUMHAM  &     3.02 &     1.00 &   985.47 &  1624.80 &  1899.60 &   325.79 &   537.39 &   628.43 \\ 
      dRUMAuxMix  &     2.66 &     1.00 &   975.11 &  1574.32 &  1871.82 &   366.61 &   591.79 &   703.62 \\ 
              FS  &     3.45 &     1.00 &   998.37 &  1668.35 &  1958.26 &   289.88 &   484.29 &   568.32 \\ 
              GP  &    17.77 &     1.00 &  2880.65 &  4365.66 &  5393.00 &   162.13 &   245.73 &   303.56
 \end{tabular}
\caption{table.nodal, N=53, P=6}
\end{table}

% table.diabetes, N=768, P=9
% \begin{tabular}{l r r r r r r r r } 
%           Method  &     time &    ARate &  ESS.min &  ESS.med &  ESS.max &  ESR.min &  ESR.med &  ESR.max \\ 
%               PG  &     8.37 &     1.00 &  3393.69 &  5743.68 &  6593.19 &   405.59 &   686.40 &   787.95 \\ 
%               FS  &     9.89 &     1.00 &  1085.33 &  2071.12 &  2527.38 &   109.76 &   209.45 &   255.59 \\ 
%            IndMH  &     2.66 &     0.87 &  5598.19 &  6886.27 &  7096.68 &  2103.60 &  2588.52 &  2667.35 \\ 
%              RAM  &     2.58 &     0.69 &  5091.41 &  5210.76 &  5275.66 &  1971.45 &  2017.49 &  2042.51 \\ 
%        dRUMIndMH  &     6.34 &     0.46 &   601.11 &  1122.32 &  1320.58 &    94.80 &   177.02 &   208.24 \\ 
%          dRUMHAM  &    11.00 &     1.00 &  1097.06 &  2069.99 &  2579.00 &    99.75 &   188.18 &   234.47 \\ 
%       dRUMAuxMix  &     9.53 &     1.00 &  1126.79 &  2129.16 &  2571.73 &   118.18 &   223.32 &   269.73 \\ 
%   IndivdRUMIndMH  &     4.61 &     0.47 &   661.28 &  1149.52 &  1337.61 &   143.53 &   249.52 &   290.36 \\ 
%               GP  &   165.87 &     1.00 &  2838.24 &  6925.85 &  9622.92 &    17.11 &    41.75 &    58.01
%  \end{tabular}

\begin{table}
\begin{tabular}{l r r r r r r r r } 
          Method  &     time &    ARate &  ESS.min &  ESS.med &  ESS.max &  ESR.min &  ESR.med &  ESR.max \\ 
              PG  &     5.50 &     1.00 &  3387.77 &  5589.29 &  6325.36 &   616.13 &  1016.26 &  1150.37 \\ 
           IndMH  &     2.19 &     0.81 &  4044.47 &  5202.09 &  5619.34 &  1851.44 &  2379.91 &  2569.74 \\ 
             RAM  &     1.90 &     0.67 &  4766.51 &  4917.57 &  5007.96 &  2508.74 &  2587.49 &  2635.62 \\ 
       dRUMIndMH  &     3.79 &     0.49 &   642.76 &  1124.83 &  1261.79 &   169.68 &   296.91 &   333.13 \\ 
  IndivdRUMIndMH  &     2.90 &     0.48 &   664.70 &  1133.56 &  1322.40 &   229.50 &   391.38 &   456.55 \\ 
         dRUMHAM  &     6.89 &     1.00 &  1134.40 &  1966.06 &  2394.83 &   164.59 &   285.33 &   347.57 \\ 
      dRUMAuxMix  &     6.00 &     1.00 &  1077.36 &  2045.34 &  2443.66 &   179.63 &   340.99 &   407.44 \\ 
              FS  &     6.51 &     1.00 &  1098.33 &  1989.25 &  2440.23 &   168.81 &   305.76 &   375.09 \\ 
              GP  &    88.06 &     1.00 &  2886.96 &  6232.88 &  9342.76 &    32.78 &    70.78 &   106.10
 \end{tabular}
\caption{table.diabetes, N=392, P=9}
\end{table}


\begin{table}
\begin{tabular}{l r r r r r r r r } 
          Method  &     time &    ARate &  ESS.min &  ESS.med &  ESS.max &  ESR.min &  ESR.med &  ESR.max \\ 
              PG  &     5.51 &     1.00 &  2117.55 &  3623.72 &  4725.49 &   384.15 &   657.40 &   857.32 \\ 
           IndMH  &     2.24 &     0.39 &   548.43 &   713.34 &   937.31 &   244.93 &   318.24 &   418.79 \\ 
             RAM  &     1.98 &     0.31 &   975.05 &  1086.12 &  1240.50 &   492.55 &   548.49 &   626.86 \\ 
       dRUMIndMH  &     3.51 &     0.34 &   297.67 &   455.98 &   607.01 &    84.81 &   129.95 &   172.99 \\ 
  IndivdRUMIndMH  &     2.89 &     0.34 &   284.84 &   471.20 &   615.59 &    98.45 &   162.80 &   212.79 \\ 
         dRUMHAM  &     7.03 &     1.00 &   611.91 &  1125.91 &  1547.60 &    86.99 &   160.06 &   220.01 \\ 
      dRUMAuxMix  &     5.95 &     1.00 &   610.12 &  1105.61 &  1515.15 &   102.48 &   185.67 &   254.48 \\ 
              FS  &     6.52 &     1.00 &   596.15 &  1115.82 &  1516.77 &    91.40 &   171.08 &   232.55 \\ 
              GP  &    65.59 &     1.00 &  1383.36 &  3094.06 &  4861.40 &    21.09 &    47.18 &    74.11
 \end{tabular}
\caption{table.heart, N=270, P=19}
\end{table}

\begin{table}
\begin{tabular}{l r r r r r r r r } 
          Method  &     time &    ARate &  ESS.min &  ESS.med &  ESS.max &  ESR.min &  ESR.med &  ESR.max \\ 
              PG  &    12.68 &     1.00 &   408.65 &  3857.54 &  5156.60 &    32.24 &   304.31 &   406.82 \\ 
           IndMH  &     3.44 &     0.21 &   210.39 &   340.61 &   401.55 &    61.13 &    98.82 &   116.74 \\ 
             RAM  &     3.92 &     0.00 &     7.35 &    10.90 &    18.70 &     1.87 &     2.78 &     4.77 \\ 
       dRUMIndMH  &     8.93 &     0.19 &    10.89 &   225.38 &   443.17 &     1.22 &    25.25 &    49.65 \\ 
  IndivdRUMIndMH  &     7.36 &     0.20 &     9.90 &   230.74 &   434.07 &     1.35 &    31.36 &    58.98 \\ 
         dRUMHAM  &    18.61 &     1.00 &    29.48 &  1007.57 &  1766.25 &     1.58 &    54.14 &    94.90 \\ 
      dRUMAuxMix  &    14.91 &     1.00 &    26.71 &  1001.67 &  1776.91 &     1.79 &    67.21 &   119.23 \\ 
              FS  &    15.02 &     1.00 &    31.89 &  1009.93 &  1760.91 &     2.12 &    67.24 &   117.23 \\ 
              GP  &   162.91 &     1.00 &   656.68 &  2699.28 &  8199.63 &     4.03 &    16.57 &    50.33
 \end{tabular}
\caption{table.aus, N=690, P=35}
\end{table}

\begin{table}
\begin{tabular}{l r r r r r r r r } 
          Method  &     time &    ARate &  ESS.min &  ESS.med &  ESS.max &  ESR.min &  ESR.med &  ESR.max \\ 
              PG  &    15.21 &     1.00 &  3080.09 &  5870.95 &  6558.04 &   202.52 &   386.02 &   431.17 \\ 
           IndMH  &     3.56 &     0.68 &  2516.99 &  3599.24 &  3929.58 &   706.15 &  1010.31 &  1103.22 \\ 
             RAM  &     4.14 &     0.43 &  1983.50 &  2307.02 &  2460.59 &   478.69 &   556.70 &   593.73 \\ 
       dRUMIndMH  &    10.56 &     0.28 &   335.00 &   684.32 &   798.90 &    31.72 &    64.79 &    75.64 \\ 
  IndivdRUMIndMH  &     8.32 &     0.29 &   359.16 &   685.09 &   800.67 &    43.14 &    82.29 &    96.18 \\ 
         dRUMHAM  &    22.05 &     1.00 &   943.64 &  2152.44 &  2476.96 &    42.81 &    97.63 &   112.36 \\ 
      dRUMAuxMix  &    18.07 &     1.00 &   984.15 &  2173.42 &  2494.83 &    54.46 &   120.27 &   138.05 \\ 
              FS  &    18.09 &     1.00 &   938.76 &  2121.63 &  2510.62 &    51.91 &   117.32 &   138.82 \\ 
              GP  &   223.75 &     1.00 &  2568.73 &  5509.13 &  8477.01 &    11.48 &    24.62 &    37.89
 \end{tabular}
\caption{table.ger.num, N=1000, P=25}
\end{table}

\begin{table}
\begin{tabular}{l r r r r r r r r } 
          Method  &     time &    ARate &  ESS.min &  ESS.med &  ESS.max &  ESR.min &  ESR.med &  ESR.max \\ 
              PG  &    22.04 &     1.00 &  2816.67 &  5815.63 &  6687.91 &   127.78 &   263.83 &   303.43 \\ 
           IndMH  &     4.72 &     0.41 &   790.39 &  1069.59 &  1295.76 &   167.17 &   226.45 &   274.29 \\ 
             RAM  &     6.03 &     0.00 &    11.43 &    25.42 &   358.67 &     1.90 &     4.22 &    59.64 \\ 
       dRUMIndMH  &    14.67 &     0.13 &   120.73 &   285.11 &   339.44 &     8.23 &    19.44 &    23.14 \\ 
  IndivdRUMIndMH  &    12.42 &     0.13 &   138.03 &   295.62 &   354.30 &    11.11 &    23.80 &    28.52 \\ 
         dRUMHAM  &    35.71 &     1.00 &   663.68 &  2084.63 &  2528.34 &    18.58 &    58.38 &    70.82 \\ 
      dRUMAuxMix  &    26.82 &     1.00 &   765.46 &  2088.36 &  2600.62 &    28.54 &    77.87 &    96.97 \\ 
              FS  &    26.36 &     1.00 &   753.35 &  2059.56 &  2563.18 &    28.58 &    78.13 &    97.24 \\ 
              GP  &   243.68 &     1.00 &  2065.53 &  5392.61 & 10000.00 &     8.48 &    22.13 &    41.04
 \end{tabular}
\caption{table.ger, N=1000, P=49}
\end{table}

%%% Local Variables:
%%% TeX-master: "benchmarks.tex"
%%% End:

Independence Metropolis works well for binomial regression; however, a major
advantage of data augmentation, and hence the \Polya-Gamma technique, is that it
is easily adapted to more complicated models.  Consider, for instance, binary
regression with multiple random intercepts, in which case the log odds are
modeled as
\[
\begin{cases}
\psi_{ij} = \alpha_i + x_{ij} \beta \\
\alpha_i \sim N(0, 1/\phi)
\end{cases}
\]
for observation $j$ in group $i$.  An extra step is easily added to the
\Polya-Gamma Gibbs sampler to estimate $(\alpha, \beta)$ and $\phi$; however, it
is not clear that an independence Metropolis sampler is as easily constructed.
The simplest possible route is to write down the new log-posterior involving
$\phi$ and use the subsequent Laplace approximation.  However, as shown in Table
(\ref{tab:mm-example}) for a synthetic dataset, this may result in an inferior
effective sample rate.  One might consider implementing a Metropolis step within
Gibbs to sample $(\alpha,\beta)$, but this would require re-estimating the
posterior mode and variance of $(\alpha,\beta)$ at each step, a time consuming
process.  Without prejudicing the other possible routes to finding a better
Metropolis-Hastings based sampler, we simply point out that what works well in
the simplest case does not necessarily work well in a slightly more complicated
case; however, the \Polya-Gamma method works equally well in both cases.  This
discrepancy would be exacerbated further were we to consider a yet more
complicated model, such as a larger mixed model or a factor model for binary
data.  In the latter one must estimate a set of factor loadings, the factor
variances, and the idiosyncratic variances---something that one may easily
incorporate into a \Polya-Gamma Gibbs sampler, but is less well suited to a
joint draw from an independence Metropolis sampler.  In sum, data augmentation
is useful, and the \Polya-Gamma technique is the best choice within that class
for binary regression.

\begin{table}
\label{tab:mm-example}
\begin{tabular}{l r r r r r r r r } 
 Method  &     time &    ARate &  ESS.min &  ESS.med &  ESS.max &  ESR.min &  ESR.med &  ESR.max \\ 
   PGMM  &     6.82 &     1.00 &  4297.65 &  6599.02 &  7052.02 &   630.41 &   968.01 &  1034.42 \\ 
IndMHMM  &     3.53 &     0.42 &  1256.11 &  1501.59 &  1694.51 &   356.27 &   425.35 &   480.10 \\ 
     PG  &     5.70 &     1.00 &  4478.19 &  6863.88 &  7367.59 &   785.55 &  1204.03 &  1292.48 \\ 
  IndMH  &     2.25 &     0.91 &  6646.28 &  7765.85 &  7938.66 &  2949.70 &  3445.48 &  3523.12
 \end{tabular}
 \caption{
   The effective sample sizes found for a synthetic data set with 
   5 groups, corresponding to 5 different intercepts drawn from the same
   distribution $N(0, 1)$, 100 observations within each group (for a total of
   500 observations), and 1 shared covariate.  The covariate was drawn from
   $N(0,1/\phi), \phi=1$.  PGMM is the \Polya-Gamma Gibbs sampler for the mixed model.  IndMHMM
   is the independence Metropolis sampler for the mixed model using a normal
   proposal based upon the Laplace approximation.  We tried using a $t_6$ proposal
   as well, but that did not improve the results.  The last two rows correspond to
   the case when $\phi$ is known, which returns us to binomial regression with a
   $N(0, I)$ prior $\alpha$.  In the mixed models we used a $\text{Ga}(1,
   \text{rate}=1)$ prior for $\phi$.  Notice that the effective sampling rate,
   which is taken over the components of $(\alpha, \beta)$, is
   excellent in the binomial regression case, but drops precipitously for
   unknown $\phi$.
 }
\end{table}

\subsection{Negative Binomial Regression}

The \Polya-Gamma method shows excellent effective sample sizes in negative
binomial regression.  However, its effective sampling rates fare less well,
especially when working with lots of counts.  Currently, our \Polya-Gamma
sampler can draw from $\PG(n, \psi)$, $n=1$ quickly.  But we have not
constructed a fast sampler for general $n$.  To sample from $\PG(n, \psi)$ for
integral $n > 1$ we take $n$ independent samples of $\PG(1, \psi)$ and sum them.
In the negative binomial case one must sample roughly $\sum_{i=1}^N y_i$, where
$y_i$ is the response, at every iteration.  When the number of counts is
relatively high this becomes a burden and the \Polya-Gamma sampler performs
poorly.  We see this in Table (\ref{tab:nb-synth}), where the \Polya-Gamma
sampler does well when working with relatively small count sizes, but poorly
when that is not the case.  In both cases $N=400$ synthetic data points were
created using the model
\[
\begin{cases}
y_i \sim NB(\textmd{mean}=\mu_i, d) \\
\log \mu_i = \alpha + x_i \beta
\end{cases}
\]
with three covariates excluding the intercept.  The parameter $d$ is estimated
using a random-walk Metropolis-Hastings step.  The model with fewer counts
corresponds to $\alpha = 2$ while the model with more counts corresponds to
$\alpha = 3$, which produced a sample mean of roughly 8 in the former and 24 in
the latter.  Notice that the \Polya-Gamma method has superior effective sample
size in both cases.  It is an open challenge to create an efficient \Polya-Gamma
sampler for arbitrary $n$, which would make it the best choice in both cases.

The \Polya-Gamma method starts to shine when working with more complicated
models that devote proportionally less time to sampling the auxiliary variables.
For instance, consider the model
\[
\begin{cases}
y_i \sim NB(\text{mean} = \mu(x_i), d) \\
\log \mu  \sim GP(0, K)
\end{cases}
\]
where $K$ is the square exponential covariance kernel,
\[
K(x_1,x_2) = \exp \Big( \frac{\|x_1 - x_2\|^2}{2 \ell^2} \Big)
\]
with characteristic length scale $\ell$.  Using either the \Polya-Gamma or
\cite{fruhwirth-schnatter-etal-2009} data augmentation techniques one arrives at
a complete conditional for $\upsilon = \log \mu$ that appears to be generated by
\[
z_i = \upsilon(x_i) + \ep_i, \; \ep_i \sim N(0, V_i)
\]
where $V_i$ is a function of the $i$th auxiliary variable and $z_i$ is some
pseudo-data.  Since the prior for $\upsilon$ is a Gaussian process one may use
conjugate formulas to sample the complete conditional of $\upsilon$; however,
producing a random variate from this distribution is expensive as one must
calculate the Cholesky decomposition of a relatively large matrix at each
iteration.  Consequently, the relative time spent sampling the auxiliary
variables in each model decreases, making the \Polya-Gamma method competitive,
and sometimes better, than the method of Fr\"{u}wirth-Schnatter et al.  We
provide one such example in Table (\ref{tab:nb-gp}).  A synthetic dataset of 256
equally spaced points were used to generate a draw $\upsilon(x_i)$ and $y_i$ for
$i=1, \ldots, 256$ where $\upsilon \sim GP(0, K)$ and $K$ has length scale $\ell
= 0.1$.  The average count value of the synthetic data set is $\bar y = 35.7$,
yielding 9137 total counts, which is roughly the same amount as in the larger
negative binomial example discussed earlier.  Whereas before the \Polya-Gamma
method lost when working with this number of total counts, it now wins.

\begin{table}
\label{tab:nb-synth}
\begin{tabular}{l r r r r r r r r }
\hline 
\multicolumn{9}{c}{Less Counts: $\alpha = 2$, $\bar y = 8.11$, $\sum y_i = 3244$} \\
\hline
Method  &     time &    ARate &  ESS.min &  ESS.med &  ESS.max &  ESR.min &  ESR.med &  ESR.max \\ 
    PG  &    26.84 &     1.00 &  7269.13 &  7646.16 &  8533.51 &   270.81 &   284.85 &   317.91 \\ 
    FS  &     8.10 &     1.00 &   697.38 &   719.36 &   759.13 &    86.10 &    88.80 &    93.70 \\ 
   RAM  &    10.17 &    30.08 &   737.95 &   748.51 &   758.57 &    72.59 &    73.62 &    74.61 \\
\hline
\multicolumn{9}{c}{More Counts: $\alpha = 3$, $\bar y = 23.98$, $\sum y_i = 9593$} \\
\hline
Method  &     time &    ARate &  ESS.min &  ESS.med &  ESS.max &  ESR.min &  ESR.med &  ESR.max \\ 
    PG  &    58.99 &     1.00 &  3088.04 &  3589.67 &  4377.21 &    52.35 &    60.85 &    74.20 \\ 
    FS  &     8.21 &     1.00 &   901.50 &   915.39 &   935.06 &   109.73 &   111.45 &   113.84 \\ 
   RAM  &     8.69 &    30.33 &   757.91 &   763.81 &   771.73 &    87.25 &    87.93 &    88.84
 \end{tabular}
 \caption{
   PG is the \Polya-Gamma Gibbs sampler.  FS follows
   \cite{fruhwirth-schnatter-etal-2009}.  RAM is the random walk
   Metropolis-Hastings sampler from the \texttt{bayesm} package. 
   $\alpha$ is the true intercept and $y_i$ is the $i$th response.  Each model
   has three continuous predictors.
 }
\end{table}

\begin{table}
\label{tab:nb-gp}
\begin{tabular}{l r r r r r r r r } 
\multicolumn{9}{c}{Gaussian Process: $\bar y = 35.7$, $\sum y_i = 9137$} \\
\hline
Method  &     time &    ARate &  ESS.min &  ESS.med &  ESS.max &  ESR.min &  ESR.med &  ESR.max \\ 
    PG  &   101.89 &     1.00 &   790.55 &  6308.65 &  9798.04 &     7.76 &    61.92 &    96.19 \\ 
    FS  &    53.17 &     1.00 &   481.36 &  1296.27 &  2257.27 &     9.05 &    24.38 &    42.45
 \end{tabular}
 \caption{PG is the \Polya-Gamma Gibbs sampler.  FS follows
   \cite{fruhwirth-schnatter-etal-2009}.  There are roughly as many total counts
   as in the larger example in Table \ref{tab:nb-synth}; however, the cost of
   drawing the posterior mean at the observed data points is much greater in
   this case, which reduces the penalty associated with sampling many
   \Polya-Gamma random variables.}
\end{table}


% If you have a bibliography.
% The file withe bibliography is name.bib.
\bibliographystyle{abbrvnat}
\bibliography{bayeslogit}{}

\end{document}

%%% Local Variables:
%%% TeX-master: t
%%% End: