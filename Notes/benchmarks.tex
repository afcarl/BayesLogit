\documentclass[11pt]{article}

\usepackage{amsmath}
\usepackage{amssymb}
\usepackage{amsthm}
\usepackage{fancyhdr}
\usepackage{makeidx}
\usepackage{url}
\usepackage{graphicx}
\newcommand{\one}{\bold{1}}
\newcommand{\ep}{\varepsilon}

%\input{commands}
\usepackage{natbib}
% \usepacakge{parskip}
\usepackage{outlines}

\newcommand{\Polya}{P\'{o}lya}
\newcommand{\PG}{\text{PG}}
\newcommand{\tablesize}{\small}

% Page Layout
\setlength{\evensidemargin}{0.0in}
\setlength{\oddsidemargin}{0.0in}
\setlength{\textwidth}{6.5in}
\setlength{\textheight}{8in}

% Set Fancy Style
%\pagestyle{fancy}
%\lhead{Left Header}
%\rhead{Right Header}

\begin{document}

\section{Benchmarks}

We benchmark the \Polya-Gamma method against several alternatives for binary
logistic regression and negative binomial regression for count data to measure
its relative performance.  All of these benchmarks are empirical and hence some
caution is urged.  Our primary metric of comparison is the effective sampling
rate, which is the effective sample size per second and which quantifies how
quickly a sampler can produce independent draws from the posterior distribution.
However, this metric is sensitive to numerous idiosyncrasies relating to the
implementation of the routines, the language in which they are written, and the
hardware on which they are run.  We generate these benchmarks using R, though
some of the routines make calls to external C code.  The specifics of each
method are discussed in further detail below.  In general, we find that the
\Polya-Gamma technique compares favorably to other data augmentation methods;
specifically, the \Polya-Gamma technique performs better than the methods of
\cite{obrien-dunson-2004}, \cite{gramacy-polson-2012}, and
\cite{fruhwirth-schnatter-fruhwirth-2010}. \cite{fruhwirth-schnatter-fruhwirth-2010}
provides a detailed comparison of several methods itself.  For instance, the
authors find that method of \cite{holmes-held-2006} did not beat their discrete
mixture of normals, something that we find as well and hence omit from the
comparisons below.  For the simple case of binary logistic regression we find
that often an independence Metropolis sampler performs well.  We draw the
following conclusions from these numerical experiments:

\begin{itemize}

\item In binary logistic regression, the \Polya-Gamma method performs better
  than other data augmentation techniques in terms of median effective sampling
  rate as seen in Appendix \ref{sec:blogit-benchmarks}.

\item In negative binomial regression, the \Polya-Gamma method performs well
  when the count sizes are not too high or when the regression coefficient has a
  complex prior structure, for instance, when the prior is a Gaussian process.

\item Independence Metropolis works well for binary logistic regression.  This
  is not surprising due to asymptotic Gaussianity.  Even when the \Polya-Gamma
  method has a superior effective sample size, it may be that the independence
  Metropolis sampler does better in terms of effective sampling rate since there
  is very little to compute at each iteration.  Nonetheless, there are times in
  which the \Polya-Gamma method is better in terms of effective sampling rate.

\item An important advantage of data augmentation, and the \Polya-Gamma
  technique in particular, is that it is easy to construct and code more
  complicated models, whereas independence Metropolis only works well in the
  simplest case.  For instance, the \Polya-Gamma method easily accommodates more
  sophisticated models such as mixed models and factor models.

\end{itemize}

\subsection{Benchmarking Procedure}

For each data set, we run 10 MCMC simulations with 12,000 samples each,
discarding the first 2,000 as burn in to leave a total of 10 batches of 10,000
samples.  The effective sample size for each regression coefficient is
calculated using the \texttt{coda} package and averaged across the 10 batches.
The component-wise minimum, median, and maximum of the (average) effective
sample sizes are reported to summarize the results.  A similar calculation is
performed to calculate minimum, median, and maximum effective sampling rates
(ESR).  The effective sampling rate is the ratio of of effective sample size to
the time taken to produce the sample.  Thus, the effective sampling rates are
normalized by the time taken to produce the 10,000 samples, disregarding the
time taken for initialization, preprocessing, and burn-in.  When discussing the
various methods the primary metric we refer to is the median effective sampling
rate, following the example of \cite{fruhwirth-schnatter-fruhwirth-2010}.

All of these experiments are carried out using R 2.15.1 on an Ubuntu machine
with 8GB or RAM and an Intel Core i5 quad core processor.  The number of cores
is a potentially important factor as some libraries, including those that
perform the matrix operations in R, may take advantage of multiple cores.  The C
code that we have written does not use parallelism.

\subsection{Binary Logistic Regression}

\subsubsection{Data Sets}

The benchmarks consist of data sets featured in \cite{holmes-held-2006} and
\cite{fruhwirth-schnatter-fruhwirth-2010}.  Details about the individual data
sets and the specific methods can be found in Appendix
\ref{sec:blogit-datasets}.

\subsubsection{Results}

As seen in Appendix \ref{sec:blogit-benchmarks} the \Polya-Gamma method beats
all other data augmentation techniques for binary logistic regression in terms
of both effective sample size and effective sampling rate.  The \Polya-Gamma
method almost always beats the independence Metropolis samplers in terms of
effective sample size; however, it does not beat the independence Metropolis
samplers in terms of effective sampling rate.  Independence Metropolis samplers
perform well since (1) proposals are cheap, a consequence of only performing
large computations in the preprocessing stage, and (2) elliptic proposal
distributions are close to the posterior distribution given enough data, a
consequence of the central limit theorem for exponential families.

However, a major advantage of data augmentation, and hence the \Polya-Gamma
technique, is that it is easily adapted to more complicated models.  Consider,
for instance, a binary logistic mixed model whose intercepts are random effects,
in which case the log odds for observation $j$ from group $i$, $\psi_{ij}$, is
modeled by:
\begin{equation}
\label{eqn:mixed-model}
\begin{cases}
\psi_{ij} = \alpha_i + x_{ij} \beta \\
\alpha_i \sim N(m, 1/\phi) \\
m \sim N(0, \kappa^2 / \phi) \\
\phi \sim Ga(1,1) \\
\beta \sim N(0, 100 I).
\end{cases}
\end{equation}
An extra step is easily added to the \Polya-Gamma Gibbs sampler to estimate
$(\alpha, \beta, m)$ and $\phi$.  It is not clear that an independence
Metropolis sampler is easily constructed.  The simplest possible route is to
write down the new log-posterior and use the subsequent Laplace approximation.
However, as seen in Table \ref{tab:mm-examples}, this procedure does not work
well for medium or large problems, presumably due to the fact that the proposal
poorly approximates the posterior.  The results above make use of a Gaussian
proposal; a $t_6$ performs just as poorly.  To improve the abysmal acceptance
rate, one might consider implementing a Metropolis step within Gibbs to sample
$(\alpha,\beta,m)$, but this would require re-estimating the posterior mode and
variance of $(\alpha,\beta,m)$ at each step for a given $\phi$, a time consuming
process.  Without prejudicing the other possible routes to finding a better
Metropolis-Hastings based sampler, such as \cite{gamerman-1997}, we simply point
out that what works well in the simplest case does not necessarily work well in
a slightly more complicated case; however, the \Polya-Gamma method works equally
well in both cases.  This discrepancy would be exacerbated further were we to
consider a yet more complicated model, such as a factor model for binary data.
In such a model one must estimate the factor loadings, the factor variances, and
the idiosyncratic variances.  A \Polya-Gamma Gibbs sampler easily incorporates
complete conditional samples of these quantities, while independence Metropolis
sampling is essentially hopeless.  Thus, while the Polya-Gamma technique may be
the runner-up in the simplest scenario, it appears to be globally the best
choice, performing well for both simple and complicated models.  A discussion of
why the independence Metropolis sampler does not work may be found in Appendix
\ref{sec:logitmm-discussion}.

\begin{table}
\tablesize
\centering
\label{tab:mm-examples}
\begin{tabular}{l r r r r r r r r } 
\hline
\multicolumn{9}{c}{Synthetic: $N=500$, $P_a=5$, $P_b=1$, samp=10,000, burn=2,000, thin=1} \\
\hline
Method   &     time &    ARate &  ESS.min &  ESS.med &  ESS.max &  ESR.min &  ESR.med &  ESR.max \\ 
PG       &     7.29 &     1.00 &  4289.29 &  6975.73 &  9651.69 &   588.55 &   957.18 &  1324.31 \\ 
Ind-Met. &     3.96 &     0.70 &  1904.71 &  3675.02 &  4043.42 &   482.54 &   928.65 &  1022.38 \\
%\end{tabular}

%\begin{tabular}{l r r r r r r r r } 
\hline
\multicolumn{9}{c}{Polls: $N=2015$, $P_a=49$, $P_b=1$, samp=100,000, burn=20,000, thin=10} \\
\hline
Method   &     time &    ARate &  ESS.min &  ESS.med &  ESS.max &  ESR.min &  ESR.med &  ESR.max \\ 
PG       &    31.94 &     1.00 &  5948.62 &  9194.42 &  9925.73 &   186.25 &   287.86 &   310.75 \\ 
Ind-Met. &   146.76 &  0.00674 &    31.36 &    52.81 &    86.54 &     0.21 &     0.36 &     0.59 \\
% \end{tabular}

% df = 6 ppsl for polls
% \begin{tabular}{l r r r r r r r r } 
%           Method  &     time &    ARate &  ESS.min &  ESS.med &  ESS.max &  ESR.min &  ESR.med &  ESR.max \\ 
%               PG  &    31.80 &     1.00 &  5886.46 &  9301.77 &  9982.79 &   185.13 &   292.55 &   313.97 \\ 
%           IndMM4  &   147.17 & 0.004617 &    33.78 &    69.14 &   121.47 &     0.23 &     0.47 &     0.83
%  \end{tabular}

% \begin{tabular}{l r r r r r r r r } 
\hline
\multicolumn{9}{c}{Xerop: $N=1200$, $P_a=275$, $P_b=8$, samp=100,000, burn=20,000, thin=10} \\
\hline
Method   &     time &     ARate &  ESS.min &  ESS.med &  ESS.max &  ESR.min &  ESR.med &  ESR.max \\ 
PG       &   174.38 &      1.00 &   850.34 &  3038.76 &  4438.99 &     4.88 &    17.43 &    25.46 \\ 
Ind-Met. &   457.86 & 0.00002.5 &     1.85 &     3.21 &    12.32 &     0.00 &     0.01 &     0.03
\end{tabular}

% df = 6 ppsl for xerop
% \begin{tabular}{l r r r r r r r r } 
%           Method  &     time &    ARate &  ESS.min &  ESS.med &  ESS.max &  ESR.min &  ESR.med &  ESR.max \\ 
%            PGMM2  &   174.36 &     1.00 &   863.39 &  2924.02 &  4337.20 &     4.95 &    16.77 &    24.88 \\ 
%           IndMM4  &   461.92 &  2.5e-05 &     2.22 &     4.01 &     9.38 &     0.00 &     0.01 &     0.02
%  \end{tabular}

\caption{A set of three benchmarks for binary logistic mixed models.  $N$
  denotes the number of samples, $P_a$ denotes the number of groups, and $P_b$
  denotes the dimension of the fixed effects coefficient.  The random effects
  are limited to group dependent intercepts.  Notice that the second and third
  benchmarks are thinned every 10 samples to produce a total of 10,000 posterior
  draws.  Even after thinning, the effective sample size for each is low compared
  to the PG method.  The effective samples sizes are taken for the collection
  $(\alpha, \beta, m)$ and do not include $\phi$.}

\end{table}


% \begin{table}
% \begin{tabular}{l r r r r r r r r } 
%  Method  &     time &    ARate &  ESS.min &  ESS.med &  ESS.max &  ESR.min &  ESR.med &  ESR.max \\ 
%    PGMM  &     6.82 &     1.00 &  4297.65 &  6599.02 &  7052.02 &   630.41 &   968.01 &  1034.42 \\ 
% IndMHMM  &     3.53 &     0.42 &  1256.11 &  1501.59 &  1694.51 &   356.27 &   425.35 &   480.10 \\ 
%      PG  &     5.70 &     1.00 &  4478.19 &  6863.88 &  7367.59 &   785.55 &  1204.03 &  1292.48 \\ 
%   IndMH  &     2.25 &     0.91 &  6646.28 &  7765.85 &  7938.66 &  2949.70 &  3445.48 &  3523.12
%  \end{tabular}
%  \caption{
%    The effective sample sizes found for a synthetic data set with 
%    5 groups, corresponding to 5 different intercepts drawn from the same
%    distribution $N(0, 1/\phi), \phi=1$, 100 observations within each group (for a total of
%    500 observations), and 1 shared covariate.  The covariate was drawn from
%    $N(0,1)$.  PGMM is the \Polya-Gamma Gibbs sampler for the mixed model.  IndMHMM
%    is the independence Metropolis sampler for the mixed model using a normal
%    proposal based upon the Laplace approximation.  We tried using a $t_6$ proposal
%    as well, but that did not improve the results.  The last two rows correspond to
%    the case when $\phi$ is known, which returns us to binomial regression with a
%    $N(0, I)$ prior $\alpha$.  In the mixed models we used a $\text{Ga}(1,
%    \text{rate}=1)$ prior for $\phi$.  Notice that the effective sampling rate,
%    which is taken over the components of $(\alpha, \beta)$, is
%    excellent in the binomial regression case, but drops precipitously for
%    unknown $\phi$.
%  }
% \end{table}

\subsection{Negative Binomial Regression}

The \Polya-Gamma method shows excellent effective sample sizes in negative
binomial regression.  However, its effective sampling rates fare less well when
working with a large numbers of counts.  Currently, our \Polya-Gamma sampler can
draw from $\PG(n, \psi)$, $n=1$ quickly.  But we have not constructed a fast
sampler for general $n$.  To sample from $\PG(n, \psi)$ for integral $n > 1$ we
take $n$ independent samples of $\PG(1, \psi)$ and sum them.  Thus, in the case
of negative binomial regression one must sample roughly $\sum_{i=1}^N y_i$
\Polya-Gamma random varieties, where $y_i$ is the response, at every MCMC
iteration.  When the number of counts is relatively high this becomes a burden
and the \Polya-Gamma sampler performs poorly.  We see this in Table
(\ref{tab:nb-synth}), where the \Polya-Gamma sampler does well when working with
relatively small count sizes, but poorly when that is not the case.  In both
synthetic data sets, $N=400$ data points are created using the model
\[
\begin{cases}
y_i \sim NB(\textmd{mean}=\mu_i, d) \\
\log \mu_i = \alpha + x_i \beta
\end{cases}
\]
where $\beta \in \mathbb{R}^3$.  The parameter $d$ is estimated using a
random-walk Metropolis-Hastings step.  The model with fewer counts corresponds
to $\alpha = 2$ while the model with more counts corresponds to $\alpha = 3$,
which produced a sample mean of roughly 8 in the former and 24 in the latter.
Notice that the \Polya-Gamma method has superior effective sample size in both
cases.  It is an open challenge to create an efficient \Polya-Gamma sampler for
arbitrary $n$, which would make it the best choice in both cases.

The \Polya-Gamma method starts to shine when working with more complicated
models that devote proportionally less time to sampling the auxiliary variables.
For instance, consider the model
\[
\begin{cases}
y_i \sim NB(\text{mean} = \mu(x_i), d) \\
\log \mu  \sim GP(0, K)
\end{cases}
\]
where $K$ is the square exponential covariance kernel,
\[
K(x_1,x_2) = \kappa + \exp \Big( \frac{\|x_1 - x_2\|^2}{2 \ell^2} \Big),
\]
with characteristic length scale $\ell$ and nugget $\kappa$.  Using either the
\Polya-Gamma or \cite{fruhwirth-schnatter-etal-2009} data augmentation
techniques, one arrives at a complete conditional for $\upsilon = \log \mu$ that
is equivalent to the posterior $(\upsilon | z)$ derived using pseudo-data
$\{z_i\}$ generated by
\[
z_i = \upsilon(x_i) + \ep_i, \; \ep_i \sim N(0, V_i)
\]
where $V_i$ is a function of the $i$th auxiliary variable.  Since the prior for
$\upsilon$ is a Gaussian process one may use conjugate formulas to sample the
complete conditional of $\upsilon$; however, producing a random variate from
this distribution is expensive as one must calculate the Cholesky decomposition
of a relatively large matrix at each iteration.  Consequently, the relative time
spent sampling the auxiliary variables in each model decreases, making the
\Polya-Gamma method competitive, and sometimes better, than the method of
Fr\"{u}hwirth-Schnatter et al.  We provide two such examples in Table
(\ref{tab:nb-gp-ex01}).  In the first synthetic data set, 256 equally spaced
points were used to generate a draw $\upsilon(x_i)$ and $y_i$ for $i=1, \ldots,
256$ where $\upsilon \sim GP(0, K)$ and $K$ has length scale $\ell = 0.1$ and a
nugget $=0.0$.  The average count value of the synthetic data set is $\bar y =
35.7$, yielding 9137 total counts, which is roughly the same amount as in the
larger negative binomial example discussed earlier.  Whereas before the
\Polya-Gamma method lost when working with this number of total counts, it now
wins.  In the second synthetic data set, 1000 randomly selected points were
chosen to generate a draw from $\upsilon(x_i)$ and $y_i$ with $\upsilon \sim
GP(0, K)$ where $K$ has length scale $\ell = 0.1$ and a nugget $=0.0001$.  The
average count value is $\bar y = 22.72$, yielding $22,720$ total counts.  The
larger problem shows an even greater improvement in performance over the method
of Fr\"{u}hwirth-Schnatter et al.

\begin{table}
\tablesize
\centering
\label{tab:nb-synth}
\begin{tabular}{l r r r r r r r r }
\hline 
\multicolumn{9}{c}{Less Counts: $\alpha = 2$, $\bar y = 8.11$, $\sum y_i = 3244$, $N=400$} \\
\hline
Method  &     time &    ARate &  ESS.min &  ESS.med &  ESS.max &  ESR.min &  ESR.med &  ESR.max \\ 
    PG  &    26.84 &     1.00 &  7269.13 &  7646.16 &  8533.51 &   270.81 &   284.85 &   317.91 \\ 
    FS  &     8.10 &     1.00 &   697.38 &   719.36 &   759.13 &    86.10 &    88.80 &    93.70 \\ 
   RAM  &    10.17 &    30.08 &   737.95 &   748.51 &   758.57 &    72.59 &    73.62 &    74.61 \\
\hline
\multicolumn{9}{c}{More Counts: $\alpha = 3$, $\bar y = 23.98$, $\sum y_i = 9593$, $N=400$} \\
\hline
Method  &     time &    ARate &  ESS.min &  ESS.med &  ESS.max &  ESR.min &  ESR.med &  ESR.max \\ 
    PG  &    58.99 &     1.00 &  3088.04 &  3589.67 &  4377.21 &    52.35 &    60.85 &    74.20 \\ 
    FS  &     8.21 &     1.00 &   901.50 &   915.39 &   935.06 &   109.73 &   111.45 &   113.84 \\ 
   RAM  &     8.69 &    30.33 &   757.91 &   763.81 &   771.73 &    87.25 &    87.93 &    88.84
 \end{tabular}
 \caption{
   Negative binomial regression.  PG is the \Polya-Gamma Gibbs sampler.  FS follows
   \cite{fruhwirth-schnatter-etal-2009}.  RAM is the random walk
   Metropolis-Hastings sampler from the \texttt{bayesm} package. 
   $\alpha$ is the true intercept and $y_i$ is the $i$th response.  Each model
   has three continuous predictors.
 }
\end{table}

\begin{table}
\tablesize
\centering
\label{tab:nb-gp-ex01}
\begin{tabular}{l r r r r r r r r } 
\hline
\multicolumn{9}{c}{Gaussian Process: $\bar y = 35.7$, $\sum y_i = 9137$, $N=256$, $\ell=0.1$, nugget=$0.0$} \\
\hline
Method  &     time &    ARate &  ESS.min &  ESS.med &  ESS.max &  ESR.min &  ESR.med &  ESR.max \\ 
    PG  &   101.89 &     1.00 &   790.55 &  6308.65 &  9798.04 &     7.76 &    61.92 &    96.19 \\ 
    FS  &    53.17 &     1.00 &   481.36 &  1296.27 &  2257.27 &     9.05 &    24.38 &    42.45 \\

% \label{tab:nb-gp-ex02}
% \begin{tabular}
\hline
\multicolumn{9}{c}{Gaussian Process: $\bar y = 22.7$, $\sum y_i = 22732$, $N=1000$, $\ell=0.1$, nugget=$0.0001$} \\
\hline
Method  &     time &    ARate &  ESS.min &  ESS.med &  ESS.max &  ESR.min &  ESR.med &  ESR.max \\ 
    PG  &  2021.78 &     1.00 &  1966.77 &  6386.43 &  9862.54 &     0.97 &     3.16 &     4.88 \\ 
    FS  &  1867.05 &     1.00 &   270.13 &  1156.52 &  1761.70 &     0.14 &     0.62 &     0.94
% \end{tabular}

 \end{tabular}
 \caption{Non-parametric negative binomial regression.  PG is the \Polya-Gamma Gibbs sampler.  FS follows
   \cite{fruhwirth-schnatter-etal-2009}.  There are roughly as many total counts
   as in the first table as their are in the larger example in Table \ref{tab:nb-synth}; however, the cost of
   drawing the posterior mean at the observed data points is much greater in
   this case, which reduces the penalty associated with sampling many
   \Polya-Gamma random variables.  The second table shows that the cost drawing
   the posterior mean is even more pronounced for larger problems.  $N$ is the
   total number of observations and $y_i$ denotes the $i$th observation.}
\end{table}

\appendix

\section{Binary Logistic Regression: Data Sets and Methods}
\label{sec:blogit-datasets}

\subsection{Data Sets}

\begin{outline}

\1 Nodal (part of the \texttt{boot} R package) :

The response indicates if cancer has spread from the prostate to surrounding
lymph nodes.  There are 53 observations and 5 binary predictors.

\1 Pima Indian:

\url{http://archive.ics.uci.edu/ml/datasets/Pima+Indians+Diabetes}

There are 768 observations and 8 continuous predictors.  It is noted on the UCI
website that there are many predictor values coded as 0, though the physical
measurement should be non-zero.  We have removed all of those entries to
generate a data set with 392 observations.  The marginal mean incidence of
diabetes is roughly 0.33 before and after removing the data.

\1 Heart:

\url{http://archive.ics.uci.edu/ml/datasets/Statlog+(Heart)}

The response represents either an absence or presence of heart disease.  There
are 270 observations and 13 attributes of which 6 are categorical or binary and
1 is ordinal.  The ordinal covariate has been stratified by dummy variables.

\1 Australian Credit: 

\url{http://archive.ics.uci.edu/ml/datasets/Statlog+(Australian+Credit+Approval)}.

The response represents either accepting or rejecting a credit card application.
The meaning of each predictor has been removed to protect the propriety of the
original data.  There are 690 observations and 14 attributes, of which 8 are
categorical or binary.  There were 37 observations with missing attribute
values.  These missing values have been replaced by the mode of the attribute in
the case of categorical data and the mean of the attribute for continuous data.
This dataset is linearly separable and results in some divergent regression
coefficients, which are kept in check by the prior.

\1 German Credit: 

\url{http://archive.ics.uci.edu/ml/datasets/Statlog+(German+Credit+Data)}

The response represents either a good or bad credit risk.  There are 1000
observations and 20 attributes including both continuous and categorical data.
We benchmark two scenarios.  In the first, the ordinal covariates have been
given integer values and have not been stratified by dummy variable, yielding a
total of 24 numeric predictors.  In the second, the ordinal data has been
stratified by dummy variables, yielding a total of 48 predictors.

\end{outline}

\subsection{Methods}

All of these routines are implemented in R, though some of them make calls to C.
In particular, the independence Metropolis samplers do not make use of any
non-standard calls to C, though their implementations have very little R
overhead in terms of function calls; the \Polya-Gamma method calls a C routine
to sample the \Polya-Gamma random variates, but otherwise only uses R.  We think
this is fair since other basic random variate generators in R call compiled
code.  As a check upon our independence Metropolis sampler we include the
independence Metropolis sampler of \cite{rossi-etal-2005}, which may be found in
the \texttt{bayesm} package \citep{bayesm-2012}, though their sampler uses a
$t_6$ proposal while ours uses a normal proposal.  The suite of routines in the
\texttt{binomlogit} package \citep{binomlogit-2012} implement the techniques
discussed in \cite{fussl-etal-2011-slides}.  One routine provided by the
\texttt{binomlogit} package coincides with the technique described in
\cite{fruhwirth-schnatter-fruhwirth-2010} for the case of binary logistic
regression.  A separate routine implements the latter and uses a single call to
C.  Gramacy and Polson's R package, \texttt{reglogit}, also calls external C
code \citep{reglogit-2012}.  For every data set the regression coefficient was
given a diffuse $N(0, 0.01 I)$ prior, except when using Gramacy and Polson's
method, in which case it was given a $\exp(\sum_{i} |\beta_i / 100|)$ prior per
the specifications of the \texttt{reglogit} package.  The following is a short
description of each method along with its abbreviated name.

\begin{outline}

\1 PG: The \Polya-Gamma method described previously.

\1 FS: \cite{fruhwirth-schnatter-fruhwirth-2010} follow \cite{holmes-held-2006}
and use the \cite{albert-chib-1993} like representation,
\begin{equation}
\label{eqn:hh-rep}
\begin{cases}
y_i = \one \{ z_i > 0 \} \\
z_i = x_i \beta + \ep_i, & \ep_i \sim \text{Lo}
\end{cases}
\end{equation}
where $\text{Lo}$ is the standard logistic distribution.  They approximate the
distribution of $\ep_i$ using a discrete mixture of normals.

\1 IndMH: Independence Metropolis with a normal proposal.  Calculate the
posterior mode and the Hessian at the mode to pick the mean and variance of the
proposal.

\1 RAM: Independence Metropolis with a $t_6$ proposal from the R package
\texttt{bayesm}.  Calculate the posterior mode and the Hessian at the mode to
pick the mean and scale matrix of the proposal.

\1 OD: The method of \cite{obrien-dunson-2004}.  Strictly speaking, this is not
logistic regression; it is binary regression using a student's T cumulative
distribution function as the inverse link function.

\1 dRUMAuxMix: Work by \cite{fussl-etal-2013} that extends the technique of
\cite{fruhwirth-schnatter-fruhwirth-2010}.  A convenient representation is found
that relies on a discrete mixture of normals approximation for posterior
inference that works for binomial logistic regression.  From the R package
\texttt{binomlogit}.

\1 dRUMIndMH: Similar to dRUMAuxMix, but instead of using a discrete mixture of
normals, use a single normal to approximate the error term and correct using
Metropolis-Hastings.  From the R package \texttt{binomlogit}.

\1 IndivdRUMIndMH: This is the same as dRUMIndMH, but specific to binary
logistic regression.  From the R package \texttt{binomlogit}.

\1 dRUMHAM: Identical to dRUMAuxMix, but now use a discrete mixture of normals
approximation in which the number of components to mix over is determined by
$y_i / n_i$.  From the R package \texttt{binomlogit}.

\1 GP: \cite{gramacy-polson-2012}: Another data augmentation scheme with only a
single layer of latents.  This routine uses a double exponential prior.  We set
the scale of this prior to agree with the scale of the normal prior we used in
all other cases above.  Below you will notice that the \Polya-Gamma method often
has a better sample size than Gramacy and Polson, but just barely.  We used the
\texttt{reglogit} package for these experiment, which is currently sluggish.  If
the runtime of this routine were to improve significantly, then it could perform
almost as well as the \Polya-Gamma method.

\end{outline}

\section{Binary Logistic Regression Benchmarks}
\label{sec:blogit-benchmarks}

% 
\begin{table}
\begin{tabular}{l r r r r r r r r } 
          Method  &     time &    ARate &  ESS.min &  ESS.med &  ESS.max &  ESR.min &  ESR.med &  ESR.max \\ 
              PG  &     2.98 &     1.00 &  3225.11 &  4818.45 &  5576.17 &  1083.93 &  1619.18 &  1873.92 \\ 
              FS  &     3.50 &     1.00 &   981.18 &  1627.18 &  1909.64 &   280.68 &   465.31 &   546.03 \\ 
           IndMH  &     1.77 &     0.67 &  1401.62 &  1553.02 &  1778.83 &   792.07 &   876.16 &  1003.38 \\ 
             RAM  &     1.29 &     0.64 &  3181.52 &  3592.19 &  4015.63 &  2475.73 &  2795.07 &  3124.86 \\ 
              OD  &     3.73 &     1.00 &  1284.37 &  2000.39 &  2274.60 &   344.03 &   535.80 &   609.07 \\ 
       dRUMIndMH  &     1.41 &     0.62 &   739.57 &  1145.32 &  1335.26 &   525.26 &   813.66 &   948.17 \\ 
         dRUMHAM  &     3.06 &     1.00 &   949.94 &  1561.52 &  1889.99 &   310.73 &   511.05 &   618.50 \\ 
      dRUMAuxMix  &     2.71 &     1.00 &   986.44 &  1575.06 &  1833.59 &   364.00 &   581.35 &   676.73 \\ 
  IndivdRUMIndMH  &     1.30 &     0.61 &   708.20 &  1108.73 &  1358.56 &   544.31 &   852.50 &  1044.73 \\ 
              GP  &    17.93 &     1.00 &  2907.02 &  4416.26 &  5191.32 &   162.13 &   246.30 &   289.53
 \end{tabular}
\caption{table.nodal, N=53, P=6}
\end{table}

\begin{table}
\begin{tabular}{l r r r r r r r r } 
          Method  &     time &    ARate &  ESS.min &  ESS.med &  ESS.max &  ESR.min &  ESR.med &  ESR.max \\ 
              PG  &     5.65 &     1.00 &  3435.04 &  5541.31 &  6562.52 &   608.05 &   981.02 &  1161.74 \\ 
              FS  &     6.63 &     1.00 &  1126.82 &  1986.91 &  2467.99 &   169.98 &   299.77 &   372.36 \\ 
           IndMH  &     2.21 &     0.81 &  4306.89 &  5373.10 &  5608.30 &  1948.23 &  2430.63 &  2536.62 \\ 
             RAM  &     1.94 &     0.68 &  4762.30 &  4905.60 &  5068.22 &  2455.39 &  2530.44 &  2613.72 \\ 
              OD  &     6.38 &     1.00 &  1431.69 &  2378.93 &  2940.13 &   224.33 &   372.78 &   460.73 \\ 
       dRUMIndMH  &     3.83 &     0.48 &   670.14 &  1094.98 &  1347.73 &   175.22 &   286.26 &   352.36 \\ 
         dRUMHAM  &     6.97 &     1.00 &  1094.53 &  2005.24 &  2422.09 &   157.13 &   287.86 &   347.71 \\ 
      dRUMAuxMix  &     6.07 &     1.00 &  1085.33 &  1975.53 &  2466.47 &   178.82 &   325.48 &   406.40 \\ 
  IndivdRUMIndMH  &     2.93 &     0.49 &   645.00 &  1138.97 &  1338.05 &   220.52 &   389.42 &   457.47 \\ 
              GP  &    88.93 &     1.00 &  2902.00 &  5477.16 &  9426.37 &    32.63 &    61.59 &   106.00
 \end{tabular}
\caption{table.diabetes, N=392, P=9}
\end{table}

\begin{table}
\begin{tabular}{l r r r r r r r r } 
          Method  &     time &    ARate &  ESS.min &  ESS.med &  ESS.max &  ESR.min &  ESR.med &  ESR.max \\ 
              PG  &     5.62 &     1.00 &  2168.93 &  3577.10 &  4807.49 &   386.04 &   636.74 &   855.69 \\ 
              FS  &     6.55 &     1.00 &   584.34 &  1107.43 &  1501.28 &    89.28 &   169.20 &   229.35 \\ 
           IndMH  &     2.25 &     0.39 &   594.56 &   761.33 &   974.43 &   264.46 &   338.34 &   432.34 \\ 
             RAM  &     1.99 &     0.31 &   954.10 &  1104.96 &  1317.57 &   478.27 &   554.01 &   660.89 \\ 
              OD  &     6.43 &     1.00 &   847.08 &  1508.32 &  2027.58 &   131.70 &   234.54 &   315.28 \\ 
       dRUMIndMH  &     3.51 &     0.34 &   280.19 &   466.01 &   599.03 &    79.83 &   132.78 &   170.60 \\ 
         dRUMHAM  &     7.08 &     1.00 &   621.26 &  1077.45 &  1535.17 &    87.78 &   152.25 &   216.93 \\ 
      dRUMAuxMix  &     5.99 &     1.00 &   570.17 &  1113.10 &  1502.19 &    95.12 &   185.73 &   250.70 \\ 
  IndivdRUMIndMH  &     2.90 &     0.35 &   282.86 &   474.98 &   638.74 &    97.62 &   163.95 &   220.41 \\ 
              GP  &    66.12 &     1.00 &  1415.32 &  2992.82 &  5479.96 &    21.41 &    45.27 &    82.88
 \end{tabular}
\caption{table.heart, N=270, P=19}
\end{table}

\begin{table}
\begin{tabular}{l r r r r r r r r } 
          Method  &     time &    ARate &  ESS.min &  ESS.med &  ESS.max &  ESR.min &  ESR.med &  ESR.max \\ 
              PG  &    12.87 &     1.00 &   416.65 &  4058.26 &  5314.51 &    32.37 &   315.33 &   412.93 \\ 
              FS  &    15.08 &     1.00 &    28.33 &  1001.75 &  1630.79 &     1.88 &    66.42 &   108.13 \\ 
           IndMH  &     3.45 &     0.20 &   206.02 &   326.38 &   380.30 &    59.58 &    94.52 &   110.31 \\ 
             RAM  &     3.93 &     0.00 &     9.46 &    12.06 &    18.14 &     2.41 &     3.06 &     4.60 \\ 
              OD  &    14.32 &     1.00 &    42.14 &   849.50 &  2716.87 &     2.94 &    59.34 &   189.76 \\ 
       dRUMIndMH  &     8.95 &     0.19 &    12.84 &   217.33 &   420.04 &     1.44 &    24.28 &    46.93 \\ 
         dRUMHAM  &    18.64 &     1.00 &    31.75 &  1011.46 &  1644.21 &     1.70 &    54.26 &    88.21 \\ 
      dRUMAuxMix  &    14.98 &     1.00 &    31.31 &  1013.37 &  1806.51 &     2.09 &    67.65 &   120.60 \\ 
  IndivdRUMIndMH  &     7.38 &     0.19 &    17.74 &   227.35 &   424.72 &     2.40 &    30.82 &    57.58 \\ 
              GP  &   164.11 &     1.00 &   625.81 &  2838.17 & 10000.00 &     3.81 &    17.29 &    60.94
 \end{tabular}
\caption{table.aus, N=690, P=35}
\end{table}

\begin{table}
\begin{tabular}{l r r r r r r r r } 
          Method  &     time &    ARate &  ESS.min &  ESS.med &  ESS.max &  ESR.min &  ESR.med &  ESR.max \\ 
              PG  &    15.39 &     1.00 &  3172.07 &  5849.99 &  6579.61 &   206.06 &   380.03 &   427.42 \\ 
              FS  &    18.19 &     1.00 &   988.29 &  2157.09 &  2496.71 &    54.34 &   118.60 &   137.28 \\ 
           IndMH  &     3.59 &     0.68 &  2351.19 &  3477.51 &  3825.86 &   656.16 &   969.25 &  1066.72 \\ 
             RAM  &     4.16 &     0.43 &  2037.25 &  2309.96 &  2454.55 &   490.07 &   555.66 &   590.49 \\ 
              OD  &    16.98 &     1.00 &  1288.67 &  2764.09 &  3269.41 &    75.90 &   162.80 &   192.55 \\ 
       dRUMIndMH  &    10.60 &     0.28 &   328.91 &   696.98 &   788.55 &    31.04 &    65.78 &    74.43 \\ 
         dRUMHAM  &    22.06 &     1.00 &   939.93 &  2152.78 &  2568.08 &    42.61 &    97.59 &   116.41 \\ 
      dRUMAuxMix  &    18.15 &     1.00 &   980.68 &  2141.21 &  2489.43 &    54.02 &   117.96 &   137.14 \\ 
  IndivdRUMIndMH  &     8.35 &     0.28 &   326.83 &   677.68 &   790.58 &    39.15 &    81.16 &    94.69 \\ 
              GP  &   225.57 &     1.00 &  2590.78 &  5502.83 &  7971.67 &    11.49 &    24.40 &    35.34
 \end{tabular}
\caption{table.ger.num, N=1000, P=25}
\end{table}

\begin{table}
\begin{tabular}{l r r r r r r r r } 
          Method  &     time &    ARate &  ESS.min &  ESS.med &  ESS.max &  ESR.min &  ESR.med &  ESR.max \\ 
              PG  &    22.20 &     1.00 &  2769.10 &  5730.47 &  6677.46 &   124.72 &   258.09 &   300.74 \\ 
              FS  &    26.40 &     1.00 &   751.99 &  2069.84 &  2552.73 &    28.49 &    78.41 &    96.71 \\ 
           IndMH  &     4.74 &     0.41 &   714.95 &  1129.58 &  1325.46 &   150.34 &   238.06 &   279.43 \\ 
             RAM  &     6.04 &     0.00 &     5.73 &    13.51 &    71.85 &     0.95 &     2.24 &    11.90 \\ 
              OD  &    24.95 &     1.00 &  1265.06 &  2767.02 &  3307.83 &    50.70 &   110.90 &   132.56 \\ 
       dRUMIndMH  &    14.66 &     0.13 &   125.85 &   293.00 &   344.66 &     8.58 &    19.99 &    23.51 \\ 
         dRUMHAM  &    35.76 &     1.00 &   747.79 &  2097.09 &  2577.80 &    20.90 &    58.65 &    72.09 \\ 
      dRUMAuxMix  &    26.83 &     1.00 &   733.42 &  2108.81 &  2552.06 &    27.34 &    78.61 &    95.13 \\ 
  IndivdRUMIndMH  &    12.41 &     0.13 &   141.81 &   290.21 &   355.47 &    11.42 &    23.38 &    28.64 \\ 
              GP  &   245.07 &     1.00 &  1991.06 &  5322.27 & 10000.00 &     8.12 &    21.72 &    40.80
 \end{tabular}
\caption{table.ger, N=1000, P=49}
\end{table}

%%% Local Variables:
%%% TeX-master: benchmarks.tex
%%% End:

%-------------------------------------------------------------------------------
\begin{table}
\label{tab:blogit-nodal}
\begin{tabular}{l r r r r r r r r } 
          Method  &     time &    ARate &  ESS.min &  ESS.med &  ESS.max &  ESR.min &  ESR.med &  ESR.max \\ 
              PG  &     2.98 &     1.00 &  3221.12 &  4859.89 &  5571.76 &  1081.55 &  1631.96 &  1871.00 \\ 
           IndMH  &     1.76 &     0.66 &  1070.23 &  1401.89 &  1799.02 &   610.19 &   794.93 &  1024.56 \\ 
             RAM  &     1.29 &     0.64 &  3127.79 &  3609.31 &  3993.75 &  2422.49 &  2794.69 &  3090.05 \\ 
              OD  &     3.95 &     1.00 &   975.36 &  1644.66 &  1868.93 &   246.58 &   415.80 &   472.48 \\ 
              FS  &     3.49 &     1.00 &   979.56 &  1575.06 &  1902.24 &   280.38 &   450.67 &   544.38 \\ 
      dRUMAuxMix  &     2.69 &     1.00 &  1015.18 &  1613.45 &  1912.78 &   376.98 &   598.94 &   710.30 \\ 
       dRUMIndMH  &     1.41 &     0.62 &   693.34 &  1058.95 &  1330.14 &   492.45 &   751.28 &   943.66 \\ 
  IndivdRUMIndMH  &     1.30 &     0.61 &   671.76 &  1148.61 &  1339.58 &   518.79 &   886.78 &  1034.49 \\ 
         dRUMHAM  &     3.06 &     1.00 &   968.41 &  1563.88 &  1903.00 &   316.82 &   511.63 &   622.75 \\ 
              GP  &    17.86 &     1.00 &  2821.49 &  4419.37 &  5395.29 &   157.93 &   247.38 &   302.00
 \end{tabular}
\caption{table.nodal, N=53, P=6}
\end{table}

%-------------------------------------------------------------------------------
\begin{table}
\label{tab:blogit-diabetes}
\begin{tabular}{l r r r r r r r r } 
          Method  &     time &    ARate &  ESS.min &  ESS.med &  ESS.max &  ESR.min &  ESR.med &  ESR.max \\ 
              PG  &     5.65 &     1.00 &  3255.25 &  5444.79 &  6437.16 &   576.14 &   963.65 &  1139.24 \\ 
           IndMH  &     2.21 &     0.81 &  3890.09 &  5245.16 &  5672.83 &  1759.54 &  2371.27 &  2562.59 \\ 
             RAM  &     1.93 &     0.68 &  4751.95 &  4881.63 &  5072.02 &  2456.33 &  2523.85 &  2621.98 \\ 
              OD  &     6.63 &     1.00 &  1188.00 &  2070.56 &  2541.70 &   179.27 &   312.39 &   383.49 \\ 
              FS  &     6.61 &     1.00 &  1087.40 &  1969.22 &  2428.81 &   164.39 &   297.72 &   367.18 \\ 
      dRUMAuxMix  &     6.05 &     1.00 &  1158.42 &  1998.06 &  2445.66 &   191.52 &   330.39 &   404.34 \\ 
       dRUMIndMH  &     3.82 &     0.49 &   647.20 &  1138.03 &  1338.73 &   169.41 &   297.98 &   350.43 \\ 
  IndivdRUMIndMH  &     2.91 &     0.48 &   614.57 &  1111.60 &  1281.51 &   211.33 &   382.23 &   440.63 \\ 
         dRUMHAM  &     6.98 &     1.00 &  1101.71 &  1953.60 &  2366.54 &   157.89 &   280.01 &   339.18 \\ 
              GP  &    88.11 &     1.00 &  2926.17 &  5075.60 &  5847.59 &    33.21 &    57.61 &    66.37
 \end{tabular}
\caption{table.diabetes, N=392, P=9}
\end{table}

%-------------------------------------------------------------------------------
\begin{table}
\label{tab:blogit-heart}
\begin{tabular}{l r r r r r r r r } 
          Method  &     time &    ARate &  ESS.min &  ESS.med &  ESS.max &  ESR.min &  ESR.med &  ESR.max \\ 
              PG  &     5.56 &     1.00 &  2097.03 &  3526.82 &  4852.37 &   377.08 &   633.92 &   872.30 \\ 
           IndMH  &     2.24 &     0.39 &   589.64 &   744.86 &   920.85 &   263.63 &   333.19 &   413.03 \\ 
             RAM  &     1.98 &     0.30 &   862.60 &  1076.04 &  1275.22 &   436.51 &   543.95 &   645.13 \\ 
              OD  &     6.68 &     1.00 &   620.90 &  1094.27 &  1596.40 &    93.03 &   163.91 &   239.12 \\ 
              FS  &     6.50 &     1.00 &   558.95 &  1112.53 &  1573.88 &    85.92 &   171.04 &   241.96 \\ 
      dRUMAuxMix  &     5.97 &     1.00 &   604.60 &  1118.89 &  1523.84 &   101.33 &   187.49 &   255.38 \\ 
       dRUMIndMH  &     3.51 &     0.34 &   256.85 &   445.87 &   653.13 &    73.24 &   127.28 &   186.38 \\ 
  IndivdRUMIndMH  &     2.88 &     0.35 &   290.41 &   467.93 &   607.80 &   100.70 &   162.25 &   210.79 \\ 
         dRUMHAM  &     7.06 &     1.00 &   592.63 &  1133.59 &  1518.72 &    83.99 &   160.72 &   215.25 \\ 
              GP  &    65.53 &     1.00 &  1398.43 &  2807.09 &  4287.55 &    21.34 &    42.84 &    65.43
 \end{tabular}
\caption{table.heart, N=270, P=19}
\end{table}

%-------------------------------------------------------------------------------
\begin{table}
\label{tab:blogit-aus}
\begin{tabular}{l r r r r r r r r } 
          Method  &     time &    ARate &  ESS.min &  ESS.med &  ESS.max &  ESR.min &  ESR.med &  ESR.max \\ 
              PG  &    12.78 &     1.00 &   409.98 &  3841.02 &  5235.53 &    32.07 &   300.44 &   409.48 \\ 
           IndMH  &     3.42 &     0.22 &   211.48 &   414.87 &   480.02 &    61.89 &   121.53 &   140.59 \\ 
             RAM  &     3.92 &     0.00 &     8.27 &    10.08 &    26.95 &     2.11 &     2.57 &     6.87 \\ 
              OD  &    14.59 &     1.00 &    28.59 &   988.30 &  1784.77 &     1.96 &    67.73 &   122.33 \\ 
              FS  &    15.05 &     1.00 &    36.22 &  1043.69 &  1768.47 &     2.41 &    69.37 &   117.53 \\ 
      dRUMAuxMix  &    14.92 &     1.00 &    29.34 &   991.32 &  1764.40 &     1.97 &    66.44 &   118.27 \\ 
       dRUMIndMH  &     8.93 &     0.19 &    13.03 &   222.92 &   435.42 &     1.46 &    24.97 &    48.76 \\ 
  IndivdRUMIndMH  &     7.38 &     0.19 &    13.61 &   220.02 &   448.76 &     1.85 &    29.83 &    60.84 \\ 
         dRUMHAM  &    18.64 &     1.00 &    28.75 &  1040.74 &  1817.85 &     1.54 &    55.84 &    97.53 \\ 
              GP  &   162.73 &     1.00 &    95.81 &  2632.74 &  4757.04 &     0.59 &    16.18 &    29.23
 \end{tabular}
\caption{table.aus, N=690, P=35}
\end{table}

%-------------------------------------------------------------------------------
\begin{table}
\label{tab:blogit-ger.num}
\begin{tabular}{l r r r r r r r r } 
          Method  &     time &    ARate &  ESS.min &  ESS.med &  ESS.max &  ESR.min &  ESR.med &  ESR.max \\ 
              PG  &    15.37 &     1.00 &  3111.71 &  5893.15 &  6462.36 &   202.45 &   383.40 &   420.44 \\ 
           IndMH  &     3.58 &     0.68 &  2332.25 &  3340.54 &  3850.71 &   651.41 &   932.96 &  1075.47 \\ 
             RAM  &     4.17 &     0.43 &  1906.23 &  2348.20 &  2478.68 &   457.11 &   563.07 &   594.30 \\ 
              OD  &    17.32 &     1.00 &  1030.53 &  2226.92 &  2637.98 &    59.51 &   128.59 &   152.33 \\ 
              FS  &    18.21 &     1.00 &   957.05 &  2154.06 &  2503.09 &    52.55 &   118.27 &   137.43 \\ 
      dRUMAuxMix  &    18.13 &     1.00 &   955.41 &  2150.59 &  2533.40 &    52.68 &   118.60 &   139.70 \\ 
       dRUMIndMH  &    10.60 &     0.29 &   360.72 &   702.89 &   809.20 &    34.03 &    66.30 &    76.33 \\ 
  IndivdRUMIndMH  &     8.35 &     0.29 &   334.83 &   693.41 &   802.33 &    40.09 &    83.04 &    96.08 \\ 
         dRUMHAM  &    22.15 &     1.00 &   958.02 &  2137.13 &  2477.10 &    43.25 &    96.48 &   111.84 \\ 
              GP  &   223.80 &     1.00 &  2588.07 &  5317.57 &  6059.81 &    11.56 &    23.76 &    27.08
 \end{tabular}
\caption{table.ger.num, N=1000, P=25}
\end{table}

%-------------------------------------------------------------------------------
\begin{table}
\label{tab:blogit-ger}
\begin{tabular}{l r r r r r r r r } 
          Method  &     time &    ARate &  ESS.min &  ESS.med &  ESS.max &  ESR.min &  ESR.med &  ESR.max \\ 
              PG  &    22.30 &     1.00 &  2803.23 &  5748.30 &  6774.82 &   125.69 &   257.75 &   303.76 \\ 
           IndMH  &     4.72 &     0.41 &   730.34 &  1050.29 &  1236.55 &   154.73 &   222.70 &   262.05 \\ 
             RAM  &     6.02 &     0.00 &     5.49 &    14.40 &   235.50 &     0.91 &     2.39 &    39.13 \\ 
              OD  &    25.34 &     1.00 &   717.94 &  2153.05 &  2655.86 &    28.33 &    84.96 &   104.80 \\ 
              FS  &    26.44 &     1.00 &   727.17 &  2083.48 &  2554.62 &    27.50 &    78.80 &    96.62 \\ 
      dRUMAuxMix  &    26.91 &     1.00 &   755.31 &  2093.68 &  2562.11 &    28.06 &    77.80 &    95.21 \\ 
       dRUMIndMH  &    14.66 &     0.13 &   132.74 &   291.11 &   345.12 &     9.05 &    19.86 &    23.54 \\ 
  IndivdRUMIndMH  &    12.45 &     0.13 &   136.57 &   290.13 &   345.22 &    10.97 &    23.31 &    27.73 \\ 
         dRUMHAM  &    35.99 &     1.00 &   742.04 &  2075.41 &  2579.42 &    20.62 &    57.67 &    71.67 \\ 
              GP  &   243.41 &     1.00 &  2181.84 &  5353.41 &  6315.71 &     8.96 &    21.99 &    25.95
 \end{tabular}
\caption{table.ger, N=1000, P=49}
\end{table}

%%% Local Variables:
%%% TeX-master: benchmarks.tex
%%% End:

\pagebreak

\section{Binary Logistic Mixed Model Data Sets}

\begin{outline}

\1 Synthetic:

A synthetically generated dataset with 5 groups, 100 observations within each
group, and a single fixed effect.

\1 Polls:

Voting data from a recent Presidential campaign.  The response indicates a vote
for or against former President George W. Bush.  There are 49 groups
corresponding to states.  Some states have few observations, necessitating a
model that shrinks coefficients towards a global mean to get reasonable
estimates.  A single fixed effect corresponding to race is included.  Entries
with missing data were deleted to yield a total of 2015 observations.

\1 Xerop:

The Xerop data set from the \texttt{epicalc} R package \citep{epicalc-2012}.
Indonesian children were observed to examine the causes of respiratory
infections; of specific interest is whether vitamin A deficiencies cause such
illness.  Multiple observations of each individual were made.  The data is
grouped by individual id yielding a total of 275 random intercepts.  A total of
5 fixed effects are included in the model---age, sex, height, stunted growth,
and season---corresponding to an 8 dimensional regression coefficient after
expanding the season covariate using dummy variables.

\end{outline}

\section{Binary Logistic Mixed Model Discussion}
\label{sec:logitmm-discussion}

While an independence Metropolis sampler works well for binary logistic
regression, it does not work well for some of the mixed models we consider.
Presumably, part of the reason it works well for logistic regression is because
the posterior distribution is well approximated by the Laplace approximation.
In the mixed model, this approximation performs adequately for the synthetic
data set; however, that data set has 100 observations per group, which is a
rather large amount of data given only a random intercept and single fixed
effect.  The posterior mode roughly coincides with the posterior mean in this
case.
% PGMM2 mean (10000) : 
% alpha: -0.6713863 -1.3605894  0.7193656 -0.6476419 -0.7775381
% beta: 1.1156864 
% m: -0.5483927
% ph: 1.569225 
% 
% MAP: 
% alpha: -0.6633656 -1.3470510  0.7239056 -0.6401559 -0.7692829  
% beta: 1.0992942
% m: -0.5391792
% phi: 1.3867589
In the case of the larger, more complex mixed models considered this is no
longer the case.  For instance, in the polls data set, one finds at least two
heuristics that suggest the Laplace approximation will be a poor proposal.
First, the posterior mode does not coincide with the posterior mean.  Second,
the Hessian at the mode is nearly singular.  Its smallest eigenvalue, in
absolute terms, corresponds to an eigenvector that points predominantly in the
direction of $\phi$.  Thus, there is a great deal of uncertainty in the
posterior mode of $\phi$.  If we iteratively solve for the MLE by starting at
the posterior mean, or if we start at the posterior mode for all the coordinates
except $\phi$, which we initialize at the posterior mean of $\phi$, then we
arrive at the same end point, suggesting that behavior is not due to a poor
choice of initial value or a poor stopping rule.

The first image in Figure \ref{fig:phi-mm} shows that the difference between the
posterior mode and posterior mean is, by far, greatest in the $\phi$ coordinate.
The second image in Figure \ref{fig:phi-mm} provides one example of the lack of
curvature in $\phi$ at the mode.  If one plots $\phi$ against the other
coordinates, then one sees a similar, though often less extreme, picture.  In
general, large values of $\phi$ are found at the tip of an isosceles triangular
whose base runs parallel to the coordinate that is not $\phi$.  While the upper
tip of the triangle may posses the most likely posterior values, the rest of the
posterior does not fall away quick enough to make that a likely posterior random
variate.

\begin{figure}
\label{fig:phi-mm}
\centering
\includegraphics[scale=0.45]{Images/polls-mean-mode.png}
\includegraphics[scale=0.45]{Images/logpost.png}

\includegraphics[scale=0.45]{Images/hist-phi.png}
\includegraphics[scale=0.45]{Images/scatterplot-phi-m.png}

\caption{Proceeding from left to right and top to bottom.  Upper left: the
  posterior mode and the posterior mean of $(\alpha, \beta, m \phi)$.  The mode
  and mean are most different in $\phi$.  Upper right: the level sets of $(\phi,
  m)$ of the log posterior when the other coordinates are evaluated at the
  posterior mode.  The log posterior is very flat when moving along $\phi$.
  Bottom right: the marginal posterior distribution of $\phi$.  When
  marginalizing, one finds that few large values of $\phi$ are likely.  Bottom
  right: a scatter plot of posterior samples of $(\phi, m)$. Again, one sees
  that upon marginalizing out the other coordinates the posterior mass is
  concentrated at relatively small values of $\phi$ compared to its value at the
  posterior mode.}
\end{figure}

% If you have a bibliography.
% The file withe bibliography is name.bib.
\bibliographystyle{abbrvnat}
\bibliography{bayeslogit}{}

\end{document}

%%% Local Variables:
%%% TeX-master: t
%%% End: